\section{Hinweise zur Softwareentwicklung}
\subsection{cvs-zugang}

 Der Zugang zum cvs--Server des Mycore Projekts f�r Entwickler erfolgt nach
 Freischaltung eines Accounts\footnote{Dieser kann �ber Frank L�tzenkirchen
   \url{luetzenkirchen@bibl.uni-essen.de} beantragt werden} �ber ssh. Dem
 Freischalten sind folgende Umgebungsvariablen zu setzten:

\begin{verbatim}
CVS_RSH=ssh
CVSROOT=:ext:mcr_username@server.mycore.de:/cvs
export CVS_RSH CVSROOT
\end{verbatim}

Es empfielt sich zuerst die MyCoRe Quellen herunterzuladen. 
\begin{verbatim}
cvs -d :ext:mcr_username@server.mycore.de:/cvs checkout mycore

\end{verbatim}

Danach k�nnen sie mit 

\begin{verbatim}
cvs -d :ext:mcr_username@server.mycore.de:/cvs commit -m "Kommentar f�rs
CVS" 'file' 

\end{verbatim}

die entsprechende Datei commiten.

Soll in ein bestehendes Projekt eine neue Datei integriert werden, so legt man
sie zun�chst lokal im vorgesehenen (und bereits ausgecheckten!) Verzeichnis
an. Dann merkt man sie mittels \verb+cvs add <filename>+ vor. Um sie dann
global zu registrieren, erfolgt ein (verk�rzt): \verb+cvs commit <filename>+
Neue Unterverzeichnisse werden auf die gleiche Weise angelegt.

Weitere und sehr ausf�hrliche Informationen gibt es zu Hauf im Internet.


\url{http://www.cvshome.org/docs/}
\url{http://cvsbook.red-bean.com/translations/german/}
\url{http://panama.informatik.uni-freiburg.de/~oberdiek/documents/OpenSourceDevWithCVS.pdf}
\url{http://sourceforge.net/docman/?group_id=1}
\url{http://www.jspwiki.org/Wiki.jsp?page=CVSTips}
\url{http://www.selflinux.org/selflinux/pdf/cvs_buch_kapitel_9.pdf}

\section{Entwicklungsumgebung}

\section{Tools}



%%% Local Variables: 
%%% mode: latex
%%% TeX-master: "ProgGuide"
%%% TeX-master: "ProgGuide"
%%% End: 
