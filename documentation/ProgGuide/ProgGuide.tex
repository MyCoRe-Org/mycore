\documentclass[a4paper,12pt]{report}
\usepackage{ngerman}

\usepackage[latin1]{inputenc} %f�r Linux
%\usepackage[ansinew]{inputenc} %fuer Windows

\usepackage{mycore}

\begin{document}
%-------   Vorspann
\title{MyCoRe Internal Design Guide}
\author{
Johannes B�hler
            }
\maketitle

\setcounter{secnumdepth}{4} 


\chapter*{Vorwort} 

\tableofcontents
\listoffigures
\listoftables

\chapter{Softwareentwicklung}
\section{Hinweise zur Softwareentwicklung}
\subsection{cvs-zugang}

 Der Zugang zum cvs--Server des Mycore Projekts f�r Entwickler erfolgt nach
 Freischaltung eines Accounts\footnote{Dieser kann �ber Frank L�tzenkirchen
   \url{luetzenkirchen@bibl.uni-essen.de} beantragt werden} �ber ssh. Dem
 Freischalten sind folgende Umgebungsvariablen zu setzten:

\begin{verbatim}
CVS_RSH=ssh
CVSROOT=:ext:mcr_username@server.mycore.de:/cvs
export CVS_RSH CVSROOT
\end{verbatim}

Es empfielt sich zuerst die MyCoRe Quellen herunterzuladen. 
\begin{verbatim}
cvs -d :ext:mcr_username@server.mycore.de:/cvs checkout mycore

\end{verbatim}

Danach k�nnen sie mit 

\begin{verbatim}
cvs -d :ext:mcr_username@server.mycore.de:/cvs commit -m "Kommentar f�rs
CVS" 'file' 

\end{verbatim}

die entsprechende Datei commiten.

Soll in ein bestehendes Projekt eine neue Datei integriert werden, so legt man
sie zun�chst lokal im vorgesehenen (und bereits ausgecheckten!) Verzeichnis
an. Dann merkt man sie mittels \verb+cvs add <filename>+ vor. Um sie dann
global zu registrieren, erfolgt ein (verk�rzt): \verb+cvs commit <filename>+
Neue Unterverzeichnisse werden auf die gleiche Weise angelegt.

Weitere und sehr ausf�hrliche Informationen gibt es zu Hauf im Internet.


\url{http://www.cvshome.org/docs/}
\url{http://cvsbook.red-bean.com/translations/german/}
\url{http://panama.informatik.uni-freiburg.de/~oberdiek/documents/OpenSourceDevWithCVS.pdf}
\url{http://sourceforge.net/docman/?group_id=1}
\url{http://www.jspwiki.org/Wiki.jsp?page=CVSTips}
\url{http://www.selflinux.org/selflinux/pdf/cvs_buch_kapitel_9.pdf}

\section{Entwicklungsumgebung}

\section{Tools}



%%% Local Variables: 
%%% mode: latex
%%% TeX-master: "ProgGuide"
%%% TeX-master: "ProgGuide"
%%% End: 


\chapter{ Einf"uhrung und Grundprinzipien} 
\section{XML/XSLT}
\section{XPath/Queries}
\section{Sessionmodell}
\section{Das Vererbungsmodell}
\section{Das API-Konzept allgemein}
\section{Klassen und Verantwortlichkeiten }
\section{Allgemeine Klassen / Exception-Modell / MCRCache}
\section{Das Metadatenmodell}
\subsection{MCRClassification}
\subsection{MCRObject}
\subsection{MCRDerivate}
\subsection{MCRLinkManager}
\subsection{Erweiterungsm"oglichkeiten}
\pagebreak
Seitenumbruch erzwingen?
\section{Das IFS Modell (ohne Stores)}
\section{Das User- und ACL-Modell}
\section{Der Backend-Store}
\subsection{CM7}
\subsection{CM8.2 (inklusive Datenspeichermodell und Query Umsetztung)}
\subsection{XMLDB (inklusive Datenspeichermodell und Query Umsetztung)}
\subsection{DB2 / MySQL / Oracle}
\subsection{Filesystem-Store}
\subsection{VideoStores}
\subsection{RemoteStore im Detail}
\subsection{Weitere Ausbaum�glichkeiten}
\section{Die Frontend Komponenten}
\subsection{Das Commandline-Tool und seine Erweiterungen}
\subsection{Zusammenspiel der Servlets und Funktion der einzelnen}
\subsection{Login-Servlet und MCRSession}
\subsection{IFS-Servlet}
\subsection{Query-Servlet}
\subsection{Innere Struktur des Editor-Servlets}
\subsection{Innere Struktur des User-Servlets}
\section{Funktionsprinzipen der Services}
\subsection{OAI}
\subsection{NBN}
\subsection{Weitere}
\chapter{Konfiguration im Detail}
\section{Mycore.properties}
\section{Suchmasken}
\section{Stylesheets}
% Glossar
%
% UserGuide - Glossar
%
\chapter*{Glossar}
{\bf NBN} \\[1.5ex]
Was ist eigentlich NBN??? \\[2ex]
{\bf OAI} \\[1.5ex]
Was ist eigentlich OAI??? \\[2ex]
{\bf XML} \\[1.5ex]
Was ist eigentlich XML??? \\[2ex]
{\bf XSLT} \\[1.5ex]
Was ist eigentlich XSLT??? \\[2ex]
%
%

\end{document}




