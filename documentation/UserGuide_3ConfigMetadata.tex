%
%
\subsection{Das Metadatensystem}
%
%
In diesem Abschnitt son nun beschrieben werden, welche Konfigurationen f�r die Speicherung der Metadaten zu bearbeiten sind. Als erstes m�ssen Sie einstellen, mit welchem Persistence-Store f�r die Metadaten Sie arbeiten wollen. Das h�ngt nat�rlich auch von den bisherigen Installationsschritten ab und wurde schon behandelt. Nun folgen die Einstellung f�r den konkreten Layer.
\subsubsection{Die Nutzung von eXist als XML:DB Backend}
Hierzu m�ssen Sie die folgenden Daten im Konfigurationsverzeichnis des Samples unter {\it \$MYCORE\_SAMPLE\_HOME/config/mycore.properties.private} entsprechend Ihren Gegebenheiten anpassen. Habe Sie die Standardvorgehensweise befolgt, sollte hier nichts zu tun sein.
\begin{verbatim}
MCR.persistence_xmldb_driver=org.exist.xmldb.DatabaseImpl
MCR.persistence_xmldb_database_url=xmldb:exist://localhost:8081/db/mycore
MCR.persistence_xmldb_database=exist
\end{verbatim}
Weiterhin m�ssen die Files {\it exist.jar}, {\it xmldb.jar} und {\it xmlrpc-1.1.jar} nach {\it \$MYCORE\_HOME/lib} kopiert werden. Achten Sie darauf, dass es in diesem verzeichnis kein File namens {\it xindice.jar} oder ein *.jar File einer anderen XML:DB gibt.\\[2ex]
Starten Sie nun den eXist-Client und f�hre Sie folgende Kommandos zum anlegen der Stores unter eXist aus:
\begin{verbatim}
mkcol mycore
chown guest guest mycore
cd mycore
mkcol legalentity
chown guest guest legalentity
mkcol document
chown guest guest document
mkcol derivate
chown guest guest
quit
\end{verbatim}
Nun sollte das Sample zum Laden der Metadaten bereit sein.
%
%
\subsubsection{Lade der Daten}
habe Sie alles Vorbereitet, so k�nnen nun die Beispiel-Metadaten geladen werden. Dies geschieht mit
\begin{itemize}
\item {\tt ant legal} oder {\tt bin/build.sh legal} f�r die Legalentities und dann
\item{\tt ant document} oder {\tt bin/build.sh document} f�r die Documents
\end{itemize}

