\documentclass[a4paper,12pt]{book}
\usepackage{german}
\usepackage{fancyheadings}
\pagestyle{fancyplain}
\addtolength{\headwidth}{\marginparsep}
\addtolength{\headwidth}{\marginparwidth}
\renewcommand{\chaptermark}[1]%
      {\markboth{#1}{}}
\renewcommand{\sectionmark}[1]%
      {\markright{\thesection\ #1}}
\lhead[\fancyplain{}{\bfseries\thepage}]%
      {\fancyplain{}{\bfseries\rightmark}}
\rhead[\fancyplain{}{\bfseries\leftmark}]%
      {\fancyplain{}{\bfseries\thepage}}
\cfoot{}
\begin{document}
%-------   Vorspann
\title{MyCoRe User Guide}
\author{
    Frank L"utzenkirchen\\
    Jens Kupferschmidt\\
    Detlef Degenhardt\\
    Ulrike Kr"onert}
\maketitle
\setcounter{secnumdepth}{10}
\chapter*{Vorwort}
In diesem Dokument sind alle Arbeiten zum Start der Beispielanwendung und zur Gestaltung eigener Anwendungen beschrieben. Teilweise wird auch auf das MyCoRe Design Guide verwiesen.
\tableofcontents
\listoffigures
\listoftables
%-------    Hauptteil
\chapter{Voraussetzungen f"ur eine MyCoRe Anwendung}
Vor- und Nachteile der einzelnen Basissysteme [IBM / Linux / Sun] 
\section{Hinweise zur Bereitstellung der einzelnen Systeme (Quick Installation der wichtigen Komponenten und Test)}
\section{Hinweise zur Nutzung m"oglicher Webserver und Servlet-Maschinen und deren Konfiguration}
% Kapitel 2
\clearpage %
%
\chapter{Download und Installation des MyCoRe Kerns}
%
%
\section{Download des MyCoRe Kerns}
%
%
Der MyCoRe Kern wird f"ur alle unterst"utzten Systeme "uber das CVS Repository
ausgeliefert. Das Holen der aktuellen Version erfolgt mit dem Kommando
\begin{center}
{\tt cvs -d :pserver:anoncvs@server.mycore.de:/cvs checkout mycore}
\end{center}
Nach dem erfolgreichen Checkout erhalten Sie folgende Dateistruktur:\\[2ex]
\bottomcaption{Dateistruktur des MyCoRe Kernes}
\tablehead{\hline}
\tabletail{\hline}
\begin{supertabular}{|p{5cm}|p{10cm}|}
\hline
{\bf mycore} &  Das Root-Verzeichnis des MyCoRe-Kerns \\
\quad {\bf bin} & Das Verzeichnis der Shellscripte \\
\qquad build.sh & Startet den Build-Prozess unter einem UNIX-System \\
\qquad build.cmd & Startet den Build-Prozess unter einem Windows-System \\
\qquad build.properties & Konfiguration Pfade und Libraries der verwendeten Datenbanken \\ 
\quad {\bf documentation} & Dokumentationen zu MyCoRe \\
\quad {\bf lib} & Notwendige zus"atzliche Java-Bibliotheken \\
\quad {\bf schema} & XMLSchema Dateien, die anwendungsunabh"angig sind \\
\quad {\bf stylesheets} & Intern verwendete, anwendungsunabh"angige XSL Stylesheets \\
\quad {\bf sources/org/mycore} & Die Wurzel des MyCoRe-Source-Baumes \\
\qquad {\bf datamodel} & Klassen zum Datenmodell \\
\quad \qquad {\bf classifications} & Klassen zur Arbeit mit den Klassifikationen \\
\quad \qquad {\bf ifs} & Klassen zur Arbeit mit dem Internal File System \\
\quad \qquad {\bf metadata} & Klassen zur Arbeit mit den Metdaten \\
\qquad {\bf common} & Klassen, die im gesamten Projekt ben�tigt werden \\
\quad \qquad {\bf xml} & Allgemeine Klassen zur XML Verarbeitung \\
\qquad {\bf backend} & Klassen f"ur die verschiedenen Data Stores \\
\quad \qquad {\bf cm7} & Klassen zur Nutzung des IBM Content Manager 7 \footnote{Die Klassen f�r den IBM Content Manager 7 werden nicht mehr weiterentwickelt!} \\
\quad \qquad {\bf cm8} & Klassen zur Nutzung des IBM Content Manager 8.2 \\
\quad \qquad {\bf filesystem} & Klassen zum Content Store im lokalen Filesystem \\
\quad \qquad {\bf realhelix} & Klassen zum Content Store in einem Helix-Server \\
\quad \qquad {\bf remote} & Klassen zum Zugriff auf Remote-MyCoRe-Daten \\
\quad \qquad {\bf sql} & Klassen zum Zugriff auf relationale Datenbanken mittels SQL-Standart \\
\quad \qquad {\bf videocharger} & Klassen zum Content Store in einen IBM Videocharger \\
\quad \qquad {\bf xmldb} & Klassen zur Speicherung der Daten mittels einer XML:DB \\
\qquad {\bf frontend} & Klassen f"ur die Frontends des MyCoRe-Systems \\
\quad \qquad {\bf cli} & Klassen des Commandline-Tools \\
\quad \qquad {\bf editor2} & Klassen zur Gestaltung von Editoren \\
\quad \qquad {\bf servlets} & Klassen zu Gestaltung von Servlets \\
\qquad {\bf services} & Klassen f"ur weiterf"uhrende Services des MyCoRe-Projektes \\
\quad \qquad {\bf nbn} & Klassen zur Arbeit mit NBN's \\
\quad \qquad {\bf oai} & Klassen zur Arbeit mit OAI Komponenten \\
\quad \qquad {\bf query} & Klassen zur Arbeit mit dem MyCoRe internen Query-System \\
\qquad {\bf user} & Klassen des User- und Rechteverwaltungssystems \\
\quad build.xml & Ant Build-Datei, steuert den MyCoRe Build-Prozess \\
\quad license.txt & Das Lizenz-File des MyCoRe-Projektes, bitte lesen Sie dieses File aufmerksam durch, bevor Sie MyCore einsetzen. \\
\hline
\end{supertabular}
%
%
\section{Konfiguration und "Ubersetzten des Kerns}
%
\begin{enumerate}
\item
MyCoRe verwendet das Apache Ant Build-Tool, um den Quellcode zu �bersetzen und eine vollst�ndige 
Beispiel-Applikation zu erzeugen. Entsprechend der Installationsanleitung des Ant-Paketes sollten Sie zun�chst die
Umgebungsvariable {\tt JAVA\_HOME} und {\tt ANT\_HOME} gesetzt haben. Sollten diese Variablen auf Ihrem System noch nicht
gesetzt sein, k�nnen Sie dies in der Datei {\tt build.sh} (Unix) bzw. {\tt build.cmd} (Windows) nachholen und korrigieren.

\item
Es ist nicht n�tig, weitere Umgebungsvariablen wie etwa den Java CLASSPATH zu setzen. Das MyCoRe Ant Build-Skript 
ignoriert den lokal gesetzten {\tt CLASSPATH} v�llig und generiert stattdessen einen eigenen {\tt CLASSPATH} entsprechend Ihrer Konfiguration. 
Somit k�nnen wir sicherstellen, dass nur die erforderlichen Pakete und Klassen in der richtigen Version verwendet werden. Die Konfiguration der 
Systemumgebung der verwendeten Datenbanken f�r die XML-Speicherung (IBM CM7, IBM CM8, Apache Xindice, eXist) und die Speicherung von Tabellen
�ber JDBC in einer relationalen Datenbank (IBM DB2, MySQL, optional auch andere) wird in der Datei {\tt bin/build.properties}
festgelegt. 

\item
In der Regel werden Sie nur die beiden entsprechenden Bl�cke f�r die verwendete XML-Datenbank ({\tt MCR.XMLStore.*}) und die
verwendete relationale Datenbank ({\tt MCR.JDBCStore.*}) durch kommentieren bzw. auskommentieren der vorgegebenen Zeilen und 
Anpassen der beiden Variablen {\tt MCR.XMLStore.BaseDir} und \\
{\tt MCR.JDBCStore.BaseDir} an die lokalen Installationsverzeichnisse
Ihrer Datenbanksysteme anpassen m�ssen. Die weiteren Variablen steuern die f�r den Betrieb notwendigen JAR-Dateien 
({\tt MCR.*Store.Jars}), eventuell zus�tzlich in den CLASSPATH einzubindende class-Dateien oder Ressourcen ({\tt MCR.*Store.ClassesDirs})
und zur Laufzeit erforderliche native Libraries bzw. DLLs ({\tt MCR.*Store.LibPath}). Passen Sie die Werte entsprechend der
Dokumentation Ihres Datenbankproduktes und der Kommentare in der Datei selbst an.

\item
Sie sollten zun�chst pr�fen, ob ihre Systemumgebung korrekt eingerichtet ist, indem Sie 
\begin{center}
{\tt build.sh info } \qquad bzw. \qquad {\tt build.cmd info }
\end{center}
ausf�hren. Das Ant Build Tool zeigt Ihnen daraufhin die verwendeten JDK- und Ant-Software-Versionen und den
generierten CLASSPATH und LIBPATH (f�r Unix Systeme) an.

\item Sollten Sie festgestellt haben, dass Ihr JDK ab 1.4.x eine andere
Xalan-Version benutzt, f�hren Sie bitte folgende Kommandos aus und pr�fen Sie danach Ihr System erneut.

{\tt
cd \$JAVA\_HOME/jre/lib \newline
mkdir endorsed \newline
cd endorsed \newline
cp \$MYCORE\_HOME/lib/xerces* . \newline
cp \$MYCORE\_HOME/lib/xalan* . \newline
cd \$JAVA\_HOME/lib \newline
ln -s ../jre/lib/endorsed endorsed
}

\item 
Eine �bersicht �ber alle wesentlichen Build-Ziele erhalten Sie mit
\begin{center}
{\tt build.sh usage } \qquad bzw. \qquad {\tt build.cmd usage } 
\end{center}

\item
�bersetzen Sie alle MyCoRe Quellcode-Dateien mit dem Befehl
\begin{center}
{\tt build.sh jar } \qquad bzw. \qquad {\tt build.cmd jar }
\end{center}
Dabei entsteht, abh�ngig von dem von Ihnen gew�hlten Datenbank-System zur Speicherung der XML-Daten eine Jar-Datei 
{\tt lib/mycore-for-[cm7|cm8|xmldb].jar}.

\item
Optional k�nnen Sie auch JavaDoc Quellcode-Dokumentation im HTML-Format generieren lassen, indem Sie
\begin{center}
{\tt build.sh javadocs } \qquad bzw. \qquad {\tt build.cmd javadocs }
\end{center}
aufrufen. Dabei entstehen HTML-Dateien im Verzeichnis {\tt documentation/html}.

\end{enumerate} 
%
%

% Kapitel 3
\chapter{Die MyCore Beispielanwendung}
\section{Grundlegender Aufbau und Ziel der Beispielanwendung}
\section{Vereinfachte Funktionsprinzipien der Anwendung}
\subsection{User- und Rechtesystem}
\subsection{Klassifikationen}
\subsection{Metadatenmodel}
\subsection{IFS und Content Store}
\subsection{Datenpr"asentation}
\subsection{Interaktive Arbeit mit den Daten}
\section{Download der Beispielanwendung}
\section{Konfiguration zur Arbeit mit den Beispieldaten}
Wenn man nur sections und subsections aneinanderf"ugt, wird offensichtlich kein Seitenumbruch gemacht??
\subsection{Grundlegende Konfigurationen (JDBC, CM, Logger, usw.)}
\subsection{Das User- und Rechtesystem  + Laden der Beispieldaten}
\subsection{Das Klassifikationsmodel + Laden der Beispieldaten}
\subsection{Das Metadatenmodel + Laden der Beispieldaten}
\subsection{Das IFS + Laden der Beispieldaten}
\section{Arbeiten mit der Kommandzeilen-Shell mycore.sh}
\subsection{"Ubersicht der Kommando und Beispiele}
\section{Arbeiten mit der Web-Anwendung}
\subsection{Konfiguration und Start der Webanwendung }
\subsubsection{Apache / Tomcat}
\subsubsection{Apache / Websphere}
\subsection{Das Layout-Setvlet und das Zusammenspiel der Servlets untereinander}
\subsection{Die Nutzung des Editor-Servlets}
\section{Zusammenarbeit mit anderen Installationen (Remote)}
\section{Vom Sample zum eigenen Dokumentserver}
Hier sollen Anpassungsschritte detailliert erkl"art werden.
\subsection{Anpassungen des Layout an eigene Bed�rfnisse}
\subsection{Weitere User und Gruppen / Nutzung zur selektierten Darstellung durch Stylesheets}
\subsection{Metadatenvererbung}
\subsection{Nutzung der OAI Schnittstelle }
\subsection{Arbeiten mit NBN}
\subsection{M"ogliche Workflow-Szenarien im Bibliotheksumfeld}
\section{Einbindung weiterer Content Stores (Helix \& Co.)}
\section{Hints \& Tips / Trobleshooting}
\chapter{Erstellen einer eigenen Anwendung auf Basis des MyCoRe-Kernes}
(am Beispiel des Papyrus-Projektes)
\section{Erforderliche Schritte zu einer Sammlungsanwendung mit eigenen Metadaten}
\begin{flushright}
{\em macht Jens}
\end{flushright}
\chapter{Weitere Anwendungen (Archivl"osung)}
\end{document}
