\chapter{Stilfestlegungen}
\section{Schriften}
Um den MyCoRe-Dokumentationen ein einheitliches Aussehen zu verleihen, werden folgende Festlegungen zu Nutzung der verschiedenen Schriftarten getroffen:\\
\begin{enumerate}
\item F�r auszuf�hrende Kommandos ist der Stil Typewriter \begin{verbatim} {\tt ...}\end{verbatim} zu verwenden.
\item F�r Datei- und Pfadnamen ist der Stil Italic \begin{verbatim} {\it ...}\end{verbatim} zu verwenden.
\item F�r besonders zu markierende Namen und Bezeichner ist der Stil Bold \begin{verbatim} {\bf ...}\end{verbatim} zu verwenden.
\item Zur Darstellung von Ausschnitten aus Source-Codes oder XML-Files k�nnen diese Daten in eine {\bf begin} - {\bf end} Block mit dem Bezeichner {\bf verbatim} eingeschlossen werden. So wird der eingeschlossene Text mit allen Zeichen und Zeilenumbr�chen 1:1 dargestellt.
\item Deutsche Umlaute k�nnen im Text als �, �, � ... stehen und m�ssen nicht gesondert codiert werden.
\end{enumerate}
