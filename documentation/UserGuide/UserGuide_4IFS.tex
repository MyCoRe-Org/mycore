%
%
\subsection{Das Speichermodell f�r die Multimediadaten (IFS)}
%
%
Im bisherigen Verlauf dieses Kapitels wurden nur die beschreibenden Daten des multimedialen Objektes erl�utert. Dieser Abschnitt besch�ftigt sich damit, wie die eigentlichen Objekte dem Gesamtsystem hinzugef�gt werden k�nnen. \\[2ex]
Im MyCoRe Projekt wurde zur Ablage der digitalen Objekte das Konzept des {\bf IFS} entwickelt. Hier ist es m�glich, �ber spezielle Konfigurationen festzulegen, in welchen Speicher (Store) die einzelnen Dateien gespeichert werden sollen. Eine genaue Beschreibung der M�glichkeiten finden Sie im Kapitel 4. \\[2ex]
Das Laden von Objekten erfolgt mittels einer Metadaten-Datei, welche alle Informationen �ber die zu speichernde(n) Datei(en) und ihre Beziehung(en) zu den Metadaten enth�lt. Die zu speichernden multimedialen Objekte werden im Weiteren als \mcridentifier{Derivate}, also Abk�mmlinge, bezeichnet, da ein Objekt in mehreren Formen, \zB Grafikformaten, auftreten kann. Die Struktur der XML-Datei f�r Derivate ist fest vorgegeben, alle Felder, die nutzerseitig ge�ndert werden k�nnen, sind unten beschrieben.\\[2ex]
\lstset{language=XML,fancyvrb=true,frame=btlr,breaklines,prebreak={\space\MyHookSign}}
\begin{lstlisting}[caption=XML-Syntax eines Derivate-Objektes,label=lst:xml_syntax_derivateobjekt]
 <?xml version="1.0" cncoding="ISO-8859-1" ?>
 <mycorederivate
  xmlns:xsi="http://www.w3.org/2001/XMLSchema-instance"
  xsi:noNamespaceSchemaLocation="....xsd"
  xmlns:xlink="http://www.w3.org/1999/xlink"
  ID="..."
  label="..."
  >
  <derivate>
   <linkmetas class="MCRMetaLinkID">
    <linkmeta xlink:type="locator" xlink:href="..." />
   </linkmetas>
   <internals class="MCRMetaIFS">
    <internal
     sourcepath="..."
     maindoc="..." 
     />
   </internals>
  </derivate>
  <service>
   ...
  </service>
 </mycoreobject>
\end{lstlisting}

F�r \mcridentifier{xsi:noNamespaceSchemaLocation} ist die entsprechende XMLSchema-Datei anzugeben (\zB  \mcrfile{Derivate.xsd})\\[2ex]
Die \mcridentifier{ID} ist die eindeutige MCRObjectID.\\[2ex]
Der \mcridentifier{label} ist ein kurzer Text-String, der bei administrativen Arbeiten an der Datenbasis das Identifizieren einzelner Datens�tze erleichtern soll. Er kann maximal 256 Zeichen lang sein.\\[2ex]
Die Referenz in \mcridentifier{linkmeta} ist die MCRObjectID des Metadatensatzes, an den das/die Objekte angeh�ngt werden sollen.\\[2ex]
Das Attribut \mcridentifier{sourcepath} enth�lt die Pfadangabe zu einer Datei oder zu einem Verzeichnis, welches als Quelle dienen soll. Aus diesen Dateien kann nun eine Datei ausgew�hlt werden, welche den Einstiegspunkt \zB bei HTML-Seiten darstellen soll. Bei einzelnen Bildern ist hier noch einmal der Dateiname anzugeben. Ist nichts angegeben, so wird versucht Dateien wie index.html usw. zu finden.\\[2ex]

