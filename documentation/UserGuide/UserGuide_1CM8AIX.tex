%
%
\section{Hinweise zur Installation des IBM Content Manager 8.2}
%
%
\subsection{Der IBM Content Manager unter AIX}
An dieser Stelle soll eine Kurzbeschreibung der Installation des IBM Content Managers 8.2 f"ur AIX von Holger K"onig, IBM Deutschland GmbH, wiedergegeben werden. \\[2ex]
\subsubsection{Vorbereitung}
\begin{enumerate}
\item Installieren Sie das AIX Betriebssystem mit dem Release 4.3.3 ML 10, 5.1 ML 01 oder 5.2.
\item Sorgen Sie daf"ur, dass 'Cultural Conversion' und 'Language' auf English US eingestellt ist.
\item Aktivieren Sie die Netzanbindung inklusive DNS.
\item F"ur die Betriebssystem-Releases 4.3.3 und 5.1 muss Java 1.3.1 entsprechend der Anleitung installiert werden. Erweitern Sie in {\it /etc/environment }  die {\it PATH} Variable um {\it /usr/java131/jre/bin} und {\it /usr/java131/bin}. Wenn Sie auch das Paket {\bf Java131.ext.java3d } mit installieren wollen, m"ussen Sie vorher die Pakete \linebreak[4]
{\bf OpenGL.OpenGL\_X.adt} und {\bf OpenGL.OpenGL\_X.rte} installiert haben.
\item Installieren Sie den VAC Compiler Version 5.x oder 6.0 entsprechend der Anleitung und tragen Sie den Suchpfad unter {\it PATH} im File {\it /etc/environment } mit ein.
\item Aktivieren Sie das Lizenzsystem {\bf ifor} und tragen Sie sie Compilerlizenzen ein.
\end{enumerate}
%
\subsubsection{DB2}
\begin{enumerate}
\item Legen Sie bitte die nachfolgenden Benutzer an. Zusammen sollten je nach geplanten Anwendungen 4-8 GB Plattenplatz eingeplant werden. Achten Sie darauf, dass man sich in die Accounts einloggen kann, z. B. via Telnet!
\begin{itemize}
\item User {\bf dasusr1} $\rightarrow$ Group {\bf dasadm1}
\item User {\bf db2inst1} $\rightarrow$ Group {\bf db2grp1}
\item User {\bf db2fenc1} $\rightarrow$ Group {\bf db2fgrp1}
\end{itemize}
\item Kopieren Sie das File {\it ese.sbcs.tar.Z} von der CD {\bf 'DB2 8.1 with FP1'} und entpacken Sie dieses.
\item {\tt ./db2setup}
\item W"ahlen Sie {\bf Install Products} $\rightarrow$ {\bf DB2 UDB Enterprise Server Edition}. Folgen Sie den Schritten:
\begin{itemize}
\item {\bf Next}
\item {\bf Accept License}
\item Auswahl 'Custom' $\rightarrow$ {\bf Next}
\item Auswahl 'Install DB2 UDB Enterprise Server Edition on this computer' $\rightarrow$ {\bf Next}
\item Standartwerte lassn, 'Appliction Development Tools' zus"atzlich ausw"ahlen
\item Sprache 'Englisch' beibehalten $\rightarrow$ {\bf Next}
\item DAS User : Standartwert {\bf db2as} wenn m"oglich beibehalten, Password setzen $\rightarrow$ {\bf Next}
\item Erzeugen der DB2 Instanz durch Auswahl 'Create a DB2 instance - 32 bit' $\rightarrow$ {\bf Next}
\item Auswahl 'Single-partition instance' $\rightarrow$ {\bf Next}
\item Eintrag des DB2 Instance owner : Standartwert ({\bf db2inst1}) m"oglichst lassen, Password setzten $\rightarrow$ {\bf Next} \footnote{\label{fn1} Achten Sie darauf, keine exotischen Sonderzeichen zu nehmen, das macht im CM Probleme!}
\item Eintrag des DB2 Fenced users : Standartwert ({\bf db2fenc1}) m"oglichst lassen, Password setzten $\rightarrow$ {\bf Next} %\footnotemark[\ref{fn1}]
\item Instance TCPIP : Auswahl 'Configure' $\rightarrow$ Service Name : db2c\_dv2inst1 $\rightarrow$ Port 50000 $\rightarrow$ {\bf Next}
\item Instance properties $\rightarrow$ Authentication Type : Server $\rightarrow$ beibehalten 'Autostart the instance at system startup' $\rightarrow$ {\bf Next}
\item Prepare the DB2 tools catalog  beibehalten 'Do not prepare the DB2 tools catalog on this computer' $\rightarrow$ {\bf Next}
\item Administrator contact : Standart beibehalten $\rightarrow$ {\bf Next} \footnote{Warnung ignorieren}
\item Contact : Standart beibehalten ({\bf db2inst1}) $\rightarrow$ {\bf Next}
\item Summary $\rightarrow$ {\bf Finish}
\item Warten (dauert etwas)
\item Setup complete $\rightarrow$ {\bf Finish}
\end{itemize}
\item Test der Installation:\\
{\tt su - db2inst1} $\rightarrow$ {\tt db2stop} $\rightarrow$ {\tt db2start} \footnote{Es sollte keine Nachricht bez"uglich der Lizenz erscheinen.} $\rightarrow$ {\tt db2level} $\rightarrow$ {\tt exit}
\item Installieren Sie nun den FixPack 4 gem�� der Anleitung. Bezugsquelle f�r den Fix ist:\\
\url{http://www-3.ibm.com/cgi-bin/db2www/data/db2/udb/winos2unix/support/download.d2w/report}\\[2ex]
Die wichtigsten Schritte sind:
\begin{itemize}
\item Zuerst m��en sie die Schritte durchf�hren, die im Readme des Fixpaks beschreiben sind um die Datenbank zu beenden.
\item Auspacken des FixPack's
\item {\tt ./installFixPak -a}
\item Aktualisieren von Exemplaren zur Verwendung der neuen Stufe von DB2
\item Restart der Datenbank
\item Neubinden der DB2 UDB Datenbanken
\end{itemize}
Eine vollst�ndige Beschriebung befindet sich in der mitgelieferten Readme-Datei.

\end{enumerate}

%

\subsubsection{NSE}
\begin{enumerate}
\item Legen Sie die CD {\bf 'DB2 Net Search Extender Version 8.1'} ein und mounten Sie diese.
\item {\tt su - db2inst1} $\rightarrow$ {\tt db2stop} $\rightarrow$ {\tt exit}
\item {\tt slibclean}
\item {\tt cd /cdrom/aix} $\rightarrow$ {\tt ./nsesetup.sh}
\begin{itemize}
\item Auswahl 1 f"ur Englisch $\rightarrow$ {\bf Enter}
\item Auswahl 1 f"ur 'Accept the license areement'
\end{itemize}
\item {\tt cd /usr/opt/db2\_08\_01/instance}
\item Installieren Sie nun den FixPack 4 gem�� der Anleitung. Dieses Update ist sehr einfach zu bewerkstelligen.\\[2ex]
Die wichtigsten Schritte sind:
\begin{itemize}
\item  Auspacken des FixPack's
\item {\tt ./nsesetup.sh}
\item {\tt cd /usr/opt/db2\_08\_01/instance ; ./db2iupdt db2inst1}
\end{itemize}

\item {\tt ./db2iupdt -u db2fenc1 db2inst1}
\item Test der Installation:\\
{\tt su - db2inst1} $\rightarrow$ {\tt db2start} $\rightarrow$ {\tt db2text start} $\rightarrow$ {\tt db2licm -l}\footnote{Pr"uft Ihren Lizenzeintrag} $\rightarrow$ {\tt exit}

\end{enumerate}
%
\subsubsection{WebSphere 5}
%
{\bf Installation WebSphere 5}
\begin{enumerate}
\item Legen Sie die CD {\bf 'WebSphere Application Server 5.0 for AIX'} ein und mounten Sie diese.
\item {\tt cd /cdrom/aix} $\rightarrow$ {\tt ./install}
\begin{itemize}
\item Auswahl {\bf 'English'}
\item Auswahl {\bf Next}
\item Auswahl 'Accept the license areement'
\item Auswahl {\bf Next}\footnote{Die Warnung "uber fehlende Voraussetzunge kann ignoriert werden.}
\item Auswahl {\bf 'Costum'} $\rightarrow$ {\bf Next}
\item Schauen Sie die Optionsliste durch, wir empfehlen {\bf 'Embedded Messaging'} und {\bf 'Tivoli Performance Viewer'} aus der Auswahl zu entfernen.
\item Auswahl der Installationsverzeichnisse {\it /usr/WebSphere/AppServer} und \linebreak[4]
{\it /usr/IBMHttpServer}. Behalten Sie die Vorgaben bei!
\item Node name $\rightarrow$ Nehmen Sie Ihren Hostnamen.
\item Hostname $\rightarrow$ Nehmen Sie Ihren voll qualifizierten Hostnamen.
\item Zusammenfassung $\rightarrow$ {\bf Next}
\item Warten Sie kurz.
\item Entfernen Sie die Auswahl {\bf 'Register this product now'} $\rightarrow$ {\bf Next}
\item {\bf Finish}
\item Nach einigen Sekunden "offnet sich das {\bf 'WebSphere First Steps'} Fenster, bitte schiessen sie dieses.
\end{itemize}
\end{enumerate}
%
{\bf Installation WebSphere Fix pack 5.0.1}
\begin{enumerate}
\item Holen Sie sich das Fixpack von \url{ftp.software.ibm.com}.
\item {\tt cd <fixpack\_dir>}
\item {\tt . /usr/WebSphere/AppServer/bin/setupCmdLine.sh} Wichtig ist das Leerzeichen zwischen dem Punkt und dem Pfad.
\item {\tt ./updateWizard.sh}\footnote{bei AIX 5.2 kommt es zu einem Fehler dass die Bibliothek \mcrfile{libjvm.a} fehlt. Dann muss 
man sich einen aktuellen installer von \url{http://www-1.ibm.com/support/docview.wss?rs=180&tc=SSEQTP&uid=swg24001908} herunterladen und entpacken. Das Verzeichnis  
\mcrpath{fixpacks} des Fixpaks in das Verzeichnis des Installers kopiert werden.  Dann den \mcrcommand{updateWizard.sh} starten.
\begin{itemize}
\item Update Installation Wizard $\rightarrow$ {\tt Next}
\item Auswahl {\bf 'IBM WebSphere Application Server 5.0.0'} $\rightarrow$ {\tt Next}
\item Auswahl {\bf 'Install Fix packs'} $\rightarrow$ {\tt Next}
\item Fixpack Verzeichnis beibehalten $\rightarrow$ {\tt Next}
\item Auswahl {\bf 'was50\/fp1\_aix'} $\rightarrow$ {\tt Next}
\item Auswahl {\bf 'Update the IBM HTTP Server'} $\rightarrow$ {\tt Next}
\item Zusammenfassung $\rightarrow$ {\tt Next}
\item Wenn Installation erfolgreich war $\rightarrow$ {\tt Finish}
\end{itemize}
\end{enumerate}
%
{\bf Test der WebSphere Installation}
\begin{enumerate}
\item Starten Sie den WebSphere Administrations-Server
\begin{itemize}
\item {\tt cd /usr/WebSphere/AppServer/bin}
\item {\tt ./startServer.sh server1}\footnote{Warten Sie auf die Nachricht {\bf 'Server 1 open for e-business'}}
\end{itemize}
\item Im Browser {\tt http://<hostname>:9090/admin}
\begin{itemize}
\item Eingabe {\bf icmadmin} als UserID $\rightarrow$ {\tt OK}
\item {\bf 'Environment'}
\item Auswahl {\bf 'Virtual hosts`}
\item Auswahl {\bf 'default\_hosts`}
\item Auswahl {\bf 'Host Aliases'}
\item Auswahl {\bf 'New'} $\rightarrow$ Hostname : * $\rightarrow$ Port : 443 $\rightarrow$ {\tt Apply}
\item {\tt Save}\footnote{Den Text finden Sie im Bild oben.}
\item {\tt Save}
\item Im linken Baum {\bf 'Update Web Server Plugin'} ausw"ahlen
\item {\tt OK}
\item {\tt Logout}
\end{itemize}
\item Starten Sie den Web-Server
\begin{itemize}
\item {\tt cd /usr/IBMHttpServer/bin}
\item {\tt ./apachectl start}
\end{itemize}
\item Im Browser {\tt http://<hostname>/snoop}
\end{enumerate}
%
{\bf Configure the SSL}
\begin{enumerate}
\item \mcrcommand{/usr/IBMHttpServer/bin/ikeyman}
\begin{itemize}
\item Select the key database {\bf 'New'}
\item Select {\bf CMS key database file} $\rightarrow$ File Name = \mcrfile{key.kdb} $\rightarrow$ Location = \mcrfile{/usr/IBMHttpServer/ssl}
\item Select {\bf 'OK'}
\item \mcrcommand{<password>} $\rightarrow$ \mcrcommand{<password>}
\item 'Stach the password to a file?' $\rightarrow$ {\bf 'OK'}
\item Select {\bf 'OK'}
\item Select {\bf 'Create'}
\item Select {\bf 'New self signed certificate'} $\rightarrow$ Label = \mcrfile{icmrm} $\rightarrow$ Version = X509V3 $\rightarrow$ Key Size = 1024 $\rightarrow$ Name = \mcrcommand{<full qualified host name>} $\rightarrow$ Organization = \mcrcommand{<your org>} $\rightarrow$ Country = \mcrcommand{DE} $\rightarrow$ Valid Period = \mcrcommand{1000}
\item Select {\bf 'OK'}
\item Select {\bf 'Key database'}
\item Select {\bf 'Exit'}
\end{itemize}
\end{enumerate}
If some environments there have been problems running ikeyman because there might appear a message: You need to register IBMJCE provider. In case you get this message - {\bf and ony then} - here is a list of ways to fix it:
\begin{itemize}
\item Change to the JDK provided by WebSphere (cahnge PATH variable) {\bf OR}
\item Remove the file \mcrfile{gskikm.jar} from the directory \mcrfile{/usr/java131/jre/lib/ext} (move to a different place or rename)
\end{itemize}
%
{\bf Activate the HTTPD for SSL}
\begin{enumerate}
\item \mcrcommand{cd /usr/IBMHttpServer}
\item \mcrcommand{./bin/setupadm}
\begin{itemize}
\item User ID = {\bf httpadm} $\rightarrow$ Group Name = {\bf httpadm} $\rightarrow$ \mcrcommand{<enter>} $\rightarrow$ {\bf 1} $\rightarrow$ {\bf 1} $\rightarrow$ {\bf 2}
\end{itemize}
\item \mcrcommand{./bin/htpasswd -c conf/admin.passwd admin}
\begin{itemize}
\item \mcrcommand{<password>} $\rightarrow$ \mcrcommand{<password>}
\end{itemize}
\item \mcrcommand{/bin/adminctl start}
\end{enumerate}
In a web-browser open the URL {\bf http://<yourhost>:8008/} and login as user {\bf admin} with \mcrcommand{<password>}.
In the web browser please do the following configurations.
\begin{enumerate}
\item Set up the security module
\begin{itemize}
\item In the left navigation panel select {\bf Basic Select} $\rightarrow$ {\bf Module Sequence}
\item In the Module Sequence panel $\rightarrow$ Scope = {\bf GLOBAL} $\rightarrow$
\item {\bf Add} $\rightarrow$ select a module to add $\rightarrow$ in the drop-down list select {\bf ibm\_ssl} $\rightarrow$ \mcrfile{modules/IBMModuleSSL128.dll} wird ersetzt $\rightarrow$ {\bf 'Apply'} $\rightarrow$ {\bf 'Close'} $\rightarrow$ {\bf 'Submit'}
\end{itemize}
\item Set up the secure host IP and aditional prot for the secure server
\begin{itemize}
\item In the left navigation panel select {\bf Basic Select} $\rightarrow$ {\bf Advanced Properties}
\item In the Module Sequence window panel $\rightarrow$ Scope = {\bf GLOBAL} $\rightarrow$
\item {\bf Add} (for the Specify aditional ports and IP address filed) $\rightarrow$ IP addresse = {\bf empty} $\rightarrow$ Port = {\bf 80 $\rightarrow$} {\bf 'Apply'}
\item {\bf Add} (for the Specify additional prots and IP adress field) $\rightarrow$ IP addresse = {\bf empty} $\rightarrow$ Port = {\bf 443 $\rightarrow$} {\bf 'Apply'} $\rightarrow$ {\bf 'Close'} $\rightarrow$ {\bf 'Submit'}
\end{itemize}
\item Set up the virtual host sructure for the secure server
\begin{itemize}
\item In the left navigation panel select {\bf Configuration Structure} $\rightarrow$ {\bf Create Scope}
\item In the Create Scope panel $\rightarrow$ in the drop-down list under Select a valid scope select Virtual Host 'Enter the virtual host IP address or full qualified domain name' = /mcrcommand{<your.full.qualified.hostname>} $\rightarrow$ Virtual Host Port = {\bf 443} $\rightarrow$ Server name = {\bf empty} $\rightarrow$ Alternate name = {\bf empty} $\rightarrow$ {\bf 'Submit'}
\end{itemize}
\item Set up the virtual host document root for the secure server
\begin{itemize}
\item In the left navigation panel select {\bf Basic Select} $\rightarrow$ {\bf Core Settings}
\item In the Core Settings panel $\rightarrow$ select {\bf Scope} $\rightarrow$ select the \mcrcommand{<Virtual host that you created for SSL>} $\rightarrow$ Server name = {\bf empty} $\rightarrow$ Document root directory = \mcrfile{/usr/IBMHttpServer/htdocs/en\_US} $\rightarrow$ {\bf 'Submit'}
\end{itemize}
\item Set the file and SSL timeout values for the secure server
\begin{itemize}
\item In the left navigation panel select {\bf Security} $\rightarrow$ {\bf Server Security}
\item In the Security panel $\rightarrow$ Ensure scope \mcrcommand{<Virtual host that you created for SSL>} $\rightarrow$ {\bf 'Enable SSL'} $\rightarrow$ select {\bf 'Yes'} $\rightarrow$ Key file name = \mcrfile{/usr/IBMHttpServer/ssl/key.kdb} $\rightarrow$ Timeout SSL 2 = {\bf 100} $\rightarrow$ Timeout SSL 3 = {\bf 1000} $\rightarrow$ {\bf 'Submit'}
\end{itemize}
\item Enable SSL and select the mode of the client authentication
\begin{itemize}
\item In the left navigation panel select {\bf Security} $\rightarrow$ {\bf Host Authorization}
\item in the Authorization panel $\rightarrow$ Ensure scope \mcrcommand{<Virtual host that you created for SSL>} $\rightarrow$ {\bf 'Enable SSL'} $\rightarrow$ select {\bf 'Yes'} $\rightarrow$ select {\bf 'Mode of client authorization to be used'} $\rightarrow$ select {\bf 'None'} $\rightarrow$ 'Server cretificate to be used for this virtual hst field' = {\bf icmrm} $\rightarrow$ Add (for the chipher specification that can be used in a secure transaction panel) $\rightarrow$ select {\bf 39} $\rightarrow$ {\bf 'Apply'} select {\bf 3A} $\rightarrow$ {\bf 'Apply'} select {\bf 62} $\rightarrow$ {\bf 'Apply'} select {\bf 64} $\rightarrow$ {\bf 'Apply'} $\rightarrow$ {\bf 'Submit'}
\end{itemize}
\end{enumerate}
Restart the HTTP Server (and leave it open) by clicking on the black circle logo that is located next to the help button in the upper-right corner of the right panel. \\[2ex]
Open \mcrcommand{https://<your hostname>/snoop} in the web browser.\\[2ex]
%
{\bf Installation of the FixPack 2}
\begin{enumerate}
\item Download from \url{ftp://ftp.software.ibm.com/software/websphere/appserv/support/fixpacks/was50/fixpack2/AIX/}
\item Unpack the tar file to a \mcrfile{install\_root}
\item READ the documentation!
\item Set the JAVA\_HOME environment variable.
\item Stop all WebSphere applications with \mcrcommand{/usr/WebSphere/AppServer/bin/stopServer ...}\footnote{like icmrm or mycoresample}
\item Start the install wizzard with \mcrcommand{install\_root/updateWizard.sh} or as command line tool with \mcrcommand{install\_root/updateSilent.sh}
\begin{itemize}
\item Select {\bf english}
\item By default let the path of WebSphere on \mcrfile{/usr/WebSphere/AppServer}.  $\rightarrow$ {\bf 'Next'}
\item Select {\bf fixpacks} $\rightarrow$ {\bf 'Next'}
\item Let the path of install source by \mcrfile{install\_root}. $\rightarrow$ {\bf 'Next'}
\item Confirm that you will install the fixpack 2. $\rightarrow$ {\bf 'Next'}
\item By default let the path of HTTPD on \mcrfile{/usr/IBMHttpServer}. $\rightarrow$ {\bf 'Next'}
\item {\bf 'Next'}
\item {\bf 'Finish'}
\end{itemize}
\end{enumerate}
%
\subsubsection{Content Manager}
\begin{enumerate}
\item Legen Sie bitte die nachfolgenden Benutzer an. {\bf icmadmin} und {\bf rmadmin} ben�tigen nicht viel Platz. F�r den Benutzer {\bf mcradmin} sollte ausreichend Platz eingeplant werden, da hier sp�ter die MyCoRe-Anwendungen installiert werden.. Achten Sie darauf, dass man sich in die Accounts einloggen kann, z. B. via Telnet! Im IBM Handbuch wir ein Accout {\bf icmconct} angegeben, dies entspricht in der Funktionalit�t unserem {\bf mcradmin}. Sicherheitshalber k�nnen sie es mit anlegen.
\begin{itemize}
\item User {\bf icmadmin} $\rightarrow$ Group {\bf db2grp1}
\item User {\bf rmadmin} $\rightarrow$ Group {\bf db2grp1}
\item User {\bf icmconct} $\rightarrow$ Group {\bf staff}
\item User {\bf mcradmin} $\rightarrow$ Group {\bf mcr}
\end{itemize}
\item F�gen sie den Nutzer {\bf root} der Gruppe {\bf db2grp1} hinzu.
\item Erg�nzen Sie das File \mcrfile{/etc/environment} um folgende Zeilen:
\begin{verbatim}
#
# Appendix for Content Manager
#
ICMROOT=/usr/lpp/icm
ICMDLL=/home/db2fenc1
ICMCOMP=/usr/vacpp/bin
CMCOMMON=/usr/lpp/cmb/cmgmt
EXTSHM=ON
DB2INSTANCE=db2inst1
DB2LIBPATH=/usr/lpp/icm/lib
\end{verbatim}
\item Erg�nzen Sie das File \mcrfile{.profile} f�r die User {\bf root}, {\bf icmadmin}, {\bf rmadmin} und {\bf mcradmin} mit den folgenden Zeilen:
\begin{verbatim}
# The following three lines have been added by UDB DB2.
if [ -f /home/db2inst1/sqllib/db2profile ]; then
    . /home/db2inst1/sqllib/db2profile
fi
\end{verbatim}
\item Das File \mcrfile{/home/db2inst1/sqllib/profile.env} sollte folgende Eintr�ge aufweisen:
\begin{verbatim}
DB2_FMP_COMM_HEAPSZ='12000'
DB2ENVLIST='LIBPATH ICMROOT ICMDLL ICMCOMP EXTSHM CMCOMMON DB2LIBPATH'
DB2_RR_TO_RS='YES'
DB2COMM='tcpip'
DB2AUTOSTART='YES'
\end{verbatim}
\item Nun sollte noch einmal die DB2 und NSE gestoppt und neu gestartet werden.\footnote{Es ist sinnvoll nach jedem Reboot des Systems vor dem Start des CM diese Schritte durchzuf�hren um einen sicheren Ausgangspunkt zu haben!}
\begin{itemize}
\item \mcrcommand{su - db2inst1}
\item \mcrcommand{db2stop}
\item \mcrcommand{db2text stop}
\item \mcrcommand{db2start}
\item {\bf Kurz warten!!!}
\item \mcrcommand{db2db2text start}
\end{itemize} 
\item Dieser Punkt ist nur erforderlich, wenn ein TSM System eingesetzt werden soll.
\begin{itemize}
\item \mcrcommand{su - root}
\item \mcrcommand{mkdir </home/rmadmin/staging>}
\item \mcrcommand{mklv -y <lvstaging> <vg..> 32}
\item \mcrcommand{crfs -vjfs -d <lvstaging> -m </home/rmadmin/staging> -A yes}
\item \mcrcommand{mount </home/rmadmin/staging>}
\item \mcrcommand{chown rmadmin.db2grp1 </home/rmadmin/staging>}
\end{itemize}
\item Now you must create a logical volume for the store of the Resource Manager.
\begin{itemize}
\item \mcrcommand{su - root}
\item \mcrcommand{mkdir </home/rmadmin/storage>}
\item \mcrcommand{mklv -y <lvstorage> <vg..> 32}
\item \mcrcommand{crfs -vjfs -d <lvstorage> -m </home/rmadmin/storage> -A yes}
\item \mcrcommand{mount </home/rmadmin/storage>}
\item \mcrcommand{chown rmadmin.db2grp1 </home/rmadmin/storage>}
\end{itemize}
\item  Now you can start the Content Manager installation. Attention, this will open a X11 window connection!
\begin{enumerate}
\item \mcrcommand{su - root}
\item \mcrcommand{cd .../English}
\item \mcrcommand{./setup.exe}
\item {\bf 'Next'}
\item Select I accept ... {\bf 'Next'}
\item Setup type $\rightarrow$ {\bf Select full} $\rightarrow$ {\bf 'Next'}
\item Identification and authorization for LS
\begin{itemize}
\item Library Server database name $\rightarrow$ {\bf ICMNLSDB}
\item Library Server scheme name $\rightarrow$ {\bf ICMADMIN}
\item Library Server database administration ID $\rightarrow$ {\bf icmadmin}
\item Password ... $\rightarrow$ Confirm password ...
\item Database connection ID $\rightarrow$ {\bf mcradmin}
\item Library Server ID $\rightarrow$ {\bf 1}
\item Unselect {\bf Enable Unicode}
\item Select {\bf Enable text search}
\end{itemize}
\item Configure Resource Manager
\begin{itemize}
\item Resource Manager database name $\rightarrow$ {\bf RMDB}
\item Resource Manager database administration ID $\rightarrow$ {\bf rmadmin}
\item Password ... $\rightarrow$ Confirm password ...
\end{itemize}
\item Installation options for resource manager database
\begin{itemize}
\item Mount point $\rightarrow$ \mcrfile{</home/rmadmin/storage>}
\item Path $\rightarrow$ \mcrfile{</home/rmadmin/staging>}
\end{itemize}
\item Resource Manager with WebSphere Application Server
\begin{itemize}
\item WebSphere home $\rightarrow$ \mcrfile{/usr/WebSphere/AppServer}
\item Web application path $\rightarrow$ {\bf /icmrm}
\item Web application name $\rightarrow$ {\bf icmrm}
\item Service Port $\rightarrow$ {\bf 7500}
\item Application server name $\rightarrow$ {\bf icmrm}
\item Your WebSphere Application Server will be stopped$ \rightarrow$ {\bf 'Yes'}
\end{itemize}
\item WebSphere V5 auto deploy options
\begin{itemize}
\item WebSphere administration user ID $\rightarrow$ {\bf icmadmin}
\item Password ... $\rightarrow$ Confirm password ...
\item Node name $\rightarrow$ {\bf <your host name>}
\end{itemize}
\item Connect Library Server to Resource Manager
\begin{itemize}
\item Resource Manager server hostname $\rightarrow$ {\bf <your full qualified hostname>}
\item Web application port $\rightarrow$ {\bf 80}
\item Secure web application port $\rightarrow$ {\bf 443}
\item Resource Manager server operating system $\rightarrow$ {\bf AIX}
\item Token duration $\rightarrow$ {\bf 48}
\end{itemize}
\item Configure components for LDAP
\begin{itemize}
\item Do NOT select Library Server
\item Do NOT select Resource Manager
\end{itemize}
\item Summary $\rightarrow$ {\bf 'Next'}
\item {\bf 'Finish'}
\end{enumerate}
\item Verify the log file \mcrfile{/usr/lpp/icm/logs/icm82install.log}.
\item Start the Resource Manager as root \mcrcommand{/usr/WebSphere/AppServer/bin/startServer.sh icmrm}.
\item Test the Resource Manager in a web browser $\rightarrow$ \mcrcommand{https://<your full qualified hostname>/icmrm/ICMResourceManager}\footnote{The response is : No order found to process, that is okay.}
\end{enumerate}
%
{\bf Installation of the FixPack 2}
\begin{enumerate}
\item Download from \url{ftp://ftp.software.ibm.com/ps/products/content_manager/fixes/v8.2/aix/820.20/}
\item READ the documentation!
\item Do the steps under 2.2.1 of the documentation.
\end{enumerate}
%
\subsubsection{Information Integrator for Content Manager}
\begin{enumerate}
\item \mcrcommand{su -}
\item Change to the install directory.
\item \mcrcommand{./frnxsetup.sh} Attention, this will open a X11 window connection!
\item License Agreement $\rightarrow$ {\bf 'Accept'}
\item {\bf 'Next'}
\item Install the following options
\begin{itemize}
\item Remote Connectors $\rightarrow$ {\bf none}
\item Local Connectors $\rightarrow$ {\bf only CM V8 connector}
\item Connector Toolkitd and Samples $\rightarrow$ {\bf only CM V8 connector}
\item Features $\rightarrow$ {\bf none}
\item Infocenter $\rightarrow$ {\bf optional}
\item System Admin Database $\rightarrow$ {\bf none}
\item {\bf 'Next'}
\end{itemize}
\item System configuration
\begin{itemize}
\item Keep {\bf Local}
\item Do not select LDAP
\item {\bf 'Next'}
\end{itemize}
\item Content Manager V8 Server Connection
\begin{itemize}
\item Database name $\rightarrow$ {\bf icmnlsdb}
\item Schema name $\rightarrow$ {\bf ICMADMIN}
\item Authentication type $\rightarrow$ {\bf Server}
\item Database connection ID $\rightarrow$ {\bf mcradmin}
\item Password ... $\rightarrow$ Confirm password ...
\item Enable single sign-on $\rightarrow$ {\bf false}
\item {\bf 'Next'}
\end{itemize}
\item Content Manager V8 Connector $\rightarrow$ {\bf 'Next'}
\item WAIT!
\item {\bf 'Finish'}
\item Check the log under \mcrfile{/tmp/frn/frnxinst.log}
\end{enumerate}
%
{\bf Installation of the FixPack 2}
\begin{enumerate}
\item Download from \url{ftp://ftp.software.ibm.com/ps/products/enterprise_information_portal/fixes/v8.2/aix/820.20}
\item READ the documentation!
\item Do the steps under 2.2.1 of the documentation.
\end{enumerate}
%
%

