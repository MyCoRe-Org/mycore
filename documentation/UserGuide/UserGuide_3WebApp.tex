\section{Erzeugen und Konfigurieren der Web-Anwendung}
\subsection{Erzeugen der Web-Anwendung}
Durch Eingabe von
\begin{center} 
{\tt build.sh webapp } \qquad bzw. \qquad {\tt build.cmd webapp } 
\end{center} 
wird die MyCoRe Sample Web Application im Verzeichnis {\tt webapps}
erzeugt. Alternativ k�nnen Sie auch ein Web Application Archive (war)
erzeugen, indem Sie
\begin{center}
{\tt build.sh war } \qquad bzw. \qquad {\tt build.cmd war }
\end{center}
aufrufen. 

Das MyCoRe Build-Script kopiert beim Erzeugen der Web Applikation
auch alle externen, erforderlichen jar-Dateien Ihrer verwendeten 
Datenbank-Systeme (IBM Content Manager / DB2, MySQL, eXist) in das
Verzeichnis {\tt WEB-INF/lib}, entsprechend den Vorgaben Ihrer
Konfiguration in {\tt build.properties}. Beachten Sie dazu bitte die 
Hinweise in der Ausgabe beim Erzeugen der Web Application.

\subsection{Konfiguration des Web Application Server}
\subsubsection{ Tomcat}
Die grundlegende Installation von Tomcat wurde bereits beschrieben. Nun soll auf dieser Basis das die WEB-Anwendung des MyCoRe-Samples installiert werden. Dabei ist an dieser Stelle nur ein einfaches Szenario auf der Basis der Tomcat-Grundinstallation beschrieben. F�r die Konfiguration komplexerer Modelle, z. B. mehrere Applikationen nebeneinander, gibt es weiter hinten in diesem Dokument eine ausf�hrliche Anleitung.\\[2ex]
Folgende Schritte sind auszuf�hren:
\begin{enumerate}
\item \mcrcommand{su -}
\item \mcrcommand{cd \$CATALINA\_HOME/webapps}
\item \mcrcommand{cp \$MYCORE\_SAMPLE\_HOME/mycoresample.war .}
\item \mcrcommand{rctomcat restart}
\item \mcrcommand{rm mycoresample/WEB-INF/lib/xerces*}
\item \mcrcommand{rm mycoresample/WEB-INF/lib/xalan*}
\item \mcrcommand{rctomcat restart}
\end{enumerate}
Nun sollten Sie auf die Beispielanwendung mit der URL \url{http://localhost:8080/mycoresample} zugreifen k�nnen. Testen Sie nun die Anwendung!
\subsubsection{ Websphere}
\subsection{Die Nutzung des Editor-Servlets}

