\section{Die Zusammenarbeit mit anderen MyCoRe-Sample-Installationen}

Das MyCoRe-System ist so konzipiert, dass hinsichtlich der metadaten gleichartige Installationen miteinander arbeiten k�nnen und von einer gemeinsamen Oberf�cher (Frontend) abgefragt werden k�nnen. Hierzu m�ssen die Remote-Instanzen definiert werden. Auch die eigene Installation kann �ber diesen Weg abgefragt werden. Voraussetzung ist die im Abschitt 'Erzeugen und Konfigurieren der Web-Anwendung' beschriebene Installation eines Web Application Servers, welcher f�r die Remote-Zugriffe via Servlets zust�ndig ist. 

\subsection{Die eigene Installation}

Die Konfiguration f�r die eigene Installation finden Sie im File \mcrfile{mycore.properties.private}. Hier muss im Normalfall nur die Hostadresse und ggf. der Port ge�ndert werden, alle anderen Angaben sollten �bernommen werden k�nnen.
\begin{verbatim}
# Configuration for the own host with remote access
MCR.remoteaccess_remote_query_class=org.mycore.backend.remote.MCRServletCommunication
MCR.remoteaccess_remote_host=pcclu02.rz.uni-leipzig.de
MCR.remoteaccess_remote_protocol=http
MCR.remoteaccess_remote_port=8080
MCR.remoteaccess_remote_query_servlet=/mycoresample/servlets/MCRQueryServlet
MCR.remoteaccess_remote_ifs_servlet=/mycoresample/servlets/MCRFileNodeServlet
\end{verbatim}

\subsection{Standard-Server des MyCoRe-Projektes}

Von den Entwicklern des MyCoRe-Projektes werden exemplarisch einige MyCoRe-Sample-Installationen bereitgehalten. Diese sind im Konfigurationsfile \mcrfile{mycore.properties.remote} notiert und sollten in der Regel verf�gbar sein. Sie repr�sentieren eine Auswahl der verschieden Persistence-Layer. Auch die Auswahl f�r die Suche in diesen Instanzen ist bereits in das Sample integriert und solle nach dem erfolgreichen Start der Web Applikation aktiv sein.\\[2ex]
\small
\bottomcaption{Feste MyCoRe-Sample-Instanzen }
\tablehead{\hline}
\tabletail{\hline}
\begin{supertabular}{|p{4cm}|p{7cm}|p{3cm}|}
\hline
{\bf Alias} & {\bf URL} & {\bf Standort}\\[1,5ex]
 \hline
mcrLpzHHttp & ibmdlh.rz.uni-leipzig.de & Uni Leipzig \\ \hline
\end{supertabular}
\normalsize

