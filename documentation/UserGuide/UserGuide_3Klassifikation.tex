%
%
\subsection{Das Klassifikationen-Datenmodel}
%
%
Wie bereits erw�hnt dienen Klassifikationen der einheitlichen Gliederung bestimmter Fakten. Sie sorgen daf�r, dass z. B. eine einheitliche Schreibweise f�r bestimmte Begriffe verwendet wird. Diese Einzelbegriffe werden in als Kategorien bezeichnet. Innerhalb einer Kategorie kann der Begriff z. B. in verschiedenen Sprachen aufgezeichnet sein. Die eindeutige Zuordnung zu einer Kategorie erfolgt �ber einen Identifizierer. Dieser besteht aus der Klassifikations- und Kategorie-ID und muss eindeutig sein.\\[2ex]
Klassifikationen werden im MyCore-Sample als extra XML-Files erstellt, in die Anwendung importiert und in Form einer Datenbank gespeichert. Dies ist f�r den Nutzer transparent und erfolgt mittels Schnittstellen. Der Zugriff auf die Daten erfolgt dann durch den oben genannten Identifizierer. Die Klassifikations-ID ist eine MCRObjectID mit dem Typ \mcridentifier{class}. Die Kategorie-ID ist dagegen frei w�hlbar. Sie darf mehrstufig ein, jede Stufe spiegelt eine Hierarchieebene wieder. Die Stufen in der ID werden mit einem Punkt voneinander getrennt, z. B. 'Uni.URZ'. Das wiederum gestattet eine Abfrage nach allen untergeordneten Stufen bzw. Sub-Kategorien wie z. B. 'Uni.*'. {\bf Achtung, sollten Sie Zahlen als Kategorie-ID's mit verwenden, so planen Sie entsprechende Vornullen ein, andernfalls wird das Suchergebnis fehlerhaft!}.\\[2ex]
Im \mcridentifier{ID} Attribut einer \mcridentifier{category} ist der eindeutige Identifizierer anzugeben. Das darunter befindliche \mcridentifier{label} Tag bietet die M�glichkeit, eine Kurzbezeichnung anzugeben. Mehrsprachige Ausf�hrungen sind erlaubt. Das selbe gilt f�r das Tag \mcridentifier{description}. Beide Werte werden als Strings aufgefasst. Eine \mcridentifier{category} kann wiederum \mcridentifier{category} Tags beinhalten.\\[2ex]

\lstset{language=XML,fancyvrb=true,frame=btlr,breaklines,prebreak={\space\MyHookSign}}
\begin{lstlisting}[caption=XML-Syntax eines Klassifikations-Objektes,label=lst:xml_syntax_metadatenclass]
 <?xml version="1.0" cncoding="iso-8859" ?>
 <mycoreclass
  xmlns:xsi="http://www.w3.org/2001/XMLSchema-instance"
  xsi:noNamespaceSchemaLocation="MCRClassification.xsd"
  xmlns:xlink="http://www.w3.org/1999/xlink"
  ID="..."
  >
  <label xml:lang="..." text="..." description="..."/>
  ...
  <categories>
   <category ID="...">
    <label xml:lang="..." text="..." description="..."/>
    ...
    <category ID="...">
     <label xml:lang="..." text="..." description="..."/>
     ...
    </category>
    <category ID="...">
     <label xml:lang="..." text="..." description="..."/>
     ...
    </category>
   </category>
   <category ID="...">
    <label xml:lang="..." text="..." description="..."/>
    ...
   </category>
  </categories>
 </mycoreclass>
\end{lstlisting}

