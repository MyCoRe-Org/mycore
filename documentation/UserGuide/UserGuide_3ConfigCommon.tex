%
%
\subsection{Grundlegende Konfigurationen}
%
%
Dieser Abschnitt besch�ftigt sich mit der Konfiguration der Beispielanwendung in allgemeinen Bereichen wie JDBC, Logger, usw. Die MyCoRe Konfigurationen f�r das Beispiel finden Sie im Verzeichnis {\tt /config}.
%
%
\subsubsection{Pfade und Systemumgebung anpassen}

Auch f�r die Zusammenstellung und Installation der Beispiel-Anwendung verwendet MyCoRe das Apache Ant Build-Tool. 
Entsprechend der Installationsanleitung des Ant-Paketes sollten Sie zun�chst die 
Umgebungsvariablen {\tt JAVA\_HOME} und {\tt ANT\_HOME} gesetzt haben. Sollten diese Variablen auf Ihrem System noch nicht 
gesetzt sein, k�nnen Sie dies in der Datei {\tt build.sh} (Unix) bzw. {\tt build.cmd} (Windows) nachholen und korrigieren. 

Die MyCoRe Beispiel-Anwendung verwendet die Dateien aus dem MyCore Kern, insbesondere die erzeugte Datei 
{\tt mycore-for[cm8|xmldb].jar} und die Konfigurationsdatei f�r den Build-Prozess {\tt build.properties}.
Wenn die Verzeichnisse {\tt mycore} und \\
{\tt mycore-sample-application} auf Ihrem System nicht in einem gemeinsamen
�bergeordneten Verzeichnis liegen, k�nnen Sie im build-Skript auch die Umgebungsvariable {\tt MYCORE\_HOME} auf einen entsprechend
korrigierten Wert setzen, da die Vorgabe {\tt ../mycore} ist.

Sie sollten zun�chst pr�fen, ob ihre Systemumgebung korrekt eingerichtet ist, indem Sie 
\begin{center} 
{\tt build.sh info } \qquad bzw. \qquad {\tt build.cmd info } 
\end{center} 
ausf�hren. Das Ant Build Tool zeigt Ihnen daraufhin die verwendeten JDK- und Ant-Software-Versionen und den 
generierten CLASSPATH und LIBPATH (f�r Unix Systeme) an. 
Eine �bersicht �ber alle wesentlichen Build-Ziele erhalten Sie mit 
\begin{center} 
{\tt build.sh usage } \qquad bzw. \qquad {\tt build.cmd usage } 
\end{center} 

\subsubsection{JDBC-Treiber konfigurieren}

Im MyCoRe-Projekt werden ein Teil der Organisations- und Metadaten in klassischen relationalen Datenbanken gespeichert. 
Um die Arbeit mit verschiedenen Anbietern m�glichst einfach zu gestalten, wurde die Arbeit mit dieser Datenbank gegen die JDBC-Schnittstellen programmiert.\\
In der Konfigurationsdatei {\it mycore.properties.private} legen Sie im Parameter \\
{\bf MCR.persistence\_sql\_driver} fest, welcher JDBC-Treiber verwendet werden soll. 
Weiterhin m�ssen Sie die Variable {\bf MCR.persistence\_sql\_database\_url} anpassen, die die JDBC URL f�r Verbindungen
zu Ihrer Datenbank festlegt. Der DB2 Library-Name {\bf LIB} muss durch den aktuellen (z. B. {\bf ICMNLSDB}) ersetzt werden. Analog dazu muss der User {\bf ODBC} bei MySQL durch den entsprechenden Nutzer (z. B. {\bf mcradmin}) erstzt werden. Beachten Sie dabei insbesondere, dass meist Gross/Kleinschreibung relevant ist!
Weiterhin k�nnen Sie die minimale und maximale Anzahl der gleichzeitigen Verbindungen zur Datenbank festlegen.

\begin{verbatim}
# JDBC parameters for connecting to DB2
#MCR.persistence_sql_database_url=jdbc:db2:LIB
#MCR.persistence_sql_driver=COM.ibm.db2.jdbc.app.DB2Driver

# JDBC parameters for connecting to MySQL
MCR.persistence_sql_database_url=jdbc:mysql://localhost/mycore?user=ODBC
MCR.persistence_sql_driver=org.gjt.mm.mysql.Driver

MCR.persistence_sql_init_connections=1
MCR.persistence_sql_max_connections=5
\end{verbatim}

\subsubsection{Debug  konfigurieren}

Innerhalb des MyCoRe-Projektes wird zum Erzeugen aller Print-Ausgaben f�r das Commandline-Tool und/oder die Stdout-Logs das externe Paket {\bf log4j}des Apache-Jakarta-Projektes benutzt, URL \url{http://jakarta.apache.org/log4j/docs/index2.html} . Dieses ist mittlerweile ein Quasistandard und erm�glicht eine gezielte Steuerung der Informationen, welche man erhalten m�chte. \\[2ex]
In der Grundkonfiguration in \mcrfile{mycore.properties.private} ist der Output-Level INFO eingestellt. Eine zweite Standardvorgabe ist DEBUG, diese ist auskommentiert und kann alternativ bei Problemen genommen werden. {\bf log4j} bietet jedoch dar�ber hinaus noch viele weitere M�glichkeiten, die Sie bitte der Dokumentation zu diesem Produkt entnehmen. Erg�nzend sei auch auf das MyCoRe-Konfigurationsfile \mcrfile{mycore.properties.logger} hingewiesen.
\begin{verbatim}
# Set the log level and appender for the general logger
MCR.log4j.rootLogger=INFO, stdout
#MCR.log4j.rootLogger=DEBUG, stdout

# Set the log level and appender special classes
#MCR.log4j.logger.org.mycore.services.oai=DEBUG, stdout
\end{verbatim}

\subsubsection{Metadaten-Store konfigurieren}

Den Typ des zur Laufzeit des Systems zu nutzenden Metadaten-Store konfigurieren
haben Sie bereits in der Datei {\tt \$MYCORE\_HOME/bin/build.properties} 
�ber den Parameter {\tt MCR.XMLStore.Type} festgelegt. Dabei sind die Werte
{\tt cm8} f�r IBM Content Manager 8, oder {\tt xmldb} f�r eine XML:DB 
kompatible XML-Datenbank wie eXist oder Tamino (Software AG) m�glich. Die 
MyCoRe Beispiel-Anwendung verwendet automatisch diese Konfiguration, sie 
m�ssen nun nur noch die Parameter der einzelnen XML-Stores konfigurieren.

\subsubsection{Konfiguration von IBM Content Manager 8}

Falls Sie {\tt MCR.XMLStore.Type=cm8} verwenden, passen Sie in der Datei \\
{\tt mycore.properties.private} die Variablen {\tt MCR.persistence\_cm8\_*} an.
Die Eintr�ge sind eigentlich selbsterkl�rend, so dass an dieser Stelle auf weitere Erl�uterungen verzichtet werden kann.

\begin{verbatim}
# Special values for the persistence layer
MCR.persistence_cm8_max_connections=2
MCR.persistence_cm8_library_server=ICMNLSDB
MCR.persistence_cm8_user_id=icmadmin
MCR.persistence_cm8_password=????????

# Special values for the text search engine
MCR.persistence_cm8_textsearch_ccsid=819
MCR.persistence_cm8_textsearch_lang=DE_DE
MCR.persistence_cm8_textsearch_indexdir=/home/db2inst1/sqllib/db2ext/indexes
MCR.persistence_cm8_textsearch_workingdir=/home/db2inst1/sqllib/db2ext/indexes
\end{verbatim}

Da der Persitence Layer CM8 auf einem Mapping der XML-Daten nach DB2 besteht, m�ssen die CM8 ItemTypes vor dem Laden der Daten separat angelegt werden. Dies geschiet mittels
\begin{center} 
{\tt build.sh create.metastore } \qquad bzw. \qquad {\tt build.cmd create.metastore } 
\end{center} 

\subsubsection{Die Nutzung von eXist als XML:DB Backend}

Falls Sie {\tt MCR.XMLStore.Type=xmldb} verwenden, passen Sie in der Datei \\
{\tt mycore.properties.private} die Variablen {\tt MCR.persistence\_xmldb\_driver} \\
und {\tt MCR.persistence\_xmldb\_database\_url} an.
Die folgenden Zeilen sind f�r Nutzer der freien Software eXist gedacht. Sollten
Sie eine andere XML:DB benutzen, passen sie die Daten bitte entsprechend den
Erfordernissen an.

\begin{verbatim}
MCR.persistence_xmldb_driver=org.exist.xmldb.DatabaseImpl
MCR.persistence_xmldb_database_url=xmldb:exist://localhost:8081/db/mycore
MCR.persistence_xmldb_database=exist
\end{verbatim}

Die folgenden Zeilen sind f�r Nutzer der freien Software eXist gedacht. Sollten
Sie eine andere XML:DB benutzen, passen sie die Daten bitte entsprechend den
Erfordernissen an.
Starten Sie nun den eXist-Client ({\it $<$eXist-installdir$>$/bin/client.sh} bzw. {\it cliend.cmd}) und f�hre Sie folgende Kommandos zum anlegen der Stores unter eXist aus:

\begin{verbatim}
mkcol mycore
chown guest guest mycore
cd mycore
mkcol legalentity
chown guest guest legalentity
mkcol document
chown guest guest document
mkcol derivate
chown guest guest derivate
quit
\end{verbatim}

\subsubsection{Speicherung von Dokumenten konfigurieren}

Neben den Metadaten sind im Sample auch eine Reihe von Dokumenten und Bildern
zum Laden in die Beispielanwendung abgelegt.\\[2ex]

{\bf File System Store}\\
In der Grundkonfiguration verwendet die Beispiel-Applikation zur Speicherung der Datei-Inhalte der Derivate das lokale Dateisystem. Passen Sie in der Datei \\
{\tt mycore.properties.private} die Variable {\tt MCR.IFS.ContentStore.FS.BaseDirectory} 
an und erzeugen Sie ein neues, leeres Verzeichnis am angegebenen Ort.\\[2ex]

{\bf Lucene Store}\\
Sollten sie in der freien Variante noch die Textindizierung f�r 
\mcrfile{*.txt} und \mcrfile{*.pdf} mittels Lucene w�nschen, so sind noch
folgende Dinge zu tun:

\begin{enumerate}

\item
{\tt build.sh compile }

\item
{\tt build.sh plugins }

\item
Setzen Sie die nachfolgenden Variablen in {\tt mycore.properties.private} auf 
die richtigen Werten. Dabei kann das Lucene.BaseDirectory auf das selbe 
Verzeichnis verweisen wie der File System Store.\\
{\tt MCR.PluginDirectory}\\
{\tt MCR.IFS.ContentStore.Lucene.IndexDirectory}\\
{\tt MCR.IFS.ContentStore.Lucene.BaseDirectory}

\item
Erzeugen Sie nun noch die eingetragenen Verzeichnisse.

\item
Als letztes muss noch in der Datei {\tt ContentStoreSelectionRules.xml} die
Sektion f�r den Lucene-Store aktiviert werden.
\end{enumerate}

{\bf CM8 Store}\\
Die Speicherung der Dokumente im IBM Content Manager bietet die M�glichkeit
einer automatischen Textindizierung. Dazu m�ssen aber alle Konfigurationen
f�r die Nutzung des TextSearch Extenders gesetzt werden. Siehe Installation
CM.

