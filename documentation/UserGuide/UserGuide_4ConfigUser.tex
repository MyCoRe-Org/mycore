%
%
\subsection{Das User- und Rechtesystem}
%
%
Im MyCoRe-Sample wird zur Demonstartion des User- und Rechtesystems auch eine konkrete Beispielkonfiguration f�r diesen Bereich mitgeliefert. Ein allgemeiner �berblick zu diesem Teilsystem wurde bereits weiter oben gegeben.\\
Im Konfigurationsverzeichnis des Samples unter 
{\it \$MYCORE\_SAMPLE\_HOME/config} ist zuerst das entsprechende Property-File 
mit dem Namen {\it mycore.properties.user} n�her zu betrachten. Sie finden hier
neben den bereits im vorigen Abschnitt besprochenen Wertzuweisungen noch
Festlegungen der Standardbenutzer. �ndern Sie diese entsprechend Ihren W�nschen.
\begin{verbatim}
# The configuration for the superuser
MCR.users_superuser_username=gandalf
MCR.users_superuser_userpasswd=alleswirdgut
MCR.users_superuser_groupname=zauberer

# The configuration for the guestuser
MCR.users_guestuser_username=aragorn
MCR.users_guestuser_userpasswd=mensch
MCR.users_guestuser_groupname=menschen
\end{verbatim}

Die beiden Abschnitte legen Vorgaben f�r zwei Benutzer fest, welche automatisch
bei der Initialisierung des User-Systems angelegt werden. Der erste ist der 
Superuser innerhalb dieses MyCoRe-Projektes, der andere ein ganz simpler 
Anwender. F�r ein eigenes Projekt, sollten Sie hier Anpassungen vornehmen.\\
Nun k�nnen noch weitere Nutzer nach dem Muster des MyCoRe-Samples unter
dem Verzeichnis {\it \$MYCORE\_SAMPLE\_HOME/content/user} erstellt werden. 
Bearbeiten Sie nun die Datei {\it \$MYCORE\_SAMPLE\_HOME/build.xml} im
Abschnitt {\it userdb}, so dass alle Benutzer, Gruppen und Privilegien nun
von dort mit dem folgenden Kommando in einem Arbeitsgang geladen werden k�nnen.
\footnote{Welches Kommando Sie nutzen h�ngt von Ihrer Umgebungseistellung ab.}

\begin{verbatim}
cd \$MYCORE\_SAMPLE\_HOME; bin/build.sh userdb
\end{verbatim} 
oder 
\begin{verbatim}
cd \$MYCORE\_SAMPLE\_HOME; ant userdb
\end{verbatim} 


Der Aufruf initialisiert zuerst das User-System und legt die in der 
Konfiguration angegebenen Tabellen an. Anschlie�end werden die Privilegien, 
Gruppen und Benutzer geladen. Bitte achten Sie darauf, dass die vorgegebene 
Reihenfolge eingehalten wird, da es innerhalb des Beispiels Abh�ngigkeiten 
zwischen Usern und Gruppen gibt. Nun sollten Sie �ber ein komplettes 
User-System f�r den Dokument-Server verf�gen.\\[2ex]

