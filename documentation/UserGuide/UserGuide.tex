\documentclass[a4paper,12pt]{report}
\usepackage{ngerman}

%neue deutsche Rechtschreibung
%\usepackage[english,ngerman]{babel}

\usepackage[latin1]{inputenc} %f�r Linux
%\usepackage[ansinew]{inputenc} %fuer Windows

\usepackage{mycore}

\begin{document}
%-------   Vorspann
\title{MyCoRe User Guide}
\author{
    Frank L"utzenkirchen\\
    Jens Kupferschmidt\\
    Detlef Degenhardt\\
    Johannes B�hler\\
    Ulrike Kr"onert}
\maketitle
\setcounter{secnumdepth}{10}
\chapter*{Vorwort}
In diesem Dokument sind alle Arbeiten zum Start der Beispielanwendung und zur Gestaltung eigener Anwendungen beschrieben. Teilweise wird auch auf das MyCoRe Design Guide verwiesen.

\pagebreak
\tableofcontents/
\pagebreak
\listoffigures
\pagebreak
\listoftables
%-------    Hauptteil
% Kapitel 1
%
%
\chapter{Voraussetzungen f"ur eine MyCoRe Anwendung}
%
%
\section{Vorabbemerkungen}
%
%
Das MyCoRe-Projekt ist so designed, dass es dem Einzelnen Anwender frei steht,
welche Komponenten er f"ur die Speicherung der Daten verwenden will. dabei spielt nat"urlich das verwendete Betriebssystem eine wesentliche Rolle. Dabei hat jeses System eine eigenen Vor- und Nachteile, die an dieser Stelle aber nicht dikutiert werden sollen. Vielmehr wollen wir es dem Anwender "uberlassen, in welchem System er f"ur seine Anwendung die gr"o"sten Vorteile sieht. Nachfolgend finden Sie eine Tabelle der wesentlichen eingesetzten Komponenten entsprechend des gew"ahlten Basissystems. \\[2ex]
\bottomcaption{MyCoRe Komponenten"ubersicht}
\tablehead{\hline}
\tabletail{\hline}
\begin{supertabular}{|p{2cm}|p{3,25cm}|p{3,25cm}|p{3,25cm}|p{3,25cm}|}
\hline
{\bf Teil} & {\bf AIX} & {\bf Solaris} & {\bf Linux} & {\bf MS Windows}\\[1,5ex] \hline
Metadaten Store & IBM CM 8.2 - parametrische und Volltextsuch mittels XPath Abfragen & IBM CM 8.2 - parametrische und Volltextsuch mittels XPath Abfragen & Xindice - parametrische Suche mittels XPath Abfragen & IBM CM 8.2 - parametrische und Volltextsuch mittels XPath Abfragen \\ \hline
TextSearch & IBM DB2 TIE (Sprachunterst�tzung nur f"ur English) & IBM DB2 TIE (Sprachunterst�tzung nur f"ur English) & htdig ??? & IBM DB2 TIE (Sprachunterst�tzung nur f"ur English) \\ \hline
Datenbank & IBM DB2 8.x & Oracle ??? & MySQL 4.x & IBM DB2 8.x \\ \hline
Objekt Store & Filesystem, & Filesystem, & Filesystem, & Filesystem, \\
 & IBM CM 8.2 Ressource Manager, & IBM CM 8.2 Ressource Manager, & & IBM CM 8.2 Ressource Manager, \\ 
 & IBM Video Charger 8, & IBM Video Charger 8, & IBM Video Charger 8, & IBM Video Charger 8, \\
 & Helix Server & Helix Server & Helix Server & Helix Server \\ 
\hline
\end{supertabular}
%
%
\section{Hinweise zur Installation des IBM Content Manager 8.2}
%
%
\subsection{Der IBM Content Manager unter AIX}
An dieser Stelle soll eine Kurzbeschreibung der Installation des IBM Content Managers 8.2 f"ur AIX von Holger K"onig, IBM Deutschland GmbH, wiedergegeben werden. \\[2ex]
\subsubsection{Vorbereitung}
\begin{enumerate}
\item Installieren Sie das AIX Betriebssystem mit dem Release 4.3.3 ML 10, 5.1 ML 01 oder 5.2.
\item Sorgen Sie daf"ur, dass 'Cultural Conversion' und 'Language' auf English US eingestellt ist.
\item Aktivieren Sie die Netzanbindung inklusive DNS.
\item F"ur die Betriebssystem-Releases 4.3.3 und 5.1 muss Java 1.3.1 entsprechend der Anleitung installiert werden. Erweitern Sie in {\it /etc/environment }  die {\it PATH} Variable um {\it /usr/java131/jre/bin} und {\it /usr/java131/bin}.
\item Installieren Sie den VAC Compiler Version 5.x oder 6.0 entsprechend der Anleitung.
\item Aktivieren Sie das Lizenzsystem {\bf ifor} und Tragen Sie sie Compilerlizenzen ein.
\end{enumerate}
\subsubsection{DB2 und NSE}
\begin{enumerate}
\item Kopieren Sie das File {\it ese.sbcs.tar.Z} von der CD {\bf 'DB2 8.1 with FP1'} und entpacken Sie dieses.
\item {\tt ./db2setup}
\item W"ahlen Sie {\bf Install Products} $\rightarrow$ {\bf DB2 UDB Enterprise Server Edition}. Folgen Sie den Schritten:
\begin{itemize}
\item {\bf Netx}
\item {\bf Accept License}
\item Select 'Custom' $\rightarrow$ {\bf Next}
\item Select 'Install DB2 UDB Enterprise Server Edition on this computer' $\rightarrow$ {\bf Next}
\item Select 'Appliction Development Tools' zus"atzlich
\item Sprache 'Englisch' beibehalten $\rightarrow$ {\bf Next}
\item Erzeugen der DB2 Instanz durch Select 'Create a DB2 instance - 32 bit' $\rightarrow$ {\bf Next}
\item Select 'Single-partition instance' $\rightarrow$ {\bf Next}
\end{itemize}
\end{enumerate}
%

%
%
\subsection{Der IBM Content Manager unter Solaris}

%
%
\subsection{Der IBM Content Manager unter Windows}

%
%
\section{Hinweise zur Installation frier Datenbanken und XML:DB's}

%
%
\subsection{Die Installation von MySQL}

%
%
\subsection{Die Installation von Xindice}

%
%
\section{Hinweise zur Arbeit mit der Servlet-Engine}

%
%
\subsection{Arbeiten mit Tomcat}

%
%
\subsection{Arbeiten mit Websphere}

%
%
\section{Weitere erforderliche Software}



% Kapitel 2
%
%
\chapter{Download und Installation des MyCoRe Kerns}
%
%
\section{Download des MyCoRe Kerns}
%
%
Der MyCoRe Kern wird f"ur alle unterst"utzten Systeme "uber das CVS Repository
ausgeliefert. Das Holen der aktuellen Version erfolgt mit dem Kommando
\begin{center}
{\tt cvs -d :pserver:anoncvs@server.mycore.de:/cvs checkout mycore}
\end{center}
Nach dem erfolgreichen Checkout erhalten Sie folgende Dateistruktur:\\[2ex]
\bottomcaption{Dateistruktur des MyCoRe Kernes}
\tablehead{\hline}
\tabletail{\hline}
\begin{supertabular}{|p{5cm}|p{10cm}|}
\hline
{\bf mycore} &  Das Root-Verzeichnis des MyCoRe-Kerns \\
\quad {\bf bin} & Das Verzeichnis der Shellscripte \\
\qquad build.sh & Startet den Build-Prozess unter einem UNIX-System \\
\qquad build.cmd & Startet den Build-Prozess unter einem Windows-System \\
\qquad build.properties & Konfiguration Pfade und Libraries der verwendeten Datenbanken \\ 
\quad {\bf documentation} & Dokumentationen zu MyCoRe \\
\quad {\bf lib} & Notwendige zus"atzliche Java-Bibliotheken \\
\quad {\bf schema} & XMLSchema Dateien, die anwendungsunabh"angig sind \\
\quad {\bf stylesheets} & Intern verwendete, anwendungsunabh"angige XSL Stylesheets \\
\quad {\bf sources/org/mycore} & Die Wurzel des MyCoRe-Source-Baumes \\
\qquad {\bf datamodel} & Klassen zum Datenmodell \\
\quad \qquad {\bf classifications} & Klassen zur Arbeit mit den Klassifikationen \\
\quad \qquad {\bf ifs} & Klassen zur Arbeit mit dem Internal File System \\
\quad \qquad {\bf metadata} & Klassen zur Arbeit mit den Metdaten \\
\qquad {\bf common} & Klassen, die im gesamten Projekt ben�tigt werden \\
\quad \qquad {\bf xml} & Allgemeine Klassen zur XML Verarbeitung \\
\qquad {\bf backend} & Klassen f"ur die verschiedenen Data Stores \\
\quad \qquad {\bf cm7} & Klassen zur Nutzung des IBM Content Manager 7 \footnote{Die Klassen f�r den IBM Content Manager 7 werden nicht mehr weiterentwickelt!} \\
\quad \qquad {\bf cm8} & Klassen zur Nutzung des IBM Content Manager 8.2 \\
\quad \qquad {\bf filesystem} & Klassen zum Content Store im lokalen Filesystem \\
\quad \qquad {\bf realhelix} & Klassen zum Content Store in einem Helix-Server \\
\quad \qquad {\bf remote} & Klassen zum Zugriff auf Remote-MyCoRe-Daten \\
\quad \qquad {\bf sql} & Klassen zum Zugriff auf relationale Datenbanken mittels SQL-Standart \\
\quad \qquad {\bf videocharger} & Klassen zum Content Store in einen IBM Videocharger \\
\quad \qquad {\bf xmldb} & Klassen zur Speicherung der Daten mittels einer XML:DB \\
\qquad {\bf frontend} & Klassen f"ur die Frontends des MyCoRe-Systems \\
\quad \qquad {\bf cli} & Klassen des Commandline-Tools \\
\quad \qquad {\bf editor2} & Klassen zur Gestaltung von Editoren \\
\quad \qquad {\bf servlets} & Klassen zu Gestaltung von Servlets \\
\qquad {\bf services} & Klassen f"ur weiterf"uhrende Services des MyCoRe-Projektes \\
\quad \qquad {\bf nbn} & Klassen zur Arbeit mit NBN's \\
\quad \qquad {\bf oai} & Klassen zur Arbeit mit OAI Komponenten \\
\quad \qquad {\bf query} & Klassen zur Arbeit mit dem MyCoRe internen Query-System \\
\qquad {\bf user} & Klassen des User- und Rechteverwaltungssystems \\
\quad build.xml & Ant Build-Datei, steuert den MyCoRe Build-Prozess \\
\quad license.txt & Das Lizenz-File des MyCoRe-Projektes, bitte lesen Sie dieses File aufmerksam durch, bevor Sie MyCore einsetzen. \\
\hline
\end{supertabular}
%
%
\section{Konfiguration und "Ubersetzten des Kerns}
%
\begin{enumerate}
\item
MyCoRe verwendet das Apache Ant Build-Tool, um den Quellcode zu �bersetzen und eine vollst�ndige 
Beispiel-Applikation zu erzeugen. Entsprechend der Installationsanleitung des Ant-Paketes sollten Sie zun�chst die
Umgebungsvariable {\tt JAVA\_HOME} und {\tt ANT\_HOME} gesetzt haben. Sollten diese Variablen auf Ihrem System noch nicht
gesetzt sein, k�nnen Sie dies in der Datei {\tt build.sh} (Unix) bzw. {\tt build.cmd} (Windows) nachholen und korrigieren.

\item
Es ist nicht n�tig, weitere Umgebungsvariablen wie etwa den Java CLASSPATH zu setzen. Das MyCoRe Ant Build-Skript 
ignoriert den lokal gesetzten {\tt CLASSPATH} v�llig und generiert stattdessen einen eigenen {\tt CLASSPATH} entsprechend Ihrer Konfiguration. 
Somit k�nnen wir sicherstellen, dass nur die erforderlichen Pakete und Klassen in der richtigen Version verwendet werden. Die Konfiguration der 
Systemumgebung der verwendeten Datenbanken f�r die XML-Speicherung (IBM CM7, IBM CM8, Apache Xindice, eXist) und die Speicherung von Tabellen
�ber JDBC in einer relationalen Datenbank (IBM DB2, MySQL, optional auch andere) wird in der Datei {\tt bin/build.properties}
festgelegt. 

\item
In der Regel werden Sie nur die beiden entsprechenden Bl�cke f�r die verwendete XML-Datenbank ({\tt MCR.XMLStore.*}) und die
verwendete relationale Datenbank ({\tt MCR.JDBCStore.*}) durch kommentieren bzw. auskommentieren der vorgegebenen Zeilen und 
Anpassen der beiden Variablen {\tt MCR.XMLStore.BaseDir} und \\
{\tt MCR.JDBCStore.BaseDir} an die lokalen Installationsverzeichnisse
Ihrer Datenbanksysteme anpassen m�ssen. Die weiteren Variablen steuern die f�r den Betrieb notwendigen JAR-Dateien 
({\tt MCR.*Store.Jars}), eventuell zus�tzlich in den CLASSPATH einzubindende class-Dateien oder Ressourcen ({\tt MCR.*Store.ClassesDirs})
und zur Laufzeit erforderliche native Libraries bzw. DLLs ({\tt MCR.*Store.LibPath}). Passen Sie die Werte entsprechend der
Dokumentation Ihres Datenbankproduktes und der Kommentare in der Datei selbst an.

\item
Sie sollten zun�chst pr�fen, ob ihre Systemumgebung korrekt eingerichtet ist, indem Sie 
\begin{center}
{\tt build.sh info } \qquad bzw. \qquad {\tt build.cmd info }
\end{center}
ausf�hren. Das Ant Build Tool zeigt Ihnen daraufhin die verwendeten JDK- und Ant-Software-Versionen und den
generierten CLASSPATH und LIBPATH (f�r Unix Systeme) an.

\item Sollten Sie festgestellt haben, dass Ihr JDK ab 1.4.x eine andere
Xalan-Version benutzt, f�hren Sie bitte folgende Kommandos aus und pr�fen Sie danach Ihr System erneut.

{\tt
cd \$JAVA\_HOME/jre/lib \newline
mkdir endorsed \newline
cd endorsed \newline
cp \$MYCORE\_HOME/lib/xerces* . \newline
cp \$MYCORE\_HOME/lib/xalan* . \newline
cd \$JAVA\_HOME/lib \newline
ln -s ../jre/lib/endorsed endorsed
}

\item 
Eine �bersicht �ber alle wesentlichen Build-Ziele erhalten Sie mit
\begin{center}
{\tt build.sh usage } \qquad bzw. \qquad {\tt build.cmd usage } 
\end{center}

\item
�bersetzen Sie alle MyCoRe Quellcode-Dateien mit dem Befehl
\begin{center}
{\tt build.sh jar } \qquad bzw. \qquad {\tt build.cmd jar }
\end{center}
Dabei entsteht, abh�ngig von dem von Ihnen gew�hlten Datenbank-System zur Speicherung der XML-Daten eine Jar-Datei 
{\tt lib/mycore-for-[cm7|cm8|xmldb].jar}.

\item
Optional k�nnen Sie auch JavaDoc Quellcode-Dokumentation im HTML-Format generieren lassen, indem Sie
\begin{center}
{\tt build.sh javadocs } \qquad bzw. \qquad {\tt build.cmd javadocs }
\end{center}
aufrufen. Dabei entstehen HTML-Dateien im Verzeichnis {\tt documentation/html}.

\end{enumerate} 
%
%

% Kapitel 3
\chapter{Die MyCore Beispielanwendung}
% Abschnitt 3.1
% %
%
\section{Grundlegender Aufbau und Ziel der Beispielanwendung}
%
%
\subsection{Allgemeines}
Um die Funktionsweise des MyCoRe-Kernes zu testen wurde eine Beispielanwendung basiered auf diesem Kern entwickelt. Sie soll dem Anwender eine voll funktionsf"ahige Applikation in die Hand geben, welche eine Vielzahl von Komponenten des MyCoRe-Projektes nutzt und deren Arbeitweise klar werden l"asst. Um die Anwendung, im weiteren MyCore-Sample genannt, gleichzeitig sinnvoll einsetzten zu k"onnen, wurde als Beispielszenario ein Dokumentserver gew"ahlt, wie er bei vielen potentiellen Nutzern zur Anwendung kommt. Auch soll das MyCore-Sample die Nachfolge des erfolgreichen MILESS-Dokumentservers sein und den Migrationspfad zu MyCoRe hin aufzeigen. \\[2ex]
Analog zum MILESS werden auch hier die Metadaten in drei Gruppen aufgeteilt und von den eigentlichen multimedialen Objekten getrennt behandelt. Zur Modellierung des Datenmodells f"ur die Multimedia-metadaten wurde das allgemein verbreitete Dublin Core Datenmodell herangezogen und umgesetzt. Zur Realisierung wurden folgende Metadaten-Objekte entwickelt:
\begin{itemize}
\item {\bf Document} - Dieses Objekt beinhaltet die eigentlichen Metadaten des Multimedi-Objektes, in unserem Fall im Dublin Core Format, sowie Verweise auf extern gehaltene Daten wie die drei im weiteren genannten. Das Datenmodell besteht aus 15 Feldern, zur Speicherung verschiedener Datentypen gibt es atomare Grundkomponenetn. Weiterhin beinhalten die Metaddaten noch Teile zu ihrer Struktur und Verwaltung.
\item {\bf Klassifikation} - Hier k"onnen lineare oder hierarchich strukturierte Klassifikationen mit ihren Kategorien abgelegt werden. Jeder Verweis auf eine solche Kategorie aus einem Dokument heraus erfolgt mittels der ID der selben und nat"urlich der ID der Klassifikation als Identifiziererpaar. F"ur die Gestaltung der Klassifikationen wurde ein eigens Datenmmodel entwicklet.
\item {\bf LegalEntity} - Auch diese Metadaten sind analog dem {\bf Document} aus atomaren Komponenten zusammengesetzt und repr"asentieren nat"urliche und juristische personen, wie Autoren oder Einrichtunge.
\item {\bf Derivate} - Hinter diesem Begriff verbergen sich Darstellungsformen der eigentlichen multimedialen Objekte. Beispielsweise k"onnen das verschieden Dateiformate eines Images sein. Diese werden an das eigentliche Dokument gebunden.
\end{itemize}
Eine "Ubersicht zu den Beziehungen der Objektteile zueinander soll die folgende Grafik vermitteln. Wir haben uns aus den guten Erfahrungen von MILESS herraus entschlossen, diese Aufteilung beizubehalten, da so Einzelkomponeneten in Projekten schon geladen werden k"onnen, bevor andere Teile, z. B. das Scannen, abgeschlossen sind. \\
\begin{figure}[H]
\begin{center}
\fbox{\includegraphics[scale=0.8]{UserGuide_3Goals_Pic01.jpg}}
\caption{"Ubersicht der Objekte im MyCoRe-Sample}
\end{center}
\end{figure}
Wie man der Grafik entnehmen kann, ist es m"oglich, neben Kategorien in Klassifikationen auch Metadaten der Documnets und LegalEntities hierarchich ineienander zu verschachteln. Auf dieser Basis k"onnen dann Strukturen wie z. B. die eines Buches abgebildet werden. Alle Beziehungen zwischen den Metadaten sind "uber eine eindeutige MCRObjectID gel"ost. \\
\subsection{Die MCRObjectID}
Dreh- und Angelpunkt aller Beziehungen von Metadaten und der daran angekn�pften Objekte ist ein eindeutiger Identifizierer. Im Falle des MyCoRe Projektes hat dieser den Namen {\bf MCRObjectID} erhalten. Er hat f�r alle Matadaten einen einheitlichen Aufbau und wird st�ndig zur Identifizierung genutzt. Auch das API des Systems bietet die M�glichkeit, diese ID automatisch zu generieren.\footnote{Siehe Programmers Guide.} Der Syntax der {\bf MCRObjectID} ist wie folgend festgelegt:\\
\begin{center}
{\bf MCRObjectID = {\it project\_type\_number}}
\end{center}
{\it project} - Dieses Element der ID spezifiziert ein bestimmtes Projekt. Es dient dazu, Daten verschiedener Projekte, welche sich gemeinsam auf einem System befinden, zu unterscheiden. Der Projektname sollte eindeutig sein und besteht aus einem Wort {\bf ohne Sonderzeichen}. Achten Sie immer darauf, dass Sie sich bei Vorhaben mit mehreren MyCoRe-Partnern von Anfang an auf einen Namen pro Projektteilnehmer verst�ndigen. F�r unser Sample haben wir {\bf MyCoReDemoDC} ausgew�hlt. {\bf Dieser Teil der ID ist case sensitive!}\\[2ex]
{\it type} - Jedem Metadatentyp des Datenmodells ist ein Typbezeichner zuzuordnen. Dieser stellt in MyCoRe die Verbindung von der Konfiguration des Metadatums bis hin zu dessen Suche und Pr�sentation dar. Der Typname der MCRObjectID wird immer in Kleinbuchstaben (lower case) innerhalb des Systems verwendet. Achten Sie von Anfang an darauf, dies umzusetzen. Im MyCoRe-Projekt gibt es reservierte Typnamen, diese sind\\
\begin{itemize}
\item {\bf class} - f�r Klassifikationen
\item {\bf derivate} - f�r Derivate
\end{itemize}
Zur Umsetztung des Datenmodells wurden weiterhin die Typen {\bf document} f�r Dokumente und {\bf legalentity} f�r Personen verwendet.
{\it number} - Dieser Teil dient der Identifizierung des Einzelobjektes. Es d�rfen nur nat�rliche Zahlen angegeben werden. Planen Sie Ihre Vergabe der Nummern bei eigenen Objekten sorgf�ltig. Da die Nummern teilweise als Strings verarbeitet werden, ist es sinnvoll, sie mit einer ausreichenden Zahl von Vornullen zu versehen. Dies erledigt das MyCoRe-System automatisch unter Zuhilfenahme des Konfogurationswertes {\tt MCR.metadata\_objectid\_number\_pattern } im File {\it mycore.properties.private }.
\subsection{Das Sample-Datenmodell}
Das Datenmodell des MyCoRe-Samples soll exemplarisch einen kleinen Dokumentserver abbliden, wie er dann f�r gro�e L�sungen �bernommen werden kann. Dabei sollen Multimediale Objekte verschiedenster Art und Herkunft gemeinsam verwaltet werden.

%
\section{Vereinfachte Funktionsprinzipien der Anwendung}
\subsection{User- und Rechtesystem}
\subsection{Klassifikationen}
\subsection{Metadatenmodel}
\subsection{IFS und Content Store}
\subsection{Datenpr"asentation}
\subsection{Interaktive Arbeit mit den Daten}
% Download der Beispieldaten
%
%
\section{Download der Beispielanwendung}
%
%
Nachdem Sie den MyCoRe-kern erfolgreich installiert haben, ist nun die Installation der mitgelieferten Beispielanwendung sinnvoll. Hier k�nnen Sie ein erstes gef�hl daf�r gewinnen, wie eine eigene Anwendung gestaltet sein k�nnte. Das MyCoRe Sample wird f�r alle unterst�tzten Systeme �ber das CVS Repository ausgeliefert. Das Holen der aktuellen Version erfolgt mit dem Kommando
\begin{center}
{\tt cvs -d :pserver:anoncvs@server.mycore.de:/cvs checkout mycore-sample-application}
\end{center}
Nach dem erfolgreichen Checkout erhalten Sie folgende Dateistruktur:\\[2ex]
\bottomcaption{Dateistruktur des MyCoRe Samples}
\tablehead{\hline}
\tabletail{\hline}
\begin{supertabular}{|p{5cm}|p{10cm}|}
\hline
{\bf mycore-sample-application} &  Das Root-Verzeichnis des MyCoRe-Samples \\
\quad {\bf bin} & Das Verzeichnis der Shellscripte \\
\qquad build.sh & Shellscript zum Compilieren unter einem UNIX-System \\
\qquad build.cmd & Shellscript zum Compilieren unter einem UNIX-System \\
\qquad setup.sh & Shellscript, welches die Umgebung f�r das Sample setzt \\
\qquad setup.cmd & Shellscript, welches die Umgebung f�r das Sample setzt \\
\quad {\bf config} & Das Verzeichnis der Konfigurationsdateien \\
\qquad ContentStoreSelectionRules.xml & Das Regelwerk zur Speicherung des Contents in die einzelnen Stores. \\
\qquad FileContentTypes.xml & Eine Liste der verschiedenen Content-Typen \\
\qquad MyCoReDemoDC\_....xml & Die Konfigurationsdateien der einzelnen Sample-Metadatentypen. \\
\qquad SearchMask\_....xml & Die Konfigurationsdateien der Suchmasken \\
\qquad editor-...-nbn.xml & Die Konfigurationsdateien f�r die NBN Funktionalit�t. \\
\qquad ibm-web-... & Zus�tzliche Dateien zur Nutzung von IBM WebSphere. \\
\qquad mycore.properties & Das Master-Property-File f�r MyCoRe \\
\qquad mycore.properties.application & Ein Platzhalter-Property-File f�r andere MyCoRe-Anwendungen. \\
\qquad mycore.properties.classification & Das Property-File f�r den Klassifikationsbereich. \\
\qquad mycore.properties.cm7 & Das Property-File f�r den IBM CM 7.x-Bereich. \\
\qquad mycore.properties.cm8 & Das Property-File f�r den IBM CM 8.x-Bereich. \\
\qquad mycore.properties.ifs & Das Property-File f�r den Bereich des Internal File Systems. \\
\qquad mycore.properties.logger & Das Property-File f�r den Logger-Bereich. \\
\qquad mycore.properties.nbn & Das Property-File f�r den NBN-Bereich. \\
\qquad mycore.properties.oai & Das Property-File f�r den OAI-Bereich. \\
\qquad mycore.properties.private & Das Property-File f�r den Bereich, in welchem die meisten Anpassungen der jeweiligen Sample Anwender get�tigt werden m�ssen. \\
\qquad mycore.properties.remote & Das Property-File f�r den Bereich der Remote-Zugriffe. \\
\qquad mycore.properties.user & Das Property-File f�r den User- und Rechteverwaltungs-Bereich. \\
\qquad mycore.properties.xmlsortkeys & Das Property-File f�r den Pr�sentationsbereich. \\
\qquad reservation.xml & ein Konfigurations-File f�r den NBN-Bereich. \\
\qquad {\bf user} & In diesem Verzeichnis sind alle Gruppen, Nutzer und Privilegien abgelegt, welche f�r das Sample ben�tigt werden. \\
\qquad web.xml & Das Konfigurations-File f�r die Servlet-Engine \\
\quad {\bf content} & Hier finden Sie alle Beispieldaten. \\
\qquad {\bf classifications} & Die Daten der Klassifikationen. \\
\qquad {\bf documents} & Die Daten der Dokumente. \\
\qquad {\bf derivates} & Die Daten der Derivate. \\
\qquad {\bf legalentities} & Die Daten der LegalEntities. \\
\qquad {\bf objects} & Die eigentlichen multimedialen Objekte. \\
\quad {\bf schema} & Das Verzeichnis der XML-Schema-Files f�r das Sample. \\
\quad {\bf sources} & Hier sind zus�tzliche Java-Klassen abgelegt, welche nur f�r diese Anwendung g�ltig sind. Die Struktur ist analog dem MyCoRe-Kern. \\
\quad {\bf stylesheets} & Die verwendeten XSLT-Stylesheets der Pr�sentation dieses Samples. \\
\quad build.xml & Konfigurations-File f�r die Arbeit mit ANT \\
\quad license.txt & Das Lizenz-File des MyCoRe-Projektes, bitte lesen Sie dieses File aufmerksam durch, bevor Sie MyCore einsetzen. \\
\hline
\end{supertabular}

\section{Konfiguration zur Arbeit mit den Beispieldaten}
% Grundlegende Konfigurationen
%
%
\subsection{Grundlegende Konfigurationen}
%
%
Dieser Abschnitt besch�ftigt sich mit der Konfiguration der Beispielanwendung in allgemeinen Bereichen wie JDBC, Logger, usw. Die MyCoRe Konfigurationen f�r das Beispiel finden Sie im Verzeichnis {\tt /config}.
%
%
\subsubsection{Pfade und Systemumgebung anpassen}

Auch f�r die Zusammenstellung und Installation der Beispiel-Anwendung verwendet MyCoRe das Apache Ant Build-Tool. 
Entsprechend der Installationsanleitung des Ant-Paketes sollten Sie zun�chst die 
Umgebungsvariablen {\tt JAVA\_HOME} und {\tt ANT\_HOME} gesetzt haben. Sollten diese Variablen auf Ihrem System noch nicht 
gesetzt sein, k�nnen Sie dies in der Datei {\tt build.sh} (Unix) bzw. {\tt build.cmd} (Windows) nachholen und korrigieren. 

Die MyCoRe Beispiel-Anwendung verwendet die Dateien aus dem MyCore Kern, insbesondere die erzeugte Datei 
{\tt mycore-for[cm7|cm8|xmldb].jar} und die Konfigurationsdatei f�r den Build-Prozess {\tt build.properties}.
Wenn die Verzeichnisse {\tt mycore} und \\
{\tt mycore-sample-application} auf Ihrem System nicht in einem gemeinsamen
�bergeordneten Verzeichnis liegen, k�nnen Sie im build-Skript auch die Umgebungsvariable {\tt MYCORE\_HOME} auf einen entsprechend
korrigierten Wert setzen, da die Vorgabe {\tt ../mycore} ist.

Sie sollten zun�chst pr�fen, ob ihre Systemumgebung korrekt eingerichtet ist, indem Sie 
\begin{center} 
{\tt build.sh info } \qquad bzw. \qquad {\tt build.cmd info } 
\end{center} 
ausf�hren. Das Ant Build Tool zeigt Ihnen daraufhin die verwendeten JDK- und Ant-Software-Versionen und den 
generierten CLASSPATH und LIBPATH (f�r Unix Systeme) an. 
Eine �bersicht �ber alle wesentlichen Build-Ziele erhalten Sie mit 
\begin{center} 
{\tt build.sh usage } \qquad bzw. \qquad {\tt build.cmd usage } 
\end{center} 

\subsubsection{JDBC-Treiber konfigurieren}

Im MyCoRe-Projekt werden ein Teil der Organisations- und Metadaten in klassischen relationalen Datenbanken gespeichert. 
Um die Arbeit mit verschiedenen Anbietern m�glichst einfach zu gestalten, wurde die Arbeit mit dieser Datenbank gegen die JDBC-Schnittstellen programmiert.\\
In der Konfigurationsdatei {\it mycore.properties.private} legen Sie im Parameter \\
{\bf MCR.persistence\_sql\_driver} fest, welcher JDBC-Treiber verwendet werden soll. 
Weiterhin m�ssen Sie die Variable {\bf MCR.persistence\_sql\_database\_url} anpassen, die die JDBC URL f�r Verbindungen
zu Ihrer Datenbank festlegt. Der DB2 Library-Name {\bf LIB} muss durch den aktuellen (z. B. {\bf ICMNLSDB}) ersetzt werden. Analog dazu muss der User {\bf ODBC} bei MySQL durch den entsprechenden Nutzer (z. B. {\bf mcradmin}) erstzt werden. Beachten Sie dabei insbesondere, dass meist Gross/Kleinschreibung relevant ist!
Weiterhin k�nnen Sie die minimale und maximale Anzahl der gleichzeitigen Verbindungen zur Datenbank festlegen.

\begin{verbatim}
# JDBC parameters for connecting to DB2
#MCR.persistence_sql_database_url=jdbc:db2:LIB
#MCR.persistence_sql_driver=COM.ibm.db2.jdbc.app.DB2Driver

# JDBC parameters for connecting to MySQL
MCR.persistence_sql_database_url=jdbc:mysql://localhost/mycore?user=ODBC
MCR.persistence_sql_driver=org.gjt.mm.mysql.Driver

MCR.persistence_sql_init_connections=1
MCR.persistence_sql_max_connections=5
\end{verbatim}

\subsubsection{XML-Datenbank konfigurieren}

Den Typ der zur Laufzeit des Systems zu w�hlenden XML-Datenbank haben Sie bereits in der Datei {\tt build.properties} 
�ber den Parameter {\tt MCR.XMLStore.Type} festgelegt. Dabei sind die Werte {\tt cm7} f�r IBM Content Manager 7, 
{\tt cm8} f�r IBM Content Manager 8, oder {\tt xmldb} f�r eine XML:DB kompatible XML-Datenbank wie Apache Xindice, eXist 
oder Tamino (Software AG) m�glich. Die MyCoRe Beispiel-Anwendung verwendet automatisch diese Konfiguration, sie 
m�ssen nun nur noch die Parameter der einzelnen XML-Datenbanken konfigurieren.

\subsubsection{Konfiguration von IBM Content Manager 7}

Falls Sie {\tt MCR.XMLStore.Type=cm7} verwenden, passen Sie in der Datei \\
{\tt mycore.properties.private} die Variablen {\tt MCR.persistence\_cm7\_*} an.

\subsubsection{Konfiguration von IBM Content Manager 8}

Falls Sie {\tt MCR.XMLStore.Type=cm8} verwenden, passen Sie in der Datei \\
{\tt mycore.properties.private} die Variablen {\tt MCR.persistence\_cm8\_*} an.
Die Eintr�ge sind eigentlich selbsterkl�rend, so dass an dieser Stelle auf weitere Erl�uterungen verzichtet werden kann.

\begin{verbatim}
# Special values for the persistence layer
MCR.persistence_cm8_max_connections=2
MCR.persistence_cm8_library_server=ICMNLSDB
MCR.persistence_cm8_user_id=icmadmin
MCR.persistence_cm8_password=????????

# Special values for the text search engine
MCR.persistence_cm8_textsearch_ccsid=819
MCR.persistence_cm8_textsearch_lang=DE_DE
MCR.persistence_cm8_textsearch_indexdir=/home/db2inst1/sqllib/db2ext/indexes
MCR.persistence_cm8_textsearch_workingdir=/home/db2inst1/sqllib/db2ext/indexes
\end{verbatim}

Da der Persitence Layer CM8 auf einem Mapping der XML-Daten nach DB2 besteht, m�ssen die CM8 ItemTypes vor dem Laden der Daten separat angelegt werden. Dies geschiet mittels
\begin{center} 
{\tt build.sh create.metastore } \qquad bzw. \qquad {\tt build.cmd create.metastore } 
\end{center} 

\subsubsection{Die Nutzung von eXist als XML:DB Backend}

Falls Sie {\tt MCR.XMLStore.Type=xmldb} verwenden, passen Sie in der Datei \\
{\tt mycore.properties.private} die Variablen {\tt MCR.persistence\_xmldb\_driver} \\
und {\tt MCR.persistence\_xmldb\_database\_url} an.

\begin{verbatim}
MCR.persistence_xmldb_driver=org.exist.xmldb.DatabaseImpl
MCR.persistence_xmldb_database_url=xmldb:exist://localhost:8081/db/mycore
MCR.persistence_xmldb_database=exist
\end{verbatim}

Starten Sie nun den eXist-Client ({\it <eXist-installdir>/bin/client.sh} bzw. {\it cliend.cmd}) und f�hre Sie folgende Kommandos zum anlegen der Stores unter eXist aus:

\begin{verbatim}
mkcol mycore
chown guest guest mycore
cd mycore
mkcol legalentity
chown guest guest legalentity
mkcol document
chown guest guest document
mkcol derivate
chown guest guest
quit
\end{verbatim}

\subsubsection{Speicherung von Datei-Inhalten konfigurieren}

In der Grundkonfiguration verwendet die Beispiel-Applikation zur Speicherung der Datei-Inhalte der Derivate das lokale Dateisystem. Passen Sie in der Datei \\
{\tt mycore.properties.private} die Variable {\tt MCR.IFS.ContentStore.FS.BaseDirectory} 
an und erzeugen Sie ein neues, leeres Verzeichnis am angegebenen Ort.


%Laden der Beispieldaten
%
%
\subsection{Das Metadatensystem}
%
%
In diesem Abschnitt son nun beschrieben werden, welche Konfigurationen f�r die Speicherung der Metadaten zu bearbeiten sind. Als erstes m�ssen Sie einstellen, mit welchem Persistence-Store f�r die Metadaten Sie arbeiten wollen. Das h�ngt nat�rlich auch von den bisherigen Installationsschritten ab und wurde schon behandelt. Nun folgen die Einstellung f�r den konkreten Layer.
\subsubsection{Die Nutzung von eXist als XML:DB Backend}
Hierzu m�ssen Sie die folgenden Daten im Konfigurationsverzeichnis des Samples unter {\it \$MYCORE\_SAMPLE\_HOME/config/mycore.properties.private} entsprechend Ihren Gegebenheiten anpassen. Habe Sie die Standardvorgehensweise befolgt, sollte hier nichts zu tun sein.
\begin{verbatim}
MCR.persistence_xmldb_driver=org.exist.xmldb.DatabaseImpl
MCR.persistence_xmldb_database_url=xmldb:exist://localhost:8081/db/mycore
MCR.persistence_xmldb_database=exist
\end{verbatim}
Weiterhin m�ssen die Files {\it exist.jar}, {\it xmldb.jar} und {\it xmlrpc-1.1.jar} nach {\it \$MYCORE\_HOME/lib} kopiert werden. Achten Sie darauf, dass es in diesem verzeichnis kein File namens {\it xindice.jar} oder ein *.jar File einer anderen XML:DB gibt.\\[2ex]
Starten Sie nun den eXist-Client und f�hre Sie folgende Kommandos zum anlegen der Stores unter eXist aus:
\begin{verbatim}
mkcol mycore
chown guest guest mycore
cd mycore
mkcol legalentity
chown guest guest legalentity
mkcol document
chown guest guest document
mkcol derivate
chown guest guest
quit
\end{verbatim}
Nun sollte das Sample zum Laden der Metadaten bereit sein.
%
%
\subsubsection{Lade der Daten}
habe Sie alles Vorbereitet, so k�nnen nun die Beispiel-Metadaten geladen werden. Dies geschieht mit
\begin{itemize}
\item {\tt ant legal} oder {\tt bin/build.sh legal} f�r die Legalentities und dann
\item{\tt ant document} oder {\tt bin/build.sh document} f�r die Documents
\end{itemize}


%Das User- und Rechtesystem
%
%
\subsection{Das User- und Rechtesystem}
%
%
Im MyCoRe-Sample wird zur Demonstartion des User- und Rechtesystems auch eine konkrete Beispielkonfiguration f�r diesen Bereich mitgeliefert. Ein allgemeiner �berblick zu diesem Teilsystem wurde bereits weiter oben gegeben.\\
Im Konfigurationsverzeichnis des Samples unter {\it \$MYCORE\_SAMPLE\_HOME/config} ist zuerst das entsprechende Property-File mit dem Namen {\it mycore.properties.user} n�her zu betrachten. Sie finden hier Festlegungen in der folgenden Form:
\begin{verbatim}
# The name of the Java class of the UserStore
MCR.userstore_class_name=org.mycore.backend.sql.MCRSQLUserStore

# SQL tables for the user management
MCR.users_store_sql_table_users=MCRUSERS
MCR.users_store_sql_table_groups=MCRGROUPS
MCR.users_store_sql_table_group_members=MCRGROUPMEMBERS
MCR.users_store_sql_table_group_admins=MCRGROUPADMINS
MCR.users_store_sql_table_privileges=MCRPRIVS
MCR.users_store_sql_table_privs_lookup=MCRPRIVSLOOKUP

# The configuration for the superuser
MCR.users_superuser_username=gandalf
MCR.users_superuser_userpasswd=alleswirdgut
MCR.users_superuser_groupname=zauberer

# The configuration for the guestuser
MCR.users_guestuser_username=aragorn
MCR.users_guestuser_userpasswd=mensch
MCR.users_guestuser_groupname=menschen
\end{verbatim}
Der Eintag des Parameters {\bf MCR.userstore\_class\_name} beschreibt den zu verwendenden Persitence Layer f�r die vom User-System verwendeten relationalen Datenbanken. Wir haben hier einen generischen Store entwickelt der mit allen SQL-konformen Datenbanken wie DB2, MySQL usw. laufen sollte. Sie m�ssen also diesen Wert normalerweise nicht �ndern.\\
Die n�chste Sektion in diesem Property-File legt die Tabellennamen der f�r das User-System ben�tigten Tabellen fest. F�r die Ausf�hrung des Beispieles k�nnen Sie die Einstellung �bernehmen. Achten Sie jedoch sp�ter darauf, dass Sie die Tabellennamen nur in einem Projekt verwenden!\\
Die letzten beiden Abschnitte legen Vorgaben f�r zwei Benutzer fest, welche automatisch bei der Initialisierung des User-Systems angelegt werden. Der erste ist der Superuser innerhalb dieses MyCoRe-Projektes, der andere ein ganz simpler Anwender. Auch diese Werte k�nnen Sie f�r den Anfang so lassen, f�r ein eigenes Projekt, wie es weiter unten beschrieben wird, sollten Sie hier Anpassungen vornehmen. Eine Liste aller Gruppen und Benutzer des Beispieles finden Sie im Abschnitt zur allgemeinen Beschreibung dieses Subsystems.\\
Alle f�r das Beispiel erforderlichen Privileg-, Gruppen- und Benutzerdateien stehen im Verzeichnis {\it \$MYCORE\_SAMPLE\_HOME/config/users}. Sie k�nnen diese mit
 
\begin{verbatim}
cd \$MYCORE\_SAMPLE\_HOME; bin/build.sh userdb
\end{verbatim} 
oder 
\begin{verbatim}
cd \$MYCORE\_SAMPLE\_HOME; ant userdb
\end{verbatim} 


in einem Arbeitsgang laden.\footnote{Welches Kommando Sie nutzen h�ngt von Ihrer Umgebungseistellung ab.} Der Aufruf initialisiert zuerst das User-System und legt die in der Konfiguration angegebenen Tabellen an. Anschlie�end werden die Privilegien, Gruppen und Benutzer geladen. Bitte achten Sie darauf, dass die vorgegebene Reihenfolge eingehalten wird, da es innerhalb des Beispiels Abh�ngigkeiten zwischen Usern und Gruppen gibt.\\
Nun sollten Sie �ber ein komplettes User-System, wie es zur Demonstartion des Samples erforderlich ist, verf�gen.


%Arbeiten mit dem MyCoRe Command Line Interface
\section{Arbeiten mit dem MyCoRe Command Line Interface}
\subsection{Erzeugen der Skripte mycore.sh / mycore.cmd}
Neben dem MyCoRe-Web-Interface kann f�r administrative 
Zwecke das MyCoRe Command Line Interface (CLI) genutzt werden.
Zum Aufruf des CLI m�ssen Sie zun�chst die erforderlichen Shell-Skripte erzeugen:
\begin{center} 
{\tt build.sh script } \qquad bzw. \qquad {\tt build.cmd script } 
\end{center} 
Dieser Aufruf generiert die Shell-Skripte {\tt bin/mycore.sh} (Unix) bzw. 
{\tt bin/mycore.cmd} (Windows).

\subsection{Aufruf des CommandLineInterface}
Starten Sie das MyCoRe Command Line Interface durch Aufruf von
{\tt bin/mycore.sh} (Unix) bzw. {\tt bin/mycore.cmd} (Windows).
Sie erhalten eine �bersicht �ber die verf�gbaren Befehle durch
Eingabe von
\begin{center}
{\tt help }
\end{center}
Sie verlassen das CommandLineInterface durch Eingabe von
\begin{center}
{\tt quit } \qquad oder \qquad {\tt exit } 
\end{center}


%Erzeugen und Konfigurieren der Web-Anwendung
\section{Erzeugen und Konfigurieren der Web-Anwendung}
\subsection{Erzeugen der Web-Anwendung}
Durch Eingabe von
\begin{center} 
{\tt build.sh webapp } \qquad bzw. \qquad {\tt build.cmd webapp } 
\end{center} 
wird die MyCoRe Sample Web Application im Verzeichnis {\tt webapps}
erzeugt. Alternativ k�nnen Sie auch ein Web Application Archive (war)
erzeugen, indem Sie
\begin{center}
{\tt build.sh war } \qquad bzw. \qquad {\tt build.cmd war }
\end{center}
aufrufen. 

Das MyCoRe Build-Script kopiert beim Erzeugen der Web Applikation
auch alle externen, erforderlichen jar-Dateien Ihrer verwendeten 
Datenbank-Systeme (IBM Content Manager / DB2, MySQL, eXist) in das
Verzeichnis {\tt WEB-INF/lib}, entsprechend den Vorgaben Ihrer
Konfiguration in {\tt build.properties}. Beachten Sie dazu bitte die 
Hinweise in der Ausgabe beim Erzeugen der Web Application.

\subsection{Konfiguration des Web Application Server}
\subsubsection{ Tomcat}
\subsubsection{ Websphere}
\subsection{Das Layout-Setvlet und das Zusammenspiel der Servlets untereinander}
\subsection{Die Nutzung des Editor-Servlets}


\section{Zusammenarbeit mit anderen Installationen (Remote)}
\section{Vom Sample zum eigenen Dokumentserver}
Hier sollen Anpassungsschritte detailliert erkl"art werden.
\subsection{Anpassungen des Layout an eigene Bed�rfnisse}
\subsection{Weitere User und Gruppen / Nutzung zur selektierten Darstellung durch Stylesheets}
\subsection{Metadatenvererbung}
\subsection{Nutzung der OAI Schnittstelle}

\subsubsection*{Grundlagen}

Die Open Archives Initiative (\textsl{http://www.openarchives.org/}) hat 2001 ein offenes Protokoll f"ur das Sammeln (Harvesting) von Metadaten vorgestellt. Dies geschah vor dem Hintergrund, dass g"angige Suchmaschinen im WWW f"ur
die wissenschaftliche Nutzung wegen der i.d.R. un"uberschaubaren Treffermenge und der fehlenden Qualit"at der angebotenen Treffer kaum nutzbar sind. Das \textbf{Open Archives Initiative Protocol for Metadata Harvesting 
(OAI-PMH)} liegt mittlerweile in der Version 2.0 vor.\linebreak 
Das OAI-PMH dient zur Kommunikation zwischen \textbf{Data Providern} und \textbf{Service Providern}. Unter einem Data Provider versteht man hierbei ein Archivierungssystem, dessen Metadaten von einem (oder mehreren) Service Provider(n) abgerufen werden, der diese als Basis zur Bildung von Mehrwertdiensten benutzt (z.B. der Suche "uber viele Archive gleichzeitig).\linebreak
Zum besseren Verst"andnis der weiteren Dokumentation f"uhre ich hier die wichtigsten Definitionen kurz an:
\begin{itemize}
\item Ein \textbf{Harvester} ist ein Client, der OAI-PMH Anfragen stellt. Ein Harvester wird von einem Service Provider betrieben, um Metadaten aus Repositories zu sammeln.
\item Ein \textbf{Repository} ist ein "uber das Netzwerk zug"anglicher Server, der OAI-PMH Anfragen verarbeiten kann, wie sie im Open Archives Initiative Protocol for Metadata Harvesting 2.0 vom 2002-06-14 beschrieben werden (http://www.openarchives.org/OAI/openarchivesprotocol.html). Ein Repository wird von einem Data Provider betrieben, um Harvestern den Zugriff auf Metadaten zu erm"oglichen.
\end{itemize} 

Der f"ur MyCoRe und Miless implementierte OAI Data Provider ist zertifiziert und erf"ullt den OAI-PMH 2.0 Standard.

\subsubsection*{Der OAI Data Provider}

MyCoRe bietet ein extrem flexibles Klassifikations-/Kategoriensystem. Ein OAI-Repository kann hiervon nur eine Klassifikation zur Strukturierung der Metadaten abbilden, d.h. einer MyCoRe-Klassifikation wird zu einem OAI-Repository. Es werden nur
Metadaten zu Archivaten an den Harvester ausgeliefert, die in genau dieser MyCoRe-Klassifikation erfasst sind.
Zur weiteren Einschr"ankung kann eine weitere Klassifikation angegeben werden, die f"ur den OAI Data Provider aber nicht
strukturbildend ist.\linebreak
Sollen weitere Daten "uber OAI zug"anglich gemacht werden, so bietet der OAI Data Provider die M"oglichkeit, unter verschiedenen Namen mehrere Servlet-Instanzen zu betreiben, wobei eine Instanz jeweils ein OAI-Repository darstellt.

\subsubsection*{Installation}

Zur Einbindung des OAI Data Providers m"ussen Eintragungen in den Deployment Descriptor des Servletcontainers und 
in die mycore.properties erfolgen.

\paragraph*{Der Deployment Descriptor}

F"ur jedes OAI-Repository muss eine Servlet-Instanz in den Deployment Descriptor nach folgendem Muster eingetragen werden:

\begin{verbatim}
  <servlet id="OAIDataProvider">
    <servlet-name>
      OAIDataProvider
    </servlet-name>
    <servlet-class>
      org.mycore.services.oai.MCROAIDataProvider
    </servlet-class>
  </servlet>
  <servlet-mapping>
    <servlet-name>
      OAIDataProvider
    </servlet-name>
    <url-pattern>
      /servlets/OAIDataProvider
    </url-pattern>
  </servlet-mapping>
\end{verbatim}

\paragraph*{Die mycore.properties}

Bei den einzurichtenden Properties ist zwischen \textsl{instanzunabh"angigen} und \textsl{instanzabh"angigen} Properties
zu unterscheiden. Instanzunabh"angige Properties sind hierbei f"ur jedes angebotene OAI-Repository g"ultig, instanzabh"angige
Properties beziehen sich auf das jeweilige OAI-Repository.

\subparagraph*{Instanzunabh"angige Properties}

\begin{itemize}
\item \verb MCR.oai.adminemail=admin@uni-irgendwo.de \textbf{ (notwendig)} Der Administrator der OAI-Repositories.
\item \verb MCR.oai.resumptiontoken.dir=/mycore/temp \textbf{ (notwendig)} Ein Verzeichnis, in welches der OAI Data
  Provider Informationen "uber Resumption Token ablegt.
\item \verb MCR.oai.resumptiontoken.timeout=48 \textbf{ (notwendig)} Die Zeit (in Stunden), f"ur die die Informationen
  "uber die Resumption Token nicht gel"oscht werden. Da das L"oschen nur erfolgt, wenn auf ein OAI-Repository
  zugegriffen wird, k"onnen die Dateien evtl. auch l"anger aufgehoben werden.
\item \verb MCR.oai.maxreturns=50 \textbf{ (notwendig)} Die maximale L"ange von Ergebnislisten, die an einen Harvester
  zur"uckgegeben werden. "Uberschreitet eine Ergebnisliste diesen Wert, so wird ein Resumption Token angelegt.
\item \verb MCR.oai.queryservice=org.mycore.services.oai.MCROAIQueryService \textbf{ (notwendig)} Die Klasse, die f"ur
  das Archiv das Query-Interface implementiert. F"ur Miless ist dies \verb miless.oai.OAIService .
\item \verb MCR.oai.metadata.transformer.oai_dc=MyCoReOAI-mycore2dc.xsl \textbf{ (notwendig)} Das Stylesheet, das die
  Transformation aus dem im Archiv benutzten Metadatenschema in das f"ur OAI benutzte OAI Dublin Core Metadatenschema
  durchf"uhrt. Wenn sich das im Archiv benutzte Metadatenschema "andert, muss dieses Stylesheet angepasst werden.
  Optional k"onnen weitere Stylesheets angegeben werden, die einen Harvester mit anderen Metadatenformaten versorgen,
  z.B. \linebreak \verb MCR.oai.metadata.transformer.rfc1806=MyCoReOAI-mycore2rfc.xsl. \linebreak Diese Stylesheets 
  benutzen als Eingabe das Ergebnis des ersten Stylesheets.
\end{itemize} 

\subparagraph*{Instanzabh"angige Properties}

Bei instanzabh"angigen Properties wird der im Deployment Descriptor verwendete Servletname zur Unterscheidung f"ur
die einzelnen Repositories verwendet.

\begin{itemize}
\item \verb MCR.oai.repositoryname.OAIDataProvider=Test-Repository \textbf{ (notwendig)} Der Name
  des OAI-Repositories.
\item \verb MCR.oai.repositoryidentifier.OAIDataProvider=mycore.de \textbf{ (notwendig)} Der Identifier des OAI-Repositories 
  (wird vom Harvester abgefragt).
\item \verb MCR.oai.setscheme.OAIDataProvider=MyCoReDemoDC_class_1 \textbf{ (notwendig)} Die MyCoRe-Klassifikation, die
  zur Bildung der Struktur des OAI-Repositories verwendet wird.
\item \verb MCR.oai.restriction.classification.OAIDataProvider=MyCoReDemoDC_class_2 \textbf{ (optional)} Die
  MyCoRe-Klassifikation, die zur Beschr"ankung der Suche verwendet wird.
\item \verb MCR.oai.restriction.category.OAIDataProvider=dk01 \textbf{ (optional)} Die
  MyCoRe-Kategorie, die zur Beschr"ankung der Suche verwendet wird.
\item \verb MCR.oai.friends.OAIDataProvider=miami.uni-muenster.de/servlets/OAIDataProvider \textbf{ (optional)} Unter
  dieser Property k"onnen weitere (bekannte und zertifizierte) OAI-Repositories angegeben werden, um den Harvestern
  die Suche nach weiteren Datenquellen zu vereinfachen.
\end{itemize} 

\subsubsection*{Test}

Um zu testen, ob das eigene OAI-Repository funktioniert, kann man sich des Tools bedienen, das von der \textsl{Open Archives
Initiative} unter \textbf{http://www.openarchives.org} zur Verf"ugung gestellt wird. Unter dem Men"upunkt \textbf{Tools}
wird der \textbf{OAI Repository Explorer} angeboten.

\subsubsection*{Zertifizierung und Registrierung}

Ebenfalls auf der oben angegebenen Website findet sich unter dem Men"upunkt \textbf{Community} der Eintrag
\textbf{Register as a data provider}. Dort kann man anfordern, das eigene Repository zu zertifizieren und
zu registrieren. Die Antwort wird an die in den Properties eingetragene EMail-Adresse geschickt.
\subsection{Arbeiten mit NBN}
\subsection{M"ogliche Workflow-Szenarien im Bibliotheksumfeld}
\section{Einbindung weiterer Content Stores (Helix \& Co.)}
% Hints \& Tips / Troubleshooting
%
%
\section{Hints \& Tips / Troubleshooting}
%
%
\subsection{Nach dem Einspielen eines DB2/NSE bringt der CM Fehler}
{\bf Fehlerbild: } Nach dem Einspielen eines DB2/NSE FixPack fehlen die ICM-Umgebungsvariablen.\\
{\bf Ursache: } Im File {\it \$DB2INST/sqllib/db2profile} fehlt nach dem Update der per Hand eingetragene Aufruf des Scripts zum setzten der CM Umgebungsvariablen. Dieser Aufruf muss wieder eingef�gt werden.
%
\subsection{Fehler bei der Kommunikation der Servlets untereinander}
{\bf Fehlerbild: } Die Servlets verlieren in der Abfolge ihrer Nutzung Informationen �ber den Zustand der Session.\\
{\bf Ursache: } Sie haben die Nutzung von Coockies komplett abgeschaltet. Gestatten Sie wenigstens Coockies von den eigenen WEB-Seiten in der Konfiguration Ihres WEB-Browsers.


\chapter{Erstellen einer eigenen Anwendung auf Basis des MyCoRe-Kernes}
(am Beispiel des Papyrus-Projektes)
\section{Erforderliche Schritte zu einer Sammlungsanwendung mit eigenen Metadaten}
\begin{flushright}
{\em macht Jens}
\end{flushright}
\chapter{Weitere Anwendungen (Archivl"osung)}
% Glossar
%%
% UserGuide - Glossar
%
\chapter*{Glossar}
{\bf NBN} \\[1.5ex]
Was ist eigentlich NBN??? \\[2ex]
{\bf OAI} \\[1.5ex]
Was ist eigentlich OAI??? \\[2ex]
{\bf XML} \\[1.5ex]
Was ist eigentlich XML??? \\[2ex]
{\bf XSLT} \\[1.5ex]
Was ist eigentlich XSLT??? \\[2ex]
%
%

\end{document}
