%
%
\chapter{Voraussetzungen f�r eine MyCoRe Anwendung}
%
%
\section{Vorabbemerkungen}
%
%
Das MyCoRe-Projekt ist so designed, dass es dem einzelnen Anwender frei steht,
welche Komponenten er f�r die Speicherung der Daten verwenden will. Dabei spielt nat�rlich das verwendete Betriebssystem eine wesentliche Rolle. Jedes System hat seine eigenen Vor- und Nachteile, die an dieser Stelle aber nicht dikutiert werden sollen. Vielmehr wollen wir es dem Anwender �berlassen, in welchem System er f�r seine Anwendung die gr��ten Vorteile sieht. Nachfolgend finden Sie eine Tabelle der wesentlichen eingesetzten Komponenten entsprechend des gew�hlten Basissystems. \\[2ex]
\small
\bottomcaption{MyCoRe Komponenten�bersicht}
\tablehead{\hline}
\tabletail{\hline}
\begin{supertabular}{|p{2cm}|p{6cm}|p{6cm}|}
\hline
 & {\bf IBM CM Basis} & {\bf freie Software} \\[1,5ex] \hline
Metadaten Store & IBM CM 8.2 - parametrische und Volltextsuch mittels XPath Abfragen & eXist - parametrische Suche mittels XPath Abfragen \\ \hline
TextSearch & IBM DB2 NSE & lucene  \\ \hline
Datenbank & IBM DB2 8.1 oder Oracle & MySQL 4.x \\ \hline
Objekt & Filesystem, & Filesystem,  \\
Store & IBM CM 8.2 Ressource Manager, &  \\ 
 & IBM Video Charger 8, & IBM Video Charger 8, \\
 & Helix Server & Helix Server  \\  \hline
Systembasis & IBM AIX & alle Systeme mit Java \\
 & Sun Solaris & \\
 & MS Windows & \\
\hline
\end{supertabular}
\normalsize
%
%
%
%
\section{Hinweise zur Installation des IBM Content Manager 8.2}
%
%
\subsection{Der IBM Content Manager unter AIX}
An dieser Stelle soll eine Kurzbeschreibung der Installation des IBM Content Managers 8.2 f"ur AIX von Holger K"onig, IBM Deutschland GmbH, wiedergegeben werden. \\[2ex]
\subsubsection{Vorbereitung}
\begin{enumerate}
\item Installieren Sie das AIX Betriebssystem mit dem Release 4.3.3 ML 10, 5.1 ML 01 oder 5.2.
\item Sorgen Sie daf"ur, dass 'Cultural Conversion' und 'Language' auf English US eingestellt ist.
\item Aktivieren Sie die Netzanbindung inklusive DNS.
\item F"ur die Betriebssystem-Releases 4.3.3 und 5.1 muss Java 1.3.1 entsprechend der Anleitung installiert werden. Erweitern Sie in {\it /etc/environment }  die {\it PATH} Variable um {\it /usr/java131/jre/bin} und {\it /usr/java131/bin}. Wenn Sie auch das Paket {\bf Java131.ext.java3d } mit installieren wollen, m"ussen Sie vorher die Pakete \linebreak[4]
{\bf OpenGL.OpenGL\_X.adt} und {\bf OpenGL.OpenGL\_X.rte} installiert haben.
\item Installieren Sie den VAC Compiler Version 5.x oder 6.0 entsprechend der Anleitung und tragen Sie den Suchpfad unter {\it PATH} im File {\it /etc/environment } mit ein.
\item Aktivieren Sie das Lizenzsystem {\bf ifor} und tragen Sie sie Compilerlizenzen ein.
\end{enumerate}
%
\subsubsection{DB2}
\begin{enumerate}
\item Legen Sie bitte die nachfolgenden Benutzer an. Zusammen sollten je nach geplanten Anwendungen 4-8 GB Plattenplatz eingeplant werden. Achten Sie darauf, dass man sich in die Accounts einloggen kann, z. B. via Telnet!
\begin{itemize}
\item User {\bf dasusr1} $\rightarrow$ Group {\bf dasadm1}
\item User {\bf db2inst1} $\rightarrow$ Group {\bf db2grp1}
\item User {\bf db2fenc1} $\rightarrow$ Group {\bf db2fgrp1}
\end{itemize}
\item Kopieren Sie das File {\it ese.sbcs.tar.Z} von der CD {\bf 'DB2 8.1 with FP1'} und entpacken Sie dieses.
\item {\tt ./db2setup}
\item W"ahlen Sie {\bf Install Products} $\rightarrow$ {\bf DB2 UDB Enterprise Server Edition}. Folgen Sie den Schritten:
\begin{itemize}
\item {\bf Next}
\item {\bf Accept License}
\item Auswahl 'Custom' $\rightarrow$ {\bf Next}
\item Auswahl 'Install DB2 UDB Enterprise Server Edition on this computer' $\rightarrow$ {\bf Next}
\item Standartwerte lassn, 'Appliction Development Tools' zus"atzlich ausw"ahlen
\item Sprache 'Englisch' beibehalten $\rightarrow$ {\bf Next}
\item DAS User : Standartwert {\bf db2as} wenn m"oglich beibehalten, Password setzen $\rightarrow$ {\bf Next}
\item Erzeugen der DB2 Instanz durch Auswahl 'Create a DB2 instance - 32 bit' $\rightarrow$ {\bf Next}
\item Auswahl 'Single-partition instance' $\rightarrow$ {\bf Next}
\item Eintrag des DB2 Instance owner : Standartwert ({\bf db2inst1}) m"oglichst lassen, Password setzten $\rightarrow$ {\bf Next} \footnote{\label{fn1} Achten Sie darauf, keine exotischen Sonderzeichen zu nehmen, das macht im CM Probleme!}
\item Eintrag des DB2 Fenced users : Standartwert ({\bf db2fenc1}) m"oglichst lassen, Password setzten $\rightarrow$ {\bf Next} %\footnotemark[\ref{fn1}]
\item Instance TCPIP : Auswahl 'Configure' $\rightarrow$ Service Name : db2c\_dv2inst1 $\rightarrow$ Port 50000 $\rightarrow$ {\bf Next}
\item Instance properties $\rightarrow$ Authentication Type : Server $\rightarrow$ beibehalten 'Autostart the instance at system startup' $\rightarrow$ {\bf Next}
\item Prepare the DB2 tools catalog  beibehalten 'Do not prepare the DB2 tools catalog on this computer' $\rightarrow$ {\bf Next}
\item Administrator contact : Standart beibehalten $\rightarrow$ {\bf Next} \footnote{Warnung ignorieren}
\item Contact : Standart beibehalten ({\bf db2inst1}) $\rightarrow$ {\bf Next}
\item Summary $\rightarrow$ {\bf Finish}
\item Warten (dauert etwas)
\item Setup complete $\rightarrow$ {\bf Finish}
\end{itemize}
\item Test der Installation:\\
{\tt su - db2inst1} $\rightarrow$ {\tt db2stop} $\rightarrow$ {\tt db2start} \footnote{Es sollte keine Nachricht bez"uglich der Lizenz erscheinen.} $\rightarrow$ {\tt db2level} $\rightarrow$ {\tt exit}
\item Installieren Sie nun den FixPack 4 gem�� der Anleitung. Bezugsquelle f�r den Fix ist:\\
\url{http://www-3.ibm.com/cgi-bin/db2www/data/db2/udb/winos2unix/support/download.d2w/report}\\[2ex]
Die wichtigsten Schritte sind:
\begin{itemize}
\item Zuerst m��en sie die Schritte durchf�hren, die im Readme des Fixpaks beschreiben sind um die Datenbank zu beenden.
\item Auspacken des FixPack's
\item {\tt ./installFixPak -a}
\item Aktualisieren von Exemplaren zur Verwendung der neuen Stufe von DB2
\item Restart der Datenbank
\item Neubinden der DB2 UDB Datenbanken
\end{itemize}
Eine vollst�ndige Beschriebung befindet sich in der mitgelieferten Readme-Datei.

\end{enumerate}

%

\subsubsection{NSE}
\begin{enumerate}
\item Legen Sie die CD {\bf 'DB2 Net Search Extender Version 8.1'} ein und mounten Sie diese.
\item {\tt su - db2inst1} $\rightarrow$ {\tt db2stop} $\rightarrow$ {\tt exit}
\item {\tt slibclean}
\item {\tt cd /cdrom/aix} $\rightarrow$ {\tt ./nsesetup.sh}
\begin{itemize}
\item Auswahl 1 f"ur Englisch $\rightarrow$ {\bf Enter}
\item Auswahl 1 f"ur 'Accept the license areement'
\end{itemize}
\item {\tt cd /usr/opt/db2\_08\_01/instance}
\item Installieren Sie nun den FixPack 4 gem�� der Anleitung. Dieses Update ist sehr einfach zu bewerkstelligen.\\[2ex]
Die wichtigsten Schritte sind:
\begin{itemize}
\item  Auspacken des FixPack's
\item {\tt ./nsesetup.sh}
\item {\tt cd /usr/opt/db2\_08\_01/instance ; ./db2iupdt db2inst1}
\end{itemize}

\item {\tt ./db2iupdt -u db2fenc1 db2inst1}
\item Test der Installation:\\
{\tt su - db2inst1} $\rightarrow$ {\tt db2start} $\rightarrow$ {\tt db2text start} $\rightarrow$ {\tt db2licm -l}\footnote{Pr"uft Ihren Lizenzeintrag} $\rightarrow$ {\tt exit}

\end{enumerate}
%
\subsubsection{WebSphere 5}
%
{\bf Installation WebSphere 5}
\begin{enumerate}
\item Legen Sie die CD {\bf 'WebSphere Application Server 5.0 for AIX'} ein und mounten Sie diese.
\item {\tt cd /cdrom/aix} $\rightarrow$ {\tt ./install}
\begin{itemize}
\item Auswahl {\bf 'English'}
\item Auswahl {\bf Next}
\item Auswahl 'Accept the license areement'
\item Auswahl {\bf Next}\footnote{Die Warnung "uber fehlende Voraussetzunge kann ignoriert werden.}
\item Auswahl {\bf 'Costum'} $\rightarrow$ {\bf Next}
\item Schauen Sie die Optionsliste durch, wir empfehlen {\bf 'Embedded Messaging'} und {\bf 'Tivoli Performance Viewer'} aus der Auswahl zu entfernen.
\item Auswahl der Installationsverzeichnisse {\it /usr/WebSphere/AppServer} und \linebreak[4]
{\it /usr/IBMHttpServer}. Behalten Sie die Vorgaben bei!
\item Node name $\rightarrow$ Nehmen Sie Ihren Hostnamen.
\item Hostname $\rightarrow$ Nehmen Sie Ihren voll qualifizierten Hostnamen.
\item Zusammenfassung $\rightarrow$ {\bf Next}
\item Warten Sie kurz.
\item Entfernen Sie die Auswahl {\bf 'Register this product now'} $\rightarrow$ {\bf Next}
\item {\bf Finish}
\item Nach einigen Sekunden "offnet sich das {\bf 'WebSphere First Steps'} Fenster, bitte schiessen sie dieses.
\end{itemize}
\end{enumerate}
%
{\bf Installation WebSphere Fix pack 5.0.1}
\begin{enumerate}
\item Holen Sie sich das Fixpack von \url{ftp.software.ibm.com}.
\item {\tt cd <fixpack\_dir>}
\item {\tt . /usr/WebSphere/AppServer/bin/setupCmdLine.sh} Wichtig ist das Leerzeichen zwischen dem Punkt und dem Pfad.
\item {\tt ./updateWizard.sh}\footnote{bei AIX 5.2 kommt es zu einem Fehler dass die Bibliothek \mcrfile{libjvm.a} fehlt. Dann muss 
man sich einen aktuellen installer von \url{http://www-1.ibm.com/support/docview.wss?rs=180&tc=SSEQTP&uid=swg24001908} herunterladen und entpacken. Das Verzeichnis  
\mcrpath{fixpacks} des Fixpaks in das Verzeichnis des Installers kopiert werden.  Dann den \mcrcommand{updateWizard.sh} starten.
\begin{itemize}
\item Update Installation Wizard $\rightarrow$ {\tt Next}
\item Auswahl {\bf 'IBM WebSphere Application Server 5.0.0'} $\rightarrow$ {\tt Next}
\item Auswahl {\bf 'Install Fix packs'} $\rightarrow$ {\tt Next}
\item Fixpack Verzeichnis beibehalten $\rightarrow$ {\tt Next}
\item Auswahl {\bf 'was50\/fp1\_aix'} $\rightarrow$ {\tt Next}
\item Auswahl {\bf 'Update the IBM HTTP Server'} $\rightarrow$ {\tt Next}
\item Zusammenfassung $\rightarrow$ {\tt Next}
\item Wenn Installation erfolgreich war $\rightarrow$ {\tt Finish}
\end{itemize}
\end{enumerate}
%
{\bf Test der WebSphere Installation}
\begin{enumerate}
\item Starten Sie den WebSphere Administrations-Server
\begin{itemize}
\item {\tt cd /usr/WebSphere/AppServer/bin}
\item {\tt ./startServer.sh server1}\footnote{Warten Sie auf die Nachricht {\bf 'Server 1 open for e-business'}}
\end{itemize}
\item Im Browser {\tt http://<hostname>:9090/admin}
\begin{itemize}
\item Eingabe {\bf icmadmin} als UserID $\rightarrow$ {\tt OK}
\item {\bf 'Environment'}
\item Auswahl {\bf 'Virtual hosts`}
\item Auswahl {\bf 'default\_hosts`}
\item Auswahl {\bf 'Host Aliases'}
\item Auswahl {\bf 'New'} $\rightarrow$ Hostname : * $\rightarrow$ Port : 443 $\rightarrow$ {\tt Apply}
\item {\tt Save}\footnote{Den Text finden Sie im Bild oben.}
\item {\tt Save}
\item Im linken Baum {\bf 'Update Web Server Plugin'} ausw"ahlen
\item {\tt OK}
\item {\tt Logout}
\end{itemize}
\item Starten Sie den Web-Server
\begin{itemize}
\item {\tt cd /usr/IBMHttpServer/bin}
\item {\tt ./apachectl start}
\end{itemize}
\item Im Browser {\tt http://<hostname>/snoop}
\end{enumerate}
%
{\bf Configure the SSL}
\begin{enumerate}
\item \mcrcommand{/usr/IBMHttpServer/bin/ikeyman}
\begin{itemize}
\item Select the key database {\bf 'New'}
\item Select {\bf CMS key database file} $\rightarrow$ File Name = \mcrfile{key.kdb} $\rightarrow$ Location = \mcrfile{/usr/IBMHttpServer/ssl}
\item Select {\bf 'OK'}
\item \mcrcommand{<password>} $\rightarrow$ \mcrcommand{<password>}
\item 'Stach the password to a file?' $\rightarrow$ {\bf 'OK'}
\item Select {\bf 'OK'}
\item Select {\bf 'Create'}
\item Select {\bf 'New self signed certificate'} $\rightarrow$ Label = \mcrfile{icmrm} $\rightarrow$ Version = X509V3 $\rightarrow$ Key Size = 1024 $\rightarrow$ Name = \mcrcommand{<full qualified host name>} $\rightarrow$ Organization = \mcrcommand{<your org>} $\rightarrow$ Country = \mcrcommand{DE} $\rightarrow$ Valid Period = \mcrcommand{1000}
\item Select {\bf 'OK'}
\item Select {\bf 'Key database'}
\item Select {\bf 'Exit'}
\end{itemize}
\end{enumerate}
If some environments there have been problems running ikeyman because there might appear a message: You need to register IBMJCE provider. In case you get this message - {\bf and ony then} - here is a list of ways to fix it:
\begin{itemize}
\item Change to the JDK provided by WebSphere (cahnge PATH variable) {\bf OR}
\item Remove the file \mcrfile{gskikm.jar} from the directory \mcrfile{/usr/java131/jre/lib/ext} (move to a different place or rename)
\end{itemize}
%
{\bf Activate the HTTPD for SSL}
\begin{enumerate}
\item \mcrcommand{cd /usr/IBMHttpServer}
\item \mcrcommand{./bin/setupadm}
\begin{itemize}
\item User ID = {\bf httpadm} $\rightarrow$ Group Name = {\bf httpadm} $\rightarrow$ \mcrcommand{<enter>} $\rightarrow$ {\bf 1} $\rightarrow$ {\bf 1} $\rightarrow$ {\bf 2}
\end{itemize}
\item \mcrcommand{./bin/htpasswd -c conf/admin.passwd admin}
\begin{itemize}
\item \mcrcommand{<password>} $\rightarrow$ \mcrcommand{<password>}
\end{itemize}
\item \mcrcommand{/bin/adminctl start}
\end{enumerate}
In a web-browser open the URL {\bf http://<yourhost>:8008/} and login as user {\bf admin} with \mcrcommand{<password>}.
In the web browser please do the following configurations.
\begin{enumerate}
\item Set up the security module
\begin{itemize}
\item In the left navigation panel select {\bf Basic Select} $\rightarrow$ {\bf Module Sequence}
\item In the Module Sequence panel $\rightarrow$ Scope = {\bf GLOBAL} $\rightarrow$
\item {\bf Add} $\rightarrow$ select a module to add $\rightarrow$ in the drop-down list select {\bf ibm\_ssl} $\rightarrow$ \mcrfile{modules/IBMModuleSSL128.dll} wird ersetzt $\rightarrow$ {\bf 'Apply'} $\rightarrow$ {\bf 'Close'} $\rightarrow$ {\bf 'Submit'}
\end{itemize}
\item Set up the secure host IP and aditional prot for the secure server
\begin{itemize}
\item In the left navigation panel select {\bf Basic Select} $\rightarrow$ {\bf Advanced Properties}
\item In the Module Sequence window panel $\rightarrow$ Scope = {\bf GLOBAL} $\rightarrow$
\item {\bf Add} (for the Specify aditional ports and IP address filed) $\rightarrow$ IP addresse = {\bf empty} $\rightarrow$ Port = {\bf 80 $\rightarrow$} {\bf 'Apply'}
\item {\bf Add} (for the Specify additional prots and IP adress field) $\rightarrow$ IP addresse = {\bf empty} $\rightarrow$ Port = {\bf 443 $\rightarrow$} {\bf 'Apply'} $\rightarrow$ {\bf 'Close'} $\rightarrow$ {\bf 'Submit'}
\end{itemize}
\item Set up the virtual host sructure for the secure server
\begin{itemize}
\item In the left navigation panel select {\bf Configuration Structure} $\rightarrow$ {\bf Create Scope}
\item In the Create Scope panel $\rightarrow$ in the drop-down list under Select a valid scope select Virtual Host 'Enter the virtual host IP address or full qualified domain name' = /mcrcommand{<your.full.qualified.hostname>} $\rightarrow$ Virtual Host Port = {\bf 443} $\rightarrow$ Server name = {\bf empty} $\rightarrow$ Alternate name = {\bf empty} $\rightarrow$ {\bf 'Submit'}
\end{itemize}
\item Set up the virtual host document root for the secure server
\begin{itemize}
\item In the left navigation panel select {\bf Basic Select} $\rightarrow$ {\bf Core Settings}
\item In the Core Settings panel $\rightarrow$ select {\bf Scope} $\rightarrow$ select the \mcrcommand{<Virtual host that you created for SSL>} $\rightarrow$ Server name = {\bf empty} $\rightarrow$ Document root directory = \mcrfile{/usr/IBMHttpServer/htdocs/en\_US} $\rightarrow$ {\bf 'Submit'}
\end{itemize}
\item Set the file and SSL timeout values for the secure server
\begin{itemize}
\item In the left navigation panel select {\bf Security} $\rightarrow$ {\bf Server Security}
\item In the Security panel $\rightarrow$ Ensure scope \mcrcommand{<Virtual host that you created for SSL>} $\rightarrow$ {\bf 'Enable SSL'} $\rightarrow$ select {\bf 'Yes'} $\rightarrow$ Key file name = \mcrfile{/usr/IBMHttpServer/ssl/key.kdb} $\rightarrow$ Timeout SSL 2 = {\bf 100} $\rightarrow$ Timeout SSL 3 = {\bf 1000} $\rightarrow$ {\bf 'Submit'}
\end{itemize}
\item Enable SSL and select the mode of the client authentication
\begin{itemize}
\item In the left navigation panel select {\bf Security} $\rightarrow$ {\bf Host Authorization}
\item in the Authorization panel $\rightarrow$ Ensure scope \mcrcommand{<Virtual host that you created for SSL>} $\rightarrow$ {\bf 'Enable SSL'} $\rightarrow$ select {\bf 'Yes'} $\rightarrow$ select {\bf 'Mode of client authorization to be used'} $\rightarrow$ select {\bf 'None'} $\rightarrow$ 'Server cretificate to be used for this virtual hst field' = {\bf icmrm} $\rightarrow$ Add (for the chipher specification that can be used in a secure transaction panel) $\rightarrow$ select {\bf 39} $\rightarrow$ {\bf 'Apply'} select {\bf 3A} $\rightarrow$ {\bf 'Apply'} select {\bf 62} $\rightarrow$ {\bf 'Apply'} select {\bf 64} $\rightarrow$ {\bf 'Apply'} $\rightarrow$ {\bf 'Submit'}
\end{itemize}
\end{enumerate}
Restart the HTTP Server (and leave it open) by clicking on the black circle logo that is located next to the help button in the upper-right corner of the right panel. \\[2ex]
Open \mcrcommand{https://<your hostname>/snoop} in the web browser.\\[2ex]
%
{\bf Installation of the FixPack 2}
\begin{enumerate}
\item Download from \url{ftp://ftp.software.ibm.com/software/websphere/appserv/support/fixpacks/was50/fixpack2/AIX/}
\item Unpack the tar file to a \mcrfile{install\_root}
\item READ the documentation!
\item Set the JAVA\_HOME environment variable.
\item Stop all WebSphere applications with \mcrcommand{/usr/WebSphere/AppServer/bin/stopServer ...}\footnote{like icmrm or mycoresample}
\item Start the install wizzard with \mcrcommand{install\_root/updateWizard.sh} or as command line tool with \mcrcommand{install\_root/updateSilent.sh}
\begin{itemize}
\item Select {\bf english}
\item By default let the path of WebSphere on \mcrfile{/usr/WebSphere/AppServer}.  $\rightarrow$ {\bf 'Next'}
\item Select {\bf fixpacks} $\rightarrow$ {\bf 'Next'}
\item Let the path of install source by \mcrfile{install\_root}. $\rightarrow$ {\bf 'Next'}
\item Confirm that you will install the fixpack 2. $\rightarrow$ {\bf 'Next'}
\item By default let the path of HTTPD on \mcrfile{/usr/IBMHttpServer}. $\rightarrow$ {\bf 'Next'}
\item {\bf 'Next'}
\item {\bf 'Finish'}
\end{itemize}
\end{enumerate}
%
\subsubsection{Content Manager}
\begin{enumerate}
\item Legen Sie bitte die nachfolgenden Benutzer an. {\bf icmadmin} und {\bf rmadmin} ben�tigen nicht viel Platz. F�r den Benutzer {\bf mcradmin} sollte ausreichend Platz eingeplant werden, da hier sp�ter die MyCoRe-Anwendungen installiert werden.. Achten Sie darauf, dass man sich in die Accounts einloggen kann, z. B. via Telnet! Im IBM Handbuch wir ein Accout {\bf icmconct} angegeben, dies entspricht in der Funktionalit�t unserem {\bf mcradmin}. Sicherheitshalber k�nnen sie es mit anlegen.
\begin{itemize}
\item User {\bf icmadmin} $\rightarrow$ Group {\bf db2grp1}
\item User {\bf rmadmin} $\rightarrow$ Group {\bf db2grp1}
\item User {\bf icmconct} $\rightarrow$ Group {\bf staff}
\item User {\bf mcradmin} $\rightarrow$ Group {\bf mcr}
\end{itemize}
\item F�gen sie den Nutzer {\bf root} der Gruppe {\bf db2grp1} hinzu.
\item Erg�nzen Sie das File \mcrfile{/etc/environment} um folgende Zeilen:
\begin{verbatim}
#
# Appendix for Content Manager
#
ICMROOT=/usr/lpp/icm
ICMDLL=/home/db2fenc1
ICMCOMP=/usr/vacpp/bin
CMCOMMON=/usr/lpp/cmb/cmgmt
EXTSHM=ON
DB2INSTANCE=db2inst1
DB2LIBPATH=/usr/lpp/icm/lib
\end{verbatim}
\item Erg�nzen Sie das File \mcrfile{.profile} f�r die User {\bf root}, {\bf icmadmin}, {\bf rmadmin} und {\bf mcradmin} mit den folgenden Zeilen:
\begin{verbatim}
# The following three lines have been added by UDB DB2.
if [ -f /home/db2inst1/sqllib/db2profile ]; then
    . /home/db2inst1/sqllib/db2profile
fi
\end{verbatim}
\item Das File \mcrfile{/home/db2inst1/sqllib/profile.env} sollte folgende Eintr�ge aufweisen:
\begin{verbatim}
DB2_FMP_COMM_HEAPSZ='12000'
DB2ENVLIST='LIBPATH ICMROOT ICMDLL ICMCOMP EXTSHM CMCOMMON DB2LIBPATH'
DB2_RR_TO_RS='YES'
DB2COMM='tcpip'
DB2AUTOSTART='YES'
\end{verbatim}
\item Nun sollte noch einmal die DB2 und NSE gestoppt und neu gestartet werden.\footnote{Es ist sinnvoll nach jedem Reboot des Systems vor dem Start des CM diese Schritte durchzuf�hren um einen sicheren Ausgangspunkt zu haben!}
\begin{itemize}
\item \mcrcommand{su - db2inst1}
\item \mcrcommand{db2stop}
\item \mcrcommand{db2text stop}
\item \mcrcommand{db2start}
\item {\bf Kurz warten!!!}
\item \mcrcommand{db2db2text start}
\end{itemize} 
\item Dieser Punkt ist nur erforderlich, wenn ein TSM System eingesetzt werden soll.
\begin{itemize}
\item \mcrcommand{su - root}
\item \mcrcommand{mkdir </home/rmadmin/staging>}
\item \mcrcommand{mklv -y <lvstaging> <vg..> 32}
\item \mcrcommand{crfs -vjfs -d <lvstaging> -m </home/rmadmin/staging> -A yes}
\item \mcrcommand{mount </home/rmadmin/staging>}
\item \mcrcommand{chown rmadmin.db2grp1 </home/rmadmin/staging>}
\end{itemize}
\item Now you must create a logical volume for the store of the Resource Manager.
\begin{itemize}
\item \mcrcommand{su - root}
\item \mcrcommand{mkdir </home/rmadmin/storage>}
\item \mcrcommand{mklv -y <lvstorage> <vg..> 32}
\item \mcrcommand{crfs -vjfs -d <lvstorage> -m </home/rmadmin/storage> -A yes}
\item \mcrcommand{mount </home/rmadmin/storage>}
\item \mcrcommand{chown rmadmin.db2grp1 </home/rmadmin/storage>}
\end{itemize}
\item  Now you can start the Content Manager installation. Attention, this will open a X11 window connection!
\begin{enumerate}
\item \mcrcommand{su - root}
\item \mcrcommand{cd .../English}
\item \mcrcommand{./setup.exe}
\item {\bf 'Next'}
\item Select I accept ... {\bf 'Next'}
\item Setup type $\rightarrow$ {\bf Select full} $\rightarrow$ {\bf 'Next'}
\item Identification and authorization for LS
\begin{itemize}
\item Library Server database name $\rightarrow$ {\bf ICMNLSDB}
\item Library Server scheme name $\rightarrow$ {\bf ICMADMIN}
\item Library Server database administration ID $\rightarrow$ {\bf icmadmin}
\item Password ... $\rightarrow$ Confirm password ...
\item Database connection ID $\rightarrow$ {\bf mcradmin}
\item Library Server ID $\rightarrow$ {\bf 1}
\item Unselect {\bf Enable Unicode}
\item Select {\bf Enable text search}
\end{itemize}
\item Configure Resource Manager
\begin{itemize}
\item Resource Manager database name $\rightarrow$ {\bf RMDB}
\item Resource Manager database administration ID $\rightarrow$ {\bf rmadmin}
\item Password ... $\rightarrow$ Confirm password ...
\end{itemize}
\item Installation options for resource manager database
\begin{itemize}
\item Mount point $\rightarrow$ \mcrfile{</home/rmadmin/storage>}
\item Path $\rightarrow$ \mcrfile{</home/rmadmin/staging>}
\end{itemize}
\item Resource Manager with WebSphere Application Server
\begin{itemize}
\item WebSphere home $\rightarrow$ \mcrfile{/usr/WebSphere/AppServer}
\item Web application path $\rightarrow$ {\bf /icmrm}
\item Web application name $\rightarrow$ {\bf icmrm}
\item Service Port $\rightarrow$ {\bf 7500}
\item Application server name $\rightarrow$ {\bf icmrm}
\item Your WebSphere Application Server will be stopped$ \rightarrow$ {\bf 'Yes'}
\end{itemize}
\item WebSphere V5 auto deploy options
\begin{itemize}
\item WebSphere administration user ID $\rightarrow$ {\bf icmadmin}
\item Password ... $\rightarrow$ Confirm password ...
\item Node name $\rightarrow$ {\bf <your host name>}
\end{itemize}
\item Connect Library Server to Resource Manager
\begin{itemize}
\item Resource Manager server hostname $\rightarrow$ {\bf <your full qualified hostname>}
\item Web application port $\rightarrow$ {\bf 80}
\item Secure web application port $\rightarrow$ {\bf 443}
\item Resource Manager server operating system $\rightarrow$ {\bf AIX}
\item Token duration $\rightarrow$ {\bf 48}
\end{itemize}
\item Configure components for LDAP
\begin{itemize}
\item Do NOT select Library Server
\item Do NOT select Resource Manager
\end{itemize}
\item Summary $\rightarrow$ {\bf 'Next'}
\item {\bf 'Finish'}
\end{enumerate}
\item Verify the log file \mcrfile{/usr/lpp/icm/logs/icm82install.log}.
\item Start the Resource Manager as root \mcrcommand{/usr/WebSphere/AppServer/bin/startServer.sh icmrm}.
\item Test the Resource Manager in a web browser $\rightarrow$ \mcrcommand{https://<your full qualified hostname>/icmrm/ICMResourceManager}\footnote{The response is : No order found to process, that is okay.}
\end{enumerate}
%
{\bf Installation of the FixPack 2}
\begin{enumerate}
\item Download from \url{ftp://ftp.software.ibm.com/ps/products/content_manager/fixes/v8.2/aix/820.20/}
\item READ the documentation!
\item Do the steps under 2.2.1 of the documentation.
\end{enumerate}
%
\subsubsection{Information Integrator for Content Manager}
\begin{enumerate}
\item \mcrcommand{su -}
\item Change to the install directory.
\item \mcrcommand{./frnxsetup.sh} Attention, this will open a X11 window connection!
\item License Agreement $\rightarrow$ {\bf 'Accept'}
\item {\bf 'Next'}
\item Install the following options
\begin{itemize}
\item Remote Connectors $\rightarrow$ {\bf none}
\item Local Connectors $\rightarrow$ {\bf only CM V8 connector}
\item Connector Toolkitd and Samples $\rightarrow$ {\bf only CM V8 connector}
\item Features $\rightarrow$ {\bf none}
\item Infocenter $\rightarrow$ {\bf optional}
\item System Admin Database $\rightarrow$ {\bf none}
\item {\bf 'Next'}
\end{itemize}
\item System configuration
\begin{itemize}
\item Keep {\bf Local}
\item Do not select LDAP
\item {\bf 'Next'}
\end{itemize}
\item Content Manager V8 Server Connection
\begin{itemize}
\item Database name $\rightarrow$ {\bf icmnlsdb}
\item Schema name $\rightarrow$ {\bf ICMADMIN}
\item Authentication type $\rightarrow$ {\bf Server}
\item Database connection ID $\rightarrow$ {\bf mcradmin}
\item Password ... $\rightarrow$ Confirm password ...
\item Enable single sign-on $\rightarrow$ {\bf false}
\item {\bf 'Next'}
\end{itemize}
\item Content Manager V8 Connector $\rightarrow$ {\bf 'Next'}
\item WAIT!
\item {\bf 'Finish'}
\item Check the log under \mcrfile{/tmp/frn/frnxinst.log}
\end{enumerate}
%
{\bf Installation of the FixPack 2}
\begin{enumerate}
\item Download from \url{ftp://ftp.software.ibm.com/ps/products/enterprise_information_portal/fixes/v8.2/aix/820.20}
\item READ the documentation!
\item Do the steps under 2.2.1 of the documentation.
\end{enumerate}
%
%


%
%
%
\subsection{Der IBM Content Manager unter Windows}
%
\subsubsection{Content Manager}
%
{\bf Installation of the FixPack 2}
\begin{enumerate}
\item Download from \url{ftp://ftp.software.ibm.com/ps/products/content_manager/fixes/v8.2/win/820.20}
\item READ the documentation!
\item Do the steps under 2.1.1 of the documentation.
\end{enumerate}

%
\subsubsection{Information Integrator for Content Manager}
%
{\bf Installation of the FixPack 2}
\begin{enumerate}
\item Download from \url{ftp://ftp.software.ibm.com/ps/products/enterprise_information_portal/fixes/v8.2/win/820.20}
\item READ the documentation!
\item Do the steps under 2.1.1 of the documentation.
\end{enumerate}

%
\subsubsection{Content Manager Windows Client}
%
{\bf Installation of the FixPack 2}
\begin{enumerate}
\item Download from \url{ftp://ftp.software.ibm.com/ps/products/content_manager/WinClient/fixes/v8.2/820.20}
\item READ the documentation!
\item Do the steps under 2.1.1 of the documentation.
\end{enumerate}


%
%
%
%
\subsection{Der IBM Content Manager unter Windows}


%
%

\section{Hinweise zur Installation freier und komerzieller Datenbanken, XML:DB's und Web--Komponenten}
%
%
\subsection{Installation der Komponenten unter Linux}
%
%
Die folgende Beschreibung erl�utert die Arbeit unter SuSE Linux, getestet wurde unter der Version 8.2. Es ist besonders aus Sicherheitsgr�nden besonders zu empfehlen, auch die vom Distributor angebotenen Updates f�r die nachfolgenden Komponenten zu installieren. Sollten f�r RedHat Abweichungen auftreten, so werder diese gesondert vermerkt.


\subsubsection{Vorbereitung}

Als erster Schritt sind der Java-Compiler J2SE 1.3.1 oder h�her und ANT zu installieren. Beide befinden sich unter in der SuSE in der Distribution. 
\begin{itemize}
\item {\bf java2-jre-1.3.1-521} oder h�her
\item {\bf java2-1.3.1-521} oder h�her
\item {\bf jakarta-ant-1.5-76} oder h�her
\end{itemize}

Wenn allerdings SuSE von einem ftp Server installiert wird, so ist Java nicht standardm��ig enthalten. Dann bitte die Java Linux Version von der \url{www.blackdown.org}
\footnote{\url{ftp://ftp.informatik.hu-berlin.de/pub/Java/Linux/JDK-1.4.1/i386/01/j2sdk-1.4.1-01-linux-i586-gcc3.2.bin}}
Homepage runterladen und wie folgt installieren. Download nach beispielsweise
nach {\it /usr/local} anschlie�end mit {\tt chmod +x
  j2sdk-1.4.1-01-linux-<ARCH>.bin} ausf�hrbar machen und ausf�hren {\tt
  ./j2sdk-1.4.1-01-linux-<ARCH>.bin}.  Damit die Pfad Variablen alle richtig
gesetzt werden eine Vorlage unter {\it/etc/java} entsprechend editieren und
mit {\tt setDefaultJava} als root die Links richtig setzten. Zudem sollte man
in der Datei {\it /etc/profile.local}\footnote{Falls diese noch nicht
  existiert einfach anlegen} noch die Variable {\it ANT\_HOME}\footnote{export
  ANT\_HOME=/opt/jakarta/ant} gesetzt werden. 

Sinnvollerweise sollten Sie sich einen Benutzer (z. B. {\bf mcradmin}) anlegen, unter welchem Sie dann Ihre Applikationen laufen lassen.

\subsubsection{Die Installation von MySQL}

\label{mysqlinstallation}
MySQL ist eine derzeit frei relationale Datenbank, welche zur schnellen Speicherung von Daten innerhalb des MyCoRe-Projektes ben�tigt wird. Sie besitzt eine JDBC Schnittstelle und ist SQL konform. Sie k�nnten MySQL auch durch eine andere verf�gbare Datenbank mit gleicher Funktionalit�t ersetzen.
\begin{enumerate}
\item Installieren Sie aus Ihrer Distribution die folgenden Pakete und danach ggf. noch vom Hersteller der Distribution per Netz angebotene Updates. Die angegebenen versionsnummern sind nur exemplarisch.
\begin{itemize}
\item {\bf mysql-3.23.52-83} oder h�her
\item {\bf mysql-shared-3.23.52-83} oder h�her
\item {\bf mysql-client-3.23.52-83} oder h�her
\item {\bf mysql-devel-3.23.52-83} oder h�her
\item {\bf mysql-bench-3.23.52-83} oder h�her
\end{itemize}
\item Die Dokumentation steht nun unter {\it /usr/share/doc/packages/mysql}.
\item F�hren Sie das Kommando {\tt rcmysql start} als root aus. 
\item F�hren Sie das Kommando {\tt /usr/bin/mysqladmin -u root password {\it new-password}} als root aus.
\item Die folgende Sequenz sorgt daf�r, dass der MyCoRe-User {\bf mcradmin} alle Rechte auf der Datenbank hat. 
      Dabei werden bei der Ausf�hrung von Kommandos von {\bf localhost} aus keine Passw"orter abgefragt. 
      Von anderen Hosts aus muss {\it ein-password} eingegeben werden.

{\tt
mysql --user=root -p{\it PASSWORD} mysql \newline
GRANT ALL PRIVILEGES ON *.* TO mcradmin@localhost WITH GRANT OPTION; \newline
GRANT ALL PRIVILEGES ON *.* TO mcradmin@'\%' IDENTIFIED BY '{\it ein-passwort}' \newline
WITH GRANT OPTION; \newline
quit \newline
}

\item Ist das Password einmal gesetzt, m�ssen Sie zus�tzlich die Option -p verwenden.
\item Zum Verifizieren, ob der Server l�uft nutzen Sie {\tt mysqladmin version} und  {\tt mysqladmin variables}.
\item jetzt k�nnen Sie die Datenbasis f�r MyCoRe mit nachstehendem Kommando anlegen.
{\tt
\begin{verbatim}
mysqladmin -u mcradmin create mycore 
\end{verbatim}
}
\item Falls weitere Benutzer noch das Recht auf Selects von allen Hosts aus haben sollen, verwenden Sie die Kommandos

{\tt
mysql --user=mcradmin -p{\it PASSWORD} mycore \newline
GRANT SELECT ON mycore.* TO mycorenutzer@'\%'; \newline
quit
}

\end{enumerate}

Falls sie keine Connection auf ihren Rechnernamen (nicht localhost) aufbauen k�nnen, kann es auch mit ihrer Firewall oder TCPWrapper Einstellung zu tun haben. Bei einer Firewall sollte der Port 3306 freigegeben werden und bei einem TCPWrapper der entsprechende Dienst (mysql) in die Datei {\it /etc/hosts.allow} geschrieben werden.

\subsubsection{Die Installation der freien XML:DB eXist}

eXist ist eine Frei verf�gbare XML:DB, welche die entsprechenden Interfaces implementiert. 
F�r MyCoRe wurde ein auf diesen Schnittstellen basierender Persitence-Layer implementiert. 
So ist die Nutzung von eXist direkt m�glich.
\paragraph*{Vorbedingungen} 
F�r das folgende Szenario ist darauf zu achten, dass der Tomcat bzw Websphere Server nicht auf den Port 8080 h�rt, 
da es ansonsten zu einem Konflikt mit der hier vorgestellten Installation kommt, 
weil der eXist servlet Container ebenfalls standardm��ig auf den port 8080 h�rt. 
Den Port f�r Tomact �ndern sie in der {\it tomcatinstalldir/conf/server.xml}. 
Ebenfalls wird die in Kapitel \ref{mysqlinstallation} beschriebene MySQL-Installation vorausgesetzt. 
Desweiteren sollte die aktuellste Version von MyCoRe und dem MyCoRe Sample auf ihrem Rechner 
in \$MYCORE\_HOME bzw. \$MYCORE\_SAMPLE\_HOME liegen. 

\begin{enumerate}
\item Download der aktuellen Version von eXist von \url{http://exist-db.org/}. Achtung, da dieses Produkt noch stark im Wachsen ist, sollte hier die letzte Version geholt werden (cvs). Zum Test kam Version eXist-0.9.2.
\item entpacken Sie die Distribution in ein entsprechendes Verzeichnis, z. B. unter {\it /home/mcradmin} (eXist-installdir).
\item Entfernen Sie zur Nutzung des stand-alone-Servers den Kommentar aus der Zeile {\tt uri=xmldb:exist://localhost:8081} im File {\it client.properties} und achten sie darauf, dass die Zeile {\tt uri=xmldb:exist://localhost:8080/exist/xmlrpc} auskommentiert wird ( vorwegsetzten). Dies Einstellung wird zum Beispiel f�r die direkte Einbindung in Tomcat verwendet, wenn man �ber xmlrpc auf eXist zugreifen m�chte.
\item Starten Sie den Server mit {\tt <eXist-installdir>/bin/startup.sh }.
\item unter der URL \url{http://localhost:8080/exist/index.xml} sollte jetzt die eXist-Homepage erscheinen. Dies nur als Test.
\item Wenn alles okay ist, starten Sie {\tt <eXist-installdir>/bin/server.sh }.
\item Anschliessend k�nnen Sie auch den Client mit {\tt <eXist-installdir>/bin/client.sh } starten. Hier sollten Sie dem Admin-User ein Password spendieren.
\end{enumerate}

\subsubsection{Die Installation von Apache 2}
\label{sec:apchache2_installation}

Im Linux-Umfeld kann der Apache-Webserver als Standard betrachtet werden. Er ist in allen g�ngigen Distributionen enthalten. Die allerneuesten Apache 2 Pakete sind zu finden unter \url{ftp://ftp.suse.com/pub/projects/apache/apache2/8.2-i386}. Installiern Sie folgende Komponenten:
\begin{itemize}
\item {\bf apache2-2.0.46-3} oder h�her
\item {\bf apache2-doc-2.0.46-3} oder h�her, dieses Paket ist optional
\item {\bf apache2-worker-2.0.46-3} oder h�her
\end{itemize}
Mit \mcrcommand{rcapache2 start} wird der Server per default Einstellungen gestartet und sollte jetzt �ber \url{http://localhost} erreichbar sein. Etwaige Fehlermeldunge werden in \mcrfile{/var/log/apache2/error\_log} geschrieben.\\[2ex]
Die zentrale Konfigurationsdatei des Apache2 ist die unter \mcrpath{/etc/apache2} liegende Datei \mcrfile{httpd.conf}. Bevor irgendwelche �nderungen vorgenommen werden bitte immer ein Sicherheitskopie anlegen. Der Befehl (als root) \mcrcommand{/usr/sbin/apache2ctl configtest} �berpr�ft die Syntax ihrer ge�nderten Einstellungen.
Vorerst sollten jedoch die Standard Einstellungen gen�gen.
Mit \mcrcommand{rcapache2 [start; stop; restart]} als root kann der Server gestartet gestoppt bzw neugestartet werden und �ber \mcrcommand{rcapache2 [status]} wird der aktuelle Staus abgefragt.

\subsubsection{Die Installation von Tomcat 4}

Wie Apache, dessen Installation im vorigen Abschnitt \ref{sec:apchache2_installation} beschrieben wurde,  
ist auch Tomcat als Servlet-Engine im Linux-Umfeld ein Quasi-Standard. 
F�r die Arbeit der Web-Anwendung des MyCoRe-Projektes wird dieses Tool zwingend ben�tigt. 
Installiern Sie folgende Komponenten:
\begin{itemize}
\item {\bf jakarta-tomcat-4.1.18-31} oder h�her
\item {\bf apache2-jakarta-tomcat-connectors-4.1.18-31} oder h�her
\end{itemize}
Im File \mcrfile{/etc/profile.local} ist folgender Eintrag vorzunehmen, 
damit Tomcat auch von {\bf mcradmin} genutzt werden kann. Weiterhin werden der Memory Size f�r den Server erweitert.
\begin{verbatim}
export CATALINA_HOME=/opt/jakarta/tomcat
export CATALINA_OPTS="$CATALINA_OPTS -server -Xms256m -Xmx1800m -Xincgc"
\end{verbatim}
Weiterhin ist noch die richtige Gruppenzugeh�rigkeit f�r die Tomcat-Installation festzulegen.
\begin{verbatim}
chown wwwrun:nogroup -R /opt/jakarta/tomcat
\end{verbatim}
Mit \mcrcommand{rctomcat [start; stop; restart]} als root kann der Server gestartet gestoppt bzw 
neugestartet werden und �ber \mcrcommand{rctomcat [status]} wird der aktuelle Staus abgefragt. 
Nach erfolgreichem starten des Tomcat Servers sollte er unter der Standardadresse \url{http://localhost:8080/examples} 
erreichbar sein. 
Bitte beachten Sie aber, dass der standardm"assige Port 8080 auch von eXist (falls Sie dies verwenden)
verwendet wird.
In diesem Fall m"ussen Sie in der Konfigurationsdatei {\it server.xml} einen anderen Port ausw"ahlen.
Die Dokumentation steht unter \url{http://localhost:8080/tomcat-docs}.
%
%
\subsection{Installation der Komponenten unter MS Windows}
%
%
\subsubsection{Die Installation der freien XML:DB eXist}
eXist ist eine frei verf�gbare XML:DB, welche die entsprechenden Interfaces implementiert. F�r MyCoRe wurde ein auf diesen Schnittstellen basierender Persistence-Layer implementiert. So sollte die Nutzung von eXist direkt m�glich sein.
\begin{enumerate}
\item Download der aktuellen Version von eXist von \url{http://exist-db.org/}. Achtung, da dieses Produkt noch stark im Wachsen ist, sollte hier die letzte Version geholt werden. Zum Test kam Version eXist-0.9.2.
\item entpacken Sie die Distribution in ein entsprechendes Verzeichnis.
\item Entfernen Sie zur Nutzung des stand-alone-Servers den Kommentar aus der Zeile {\tt uri=xmldb:exist://localhost:8081} im File {\it <installdir>\\eXist-0.9.2\\client.properties}.
\item Starten Sie {\tt <installdir>\\eXist-0.9.2\\bin\\server.bat }.
\item Anschlie�end k�nnen Sie auch den Client mit {\tt <installdir>\\eXist-0.9.2\\bin\\client.bat } starten. Hier sollten Sie dem Admin-User ein Password spendieren.
\end{enumerate}
%
%

%
%

%%% Local Variables: 
%%% mode: latex
%%% TeX-master: "UserGuide"
%%% End: 

%
%
%
%
\section{Weitere erforderliche oder optionale Software}
%
%
\subsection{Software f�r Unix-Systeme}
\begin{itemize}
\item {\bf Bash} (erforderlich) - Alle zus�tzlichen Unix-Shell-Scripts basieren auf der Bash-Shell. Es ist also sehr sinvoll, diese auf dem System mit zu installieren, sofern diese noch nicht vorhanden ist (z. B. auf AIX).
\item {\bf CVS-Client}(erforderlich) - Dieser dient dem Check out der aktuellen MyCoRe Distribution und ist unter Linux bereits in der Distribution. F�r andere Systeme muss er seperat geholt werden.
\item {\bf ANT}(erforderlich) - Unter Linux ist dies unterdessen in der Distribution enthalten, f�r andere Systeme muss es von Server des Apache-Projektes geholt werden.
\item {\bf ImageMagick} (optional) - Ein m�chtiges Bildkonverter-Programm, welches bei der Bearbeitung gro�er Bildmengen sehr hilfreich ist.
\end{itemize}
Quellen sind unter anderem
\begin{itemize}
\item f�r einige Produkte unter AIX $\rightarrow$ \url{http://ftp.software.ibm.com/}
\item f�r {\bf CVS} $\rightarrow$ \url{http://cvshome.org/}
\item f�r {\bf ANT} $\rightarrow$ \url{http://ant.apache.org/}
\end{itemize}
%
%
\subsection{Software f�r Microsoft Systeme}

%
%


%
%


