\chapter{Download und Installation des MyCoRe Kerns}
%
\section{Download des MyCoRe Kerns}
Das MyCoRe Projekt wird f"ur alle unterst"utzten Systeme "uber das CVS Repository
ausgeliefert. Das Holen der aktuellen Version erfolgt mit dem Kommando 
\begin{center}
{\bf cvs -d :pserver:anoncvs@server.mycore.de:/cvs checkout mycore} 
\end{center}
Nach dem checkout finden Sie folgende Dateistruktur vor:
\begin{center}
\begin{tabular}{ll}
{\bf mycore} &  Das Root-Verzeichnis des MyCoRe-Kerns \\
\quad {\bf bin} & Das Verzeichnis der Shellscripte \\
\qquad build.sh & Shellscript zum Compilieren unter einem UNIX-System \\
\qquad build.cmd & Shellscript zum Compilieren unter einem UNIX-System \\
\qquad mycore.sh & Shellscript zur Arbeit mit dem Commandline-Interface unter UNIX \\
\qquad mycore.cmd & Shellscript zur Arbeit mit dem Commandline-Interface unter Windows \\
\qquad setup\_cm7.sh & Shellscript f"ur eine UNIX-Umgebung mit IBM Content Manager 7 \\
\qquad setup\_cm7.cmd & Kommandoscript f"ur eine Windows-Umgebung mit IBM Content Manager 7 \\
\qquad setup\_cm8.sh & Shellscript f"ur eine UNIX-Umgebung mit IBM Content Manager 8.1/8.2 \\
\qquad setup\_cm8.cmd & Kommandoscript f"ur eine Windows-Umgebung mit IBM Content Manager 8.1/8.2 \\
\qquad setup\_xindice.sh & Shellscript f"ur eine UNIX-Umgebung mit Xindice \\
\qquad setup\_xindice.cmd & Kommandoscript f"ur eine Windows-Umgebung mit Xindice \\
\quad {\bf documentation} & Dokumentationen zu MyCoRe \\
\quad {\bf lib} & Noterndige zus"atzliche Java-Bibliotheken \\
\quad {\bf schema} & XMLSchema Dateien, die anwendungsunabh"angig sind \\
\end{tabular}
\end{center}
%
\section{Konfiguration zum "Ubersetzten des Kerns}
\section{Compile}
