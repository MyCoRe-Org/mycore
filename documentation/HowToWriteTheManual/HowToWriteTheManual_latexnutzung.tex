\chapter{Die Arbeit mit einem \LaTeX -- System}
\section{Die Installation}
\subsection{Die Installation unter SuSE Linux}
Die Linux-Distribution von SuSE bietet unter 8.x ein recht vollst�ndiges, umfangreiches und aktuelles \LaTeX -- System inklusive einer ganzen Reihe von zus�tzlichen Komponenten.
Folgende Pakete werden empfohlen zu installieren:\\
\begin{itemize}
\item {\bf te\_latex} - Alles zu und um \LaTeX
\item {\bf te\_pdf} - Eine TeX/\LaTeX - Version mit PDF-Ausgabeformat
\item {\bf tex4ht} - Ein TeX/\LaTeX Umwandler
\item {\bf ImageMagick} - Ein m�chtiges Bildverarbeitungs-Tool
\end{itemize}
\section{Verarbeitungskommandos}
\subsection{Unix-Kommandos}
Das Kommando {\tt latex {\it filename.tex}} erzeugt aus den \LaTeX - Quelle ein {\bf .dvi} - File, welches dann entsprechend weiterverarbeitet werden kann.\\[2ex]
Das Kommando {\tt dvips {\it filename.dvi}} generiert aus dem {\bf .dvi} - File ein Postscript-File mit der Endung {\bf .ps}.\\[2ex]
Um mit einem Rutsch direkt PDF-Dokumente zu erzeugen gen�gt das Kommando {\tt pdflatex {\it filename.tex}}.\\[2ex] 
Zus�tzlich haben wir ein Shell Script {\tt convert\_jpeg.sh } unter {\it \$MYCORE\_HOME/bin } gestellt, welches auf Basis von {\bf ImageMagick} von allen jpeg-Files Bounding Boxes zur Verarbeitung unter \LaTeX erzeugt.

\subsubsection{Installieren fehlender Pkete}
\label{sec:installation_pakete}
Die Installation wirde hier am Beispiel des listing Pakets verdeutlicht, ist aber prinzipiell auch auf andere Pakete anwendbar, ggf aber immer die README Datei des Pakets lesen. Wer genaueres wissen m�chte findet Information dazu unter \url{http://www.dante.de/faq/de-tex-faq/html/makros1.html#7}.

\begin{enumerate}
\item Alle Dateien aus \mcrfile{listings.tar.gz} in \mcrpath{temp} Verzeichnis
downloaden und entpacken
\item F�hre \mcrcommand{latex listings.ins} im erstellten
Verzeichnis aus.
\item Erstelle das Verzeichnis listings unter
\mcrpath{/usr/local/share/texmf/tex/latex/} und verschiebe alle \mcrfile{sty} und \mcrfile{cfg}
Dateien vom temp ins Verzeichnis \mcrpath{/usr/local/share/texmf/tex/latex/}
\item \mcrcommand{texhash} als root ausf�hren

\end{enumerate}

\subsection{Die Installation unter Windows}
Prinzipiell l��t sich jede (aktuelle) Distribution verwenden. Sehr gute Erfahrungen habe ich mit Miktex (\url{http://www.miktex.org/index.html}) gemacht, da diese eine umfassende Paketsammlung enth�lt, die Pakete leicht �bers Netz aktualisieren lassen und die Installation sehr einfach ist. \LaTeX Texte k�nnen in jedem beliebigen Texteditor verfasst werden, einige Editoren bieten jedoch hilfreiche Funktionen f�r die Erstellung von \LaTeX -- Dokumenten. (X)emacs wird doch den meisten Windows nutzern fremd sein und daher empfehle ich folgende Editoren WinEdt\footnote{leider nur share-- und keine freeware} (\url{http://www.winedt.com}\footnote{Extensions zu winedt sind zu finden unter \url{www.winedt.org}}) oder texniccenter (\url{http://www.toolscenter.de})
eine weitere Liste ist unter \url{http://www.dante.de/faq/de-tex-faq/html/woher.html#25} einsehbar. Wer sich f�r Winedt entscheidet findet unter \url{http://www.dante.de/help/documentation/miktex} eine gute Installationsanleitung f�r seine komplette \LaTeX Umgebung. Mit entsprechenden �nderungen kann diese Vorlage nat�rlich auch f�r andere Distributionen verwendet werden. 

F�r die Verwendung der MyCoRe Vorlage sind neben der Minimalinstallation folgende Pakete aus dem Paketmanager auszuw�hlen:

\begin{itemize}
\item Zur PDF Erzeugung
\subitem {\bf hyperref} - Alles zu und um \LaTeX
\subitem {\bf thumbpdf} - Eine TeX/\LaTeX - Version mit PDF-Ausgabeformat
\item Tabellen
\subitem {\bf supertabualar}
\item HTML \& XML
\subitem {\bf tex4ht} - Ein TeX/\LaTeX Umwandler
\item Grafik
\subitem {\bf graphics}
\subitem {\bf wrapfig}
\subitem {\bf }
\item Umgebungen
\subitem {\bf float}  
\end{itemize}

F�r die Verwendung der MyCoRe Vorlage werden folgende zus�tzlichen Programme ben�tigt
\begin{itemize}
\item {\bf ImageMagick}\footnote{www.imagemagick.org} - Ein m�chtiges Bildverarbeitungs-Tool

\end{itemize} 

\section{Hilfe bei \LaTeX Fragen}
Die umfassendste Hilfe bei \LaTeX Fragen ist sicherlich im Internet zu finden, wenngleich auch das Auffinden oftmals nicht ganz einfach ist. Eine weitere gute M�glichkeit ist es newsgroups\footnote{\url{http://groups.google.de/}} nach entsprechenden \LaTeX Schl�gw�rtern zu durchsuchen. 
Nat�rlich bieten auch etliche \LaTeX B�cher eine sehr gute Einf�hrung und Hilfestellung bei Problemen. 

