%
%
\section{Download der Beispielanwendung}
%
%
Nachdem Sie den MyCoRe-kern erfolgreich installiert haben, ist nun die Installation der mitgelieferten Beispielanwendung sinnvoll. Hier k�nnen Sie ein erstes gef�hl daf�r gewinnen, wie eine eigene Anwendung gestaltet sein k�nnte. Das MyCoRe Sample wird f�r alle unterst�tzten Systeme �ber das CVS Repository ausgeliefert. Das Holen der aktuellen Version erfolgt mit dem Kommando
\begin{center}
{\tt cvs -d :pserver:anoncvs@server.mycore.de:/cvs checkout mycore-sample-application}
\end{center}
Nach dem erfolgreichen Checkout erhalten Sie folgende Dateistruktur:\\[2ex]
\bottomcaption{Dateistruktur des MyCoRe Samples}
\tablehead{\hline}
\tabletail{\hline}
\begin{supertabular}{|p{5cm}|p{10cm}|}
\hline
{\bf mycore-sample-application} &  Das Root-Verzeichnis des MyCoRe-Samples \\
\quad {\bf bin} & Das Verzeichnis der Shellscripte \\
\qquad build.sh & Shellscript zum Compilieren unter einem UNIX-System \\
\qquad build.cmd & Shellscript zum Compilieren unter einem UNIX-System \\
\qquad setup.sh & Shellscript, welches die Umgebung f�r das Sample setzt \\
\qquad setup.cmd & Shellscript, welches die Umgebung f�r das Sample setzt \\
\quad {\bf config} & Das Verzeichnis der Konfigurationsdateien \\
\qquad ContentStoreSelectionRules.xml & Das Regelwerk zur Speicherung des Contents in die einzelnen Stores. \\
\qquad FileContentTypes.xml & Eine Liste der verschiedenen Content-Typen \\
\qquad MyCoReDemoDC\_....xml & Die Konfigurationsdateien der einzelnen Sample-Metadatentypen. \\
\qquad SearchMask\_....xml & Die Konfigurationsdateien der Suchmasken \\
\qquad editor-...-nbn.xml & Die Konfigurationsdateien f�r die NBN Funktionalit�t. \\
\qquad ibm-web-... & Zus�tzliche Dateien zur Nutzung von IBM WebSphere. \\
\qquad mycore.properties & Das Master-Property-File f�r MyCoRe \\
\qquad mycore.properties.application & Ein Platzhalter-Property-File f�r andere MyCoRe-Anwendungen. \\
\qquad mycore.properties.classification & Das Property-File f�r den Klassifikationsbereich. \\
\qquad mycore.properties.cm7 & Das Property-File f�r den IBM CM 7.x-Bereich. \\
\qquad mycore.properties.cm8 & Das Property-File f�r den IBM CM 8.x-Bereich. \\
\qquad mycore.properties.ifs & Das Property-File f�r den Bereich des Internal File Systems. \\
\qquad mycore.properties.logger & Das Property-File f�r den Logger-Bereich. \\
\qquad mycore.properties.nbn & Das Property-File f�r den NBN-Bereich. \\
\qquad mycore.properties.oai & Das Property-File f�r den OAI-Bereich. \\
\qquad mycore.properties.private & Das Property-File f�r den Bereich, in welchem die meisten Anpassungen der jeweiligen Sample Anwender get�tigt werden m�ssen. \\
\qquad mycore.properties.remote & Das Property-File f�r den Bereich der Remote-Zugriffe. \\
\qquad mycore.properties.user & Das Property-File f�r den User- und Rechteverwaltungs-Bereich. \\
\qquad mycore.properties.xmlsortkeys & Das Property-File f�r den Pr�sentationsbereich. \\
\qquad reservation.xml & ein Konfigurations-File f�r den NBN-Bereich. \\
\qquad {\bf user} & In diesem Verzeichnis sind alle Gruppen, Nutzer und Privilegien abgelegt, welche f�r das Sample ben�tigt werden. \\
\qquad web.xml & Das Konfigurations-File f�r die Servlet-Engine \\
\quad {\bf content} & Hier finden Sie alle Beispieldaten. \\
\qquad {\bf classifications} & Die Daten der Klassifikationen. \\
\qquad {\bf documents} & Die Daten der Dokumente. \\
\qquad {\bf derivates} & Die Daten der Derivate. \\
\qquad {\bf legalentities} & Die Daten der LegalEntities. \\
\qquad {\bf objects} & Die eigentlichen multimedialen Objekte. \\
\quad {\bf schema} & Das Verzeichnis der XML-Schema-Files f�r das Sample. \\
\quad {\bf sources} & Hier sind zus�tzliche Java-Klassen abgelegt, welche nur f�r diese Anwendung g�ltig sind. Die Struktur ist analog dem MyCoRe-Kern. \\
\quad {\bf stylesheets} & Die verwendeten XSLT-Stylesheets der Pr�sentation dieses Samples. \\
\quad build.xml & Konfigurations-File f�r die Arbeit mit ANT \\
\quad license.txt & Das Lizenz-File des MyCoRe-Projektes, bitte lesen Sie dieses File aufmerksam durch, bevor Sie MyCore einsetzen. \\
\hline
\end{supertabular}
