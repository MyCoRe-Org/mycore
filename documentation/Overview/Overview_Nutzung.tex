Das MyCoRe-Softwaresystem ist als Open Source unter der GNU General Public
License (GPL) verf"ugbar.
Damit darf MyCoRe von jedem ver"andert und weitergegeben werden, vorausgesetzt
die Bedingungen der GPL werden befolgt.

Die GNU General Public License ist im Original unter http://www.gnu.org/copyleft/gpl.html
nachzulesen, in vollst"andiger deutscher "Ubersetzung unter http://www.gnu.de/gpl-ger.html.

Die wichtigsten Bedingungen der GPL werden im folgenden kurz zusammen gefasst.
Diese Darstellung der GPL dient nur der groben Orientierung und ist 
\mcridentifier{kein Ersatz} f"ur die oben angegebene vollst"andige GPL.

\S 1. 
Die unver"anderte GPL-Software darf vervielf"altigt und weitergegeben werden, wenn jeder 
Kopie ein entsprechender Haftungsausschluss- und Copyright-Vermerk beigef"ugt wird.

\S 2.
Die Verbreitung ver"anderter GPL-Software bedarf der zus"atzlichen Vermerke, die dar"uber 
Auskunft geben, wer wann was ge"andert hat. 
Au�erdem muss jede neue Software, die ganz oder zum Teil auf GPL-Software Dritter basiert, 
ebenfalls unter der GPL stehen.

\S 3.
Sie d"urfen Software unter den Bedingungen von \S 1 und \S 2 nur dann weitergeben, wenn Sie 
den zugeh"origen Quellcode mitausliefern.

...

\S 6.
Jedem Empf"anger der Software m"ussen die gleichen Rechte und Pflichten zugestanden werden, 
wie Ihnen selbst. 
Einschr"ankungen sind nicht erlaubt.

...

\S 11.
Gegen"uber GPL-Software bestehen keine Gew"ahrleistungsanspr"uche bez"uglich Funktion, 
Qualit"at oder Verwendbarkeit.

\S 12.
Kein Copyright-Inhaber oder Co-Autor kann f"ur Sch"aden die aus der Benutzung oder 
Nichtbenutzbarkeit der Software entstehen, haftbar gemacht werden.

Entsprechend der Empfehlung in der GNU General Public License wird der Kopf jeder Javaklasse
in MyCoRe um einen Hinweis zur GPL erg"anzt.
Dar"uber hinaus wird der Text der GPL als Textdatei \mcrfile{license.txt} mitgeliefert. 



