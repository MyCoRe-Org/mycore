\section{Releases}
\subsection{Release 0.9, Januar 2004}

\mcridentifier{Benutzerverwaltung}\\[0.1cm]
Der in Abschnitt \ref{sec:LeistungsumfangUsermanagement} beschriebene Leistungsumfang wird
von der Benutzerverwaltung voll unterst"utzt, also insbesondere die Funktionen

\begin{itemize}
\item Authentifizierung am System
\item Rollen"ubernahme einzelner Personen auf Basis von Benutzergruppen, denen bestimmte
      Privilegien zugewiesen werden k"onnen
\item Freie Konfigurierbarkeit des Privilegiensystems (ein Satz grundlegender Privilegien
      wird in der Dublin Core Beispielanwendung von MyCoRe mitgeliefert)
\item Anlegen, L"oschen, Aktualisieren von Benutzern, Gruppen und Privilegien
\item Import und Export von Benutzern, Gruppen und Privilegien aus/in XML Dateien
\item Aktivierung/Deaktivierung einzelner Benutzer bzw. Benutzerinnen
\end{itemize}

Die Datenhaltung geschieht auf Basis einer SQL-Datenbank (z.B. IBM DB2, MySQL usw., je
nachdem, in welcher Softwareumgebung das MyCoRe-System betrieben wird).
Durch das Fehlen eines globalen grafischen User-Interfaces k"onnen alle Gesch"aftsprozesse
(mit Ausnahme der Anmeldung) in diesem Release nur "uber die MyCoRe-Kommandozeile
ausgef"uhrt werden.
Die Anmeldung am System geschieht "uber ein Servlet (MCRLoginServlet), auf das von
verschiedenen Stellen der Webanwendung hingewiesen werden kann. 
Dabei ist "uber ein MyCoRe-internes Session-Management gew"ahrleistet, dass die
Anmeldung "uber mehrere Transaktionen bzw. mehrere Anfragen an das System erhalten bleibt.
(Detaillierte Informationen hierzu finden Sie im 'MyCoRe Internal Design Guide'.)

F"ur die zuk"unftig geplanten Funktionen im Subsystem Benutzerverwaltung siehe Abschnitt
\ref{sec:PlanungenUsermanagement}.

\section{Planungen f"ur zuk"unftige Releases}

\subsection{Benutzerverwaltung}
\label{sec:PlanungenUsermanagement}
\mcridentifier{Passwortverschl"usselung}\\[0.1cm]
Derzeit werden die Passw"orter nur im Klartext in der jeweiligen SQL-Datenbank abgelegt.
Ein Verschl"usselungsmechanismus wird eingearbeitet und ist f"ur das Release 1.0 geplant.

\mcridentifier{Grafisches Verwaltungsinterface}\\[0.1cm]
Mit Einf"uhrung der globalen grafischen Benutzerschnittstelle f"ur MyCoRe werden auch
Editoren zur Durchf"uhrung der Gesch"aftsprozesse der Benutzerverwaltung zur
Verf"ugung stehen. Planung: Release 1.0.

\mcridentifier{Einbindung externer Benutzerverwaltungen}\\[0.1cm]
"Ublicherweise gibt es in den Institutionen in denen eine Digitale Bibliothek zum Einsatz kommt 
bereits bestehende Benutzerverwaltungen wie zum Beispiel zentrale LDAP-Server. 
Die MyCoRe-Benutzerverwaltung soll zuk"unftig eine einfache Schnittstelle bieten, um diese 
bereits bestehenden Verwaltungen zumindest f"ur die reine Authentifizierung mit einzubeziehen.
Planung: Release $>$ 1.0.

\mcridentifier{Benutzerprofile}\\[0.1cm]
Das Benutzermanagement soll "uber die reinen administrativen Aufgaben hinaus auch die Verwaltung 
von Benutzerprofilen erm"oglichen. 
Damit soll es Anwendern der digitalen Bibliothek erm"oglicht werden, regelm"a�ig wiederkehrende 
Aufgaben direkt abzurufen, Zwischenergebnisse abzulegen usw.
Planung: Release $>$ 1.0.

\subsection{Verwaltung von Zugriffsrechten}
\label{sec:PlanungenACL}
Der Zugriffsschutz f"ur Objekte soll f"ur Benutzer, Gruppen und/oder IP/Host-Masken realisierbar 
sein. 
Objekte einer Digitalen Bibliothek bestehen aus Metadaten und einzelnen multimedialen Komponenten. 
Die Zugriffskontrolle soll auf die kleinsten Granulen eines Objekts anwendbar sein. 
So sollen also zum Beispiel die Metadaten frei zug"anglich sein, f"ur einzelne Teile aber 
besondere Rechte erforderlich sein. 
Zur Vereinfachung der Arbeit der Autoren soll es m"oglich sein, das Setzen von Zugriffsrechten 
f"ur eine gr"o�ere Anzahl von Objekten zu automatisieren bzw. sie projektspezifisch automatisch 
zu setzen. 
Dar"uber hinaus m"ussen Objekte zeitlich begrenzt mit besonderen Zugriffsrechten versehen werden 
k"onnen. 
Insbesondere muss es m"oglich sein, Dokumente wie zum Beispiel Dissertationen erst ab einem 
bestimmten Zeitpunkt freizugeben.
An der Implementierung dieser Komponente von MyCoRe wird gearbeitet.
Planung: Release 1.0.

\subsection{Digital Rights Management}
\label{sec:PlanungenDRM}
\mcridentifier{Wasserzeichen}\\[0.1cm]
Durch die Verwaltung von Zugriffsrechten kann man das Retrieval von Objekten aus der Digitalen 
Bibliothek kontrollieren, nicht aber die unberechtigte Weiterverwertung nach etwaigem 
Herunterladen eines Objekts. 
Um das Copyright der Autoren f"ur diesen Fall zu gew"ahrleisten, sollen Autoren w"ahlen k"onnen, 
ob ihre Objekte zusammen mit einem digitalen Wasserzeichen ausgeliefert werden sollen. 
Der Signiervorgang soll dabei funktional in den Auslieferungsmechanismus des Repositories 
eingebunden sein, also ohne zus"atzlichen Arbeitsaufwand f"ur die Autoren geschehen.
An der Implementierung dieser Komponente von MyCoRe wird gearbeitet.
Planung: Release $>$ 1.0.








